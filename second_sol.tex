%#!platex main.tex
\section{second solutionの存在 1 --- 命題~\ref{prop:second_1}の証明}

本節と次節で、定理~\ref{thm:second_solution}を証明する。
本節と次節を通し、$0 < \lambda < \bar{\lambda}$とする。

\ref{eq:prob_main}の minimal solution 以外の解
$\bar{u}_\lambda$を見出すために、
以下の方程式\ref{eq:prob_sec}を考察する。
\begin{align}
 \left\{
 \begin{aligned}
   -\Delta v + a v &= b \left( (v + \underline{u}_\lambda)^p - (\underline{u}_\lambda)^p \right)
  & &\text{in~} \Omega, \\
  v &> 0 & &\text{in~} \Omega, \\
  v &= 0 & &\text{on~} \partial\Omega
 \end{aligned}
 \right. \tag*{$(\heartsuit)_\lambda$} \label{eq:prob_sec}
\end{align}

方程式\ref{eq:prob_sec}を考察するために、以下の記号をおく。

\begin{nota}
 \begin{enumerate}[1.]
  \item $\R \times \R \times \Omega$を定義域とする実数値関数$g, G$を
        以下の通りに定める。
        \begin{align}
         g(t, s, x) &= b(x) \left( (t_+ + s)^p - s^p \right) a t_+, 
         \label{eq:def_g} \\
         G(t, s, x) &= \int_0^{t_+} g(t, s, x) dt
         \notag \\
         &= b(x) \left( \frac{1}{p+1} (t_+ + s)^{p+1} - \frac{1}{p+1}
         s^{p+1} - s^p t_+ \right) - \frac{1}{2} a(x) t_+^2.
         \label{eq:def_G}
        \end{align}
        $g(v, \underline{u}_\lambda, x)$を$g(v, \underline{u}_\lambda
        )$と表記する。
        $G(v,\underline{u}_\lambda, x)$を$G(v, \underline{u}_\lambda
        )$と表記する。
  \item $I_{\lambda} \colon H_0^1(\Omega) \to \R$を以下の通りに定める。
        \begin{equation}
         I_\lambda (v) = \frac{1}{2} \int_\Omega \lvert Dv \rvert^2 dx
          - \int_\Omega G(v, \underline{u}_\lambda) dx. \label{eq:def_I}
        \end{equation}
 \end{enumerate}
\end{nota}

\ref{eq:prob_sec}の考察を始める前に、
\ref{eq:prob_main}と\ref{eq:prob_sec}の関係、および、
\ref{eq:prob_sec}と$I_\lambda$の関係を明らかにする。

\begin{lem}
 \begin{enumerate}[1.]
  \item 以下の(1), (2)は同値である。
        \begin{enumerate}[(1)]
         \item \ref{eq:prob_main}の minimal solution $\underline{u}_\lambda$
               以外の弱解$\bar{u}_\lambda \in H_0^1(\Omega)$が存在する。
         \item \ref{eq:prob_sec}の
               弱解$v \in H_0^1(\Omega)$が存在する。
        \end{enumerate}
  \item $v \in H_0^1(\Omega)$は
        \eqref{eq:def_I}で定まる$I_\lambda$の臨界点であると仮定する。
        このとき、$v$は\ref{eq:prob_sec}の弱解である。
 \end{enumerate}
\end{lem}

ここで次の記号を置く。

\begin{nota} \label{nota:S_def}
 $V \subset \R^N$を領域とする。
 \begin{equation}
  S = \inf_{u \in H^1_0(V), u \not \equiv 0}
 \frac{\left\| Du \right\|_{L^2(V)}^2}{\left\| u
                                       \right\|_{L^{p+1}(V)}^2}  
 \label{eq:S_def}
 \end{equation}
 と定める。
\end{nota}

$S$は$V$には依存しないことが知られている。

次の2つの命題を証明することにより、
定理~\ref{thm:second_solution}を証明する。

\begin{prop} \label{prop:second_1}
 $0 < \lambda < \bar{\lambda}$とする。
 $v \geq 0 ~\tin \Omega$、$v_0 \not \equiv 0$、かつ、
 \begin{equation}
  \sup_{t > 0} I_\lambda (tv_0) < = \frac{1}{NM^{(n-2)/2}} S^{N/2} 
   \label{eq:ineq_S}
 \end{equation}
 をみたす$v_0 \in H_0^1(\Omega)$が存在することを仮定する。
 このとき、\ref{eq:prob_sec}の弱解$v \in H_0^1(\Omega)$が存在する。
\end{prop}

\begin{prop} \label{prop:second_2}
 定理~\ref{thm:second_solution}の仮定のもとで、$v_0 \geq 0 ~\tin
 \Omega$、$v_0 \not \equiv 0$、および、\eqref{eq:ineq_S}をみたす
 $v_0 \in H_0^1(\Omega)$が存在する。
\end{prop}

命題~\ref{prop:second_1}の証明は本節、
命題~\ref{prop:second_2}の証明は次節でおこなう。

\begin{lem} \label{lem:conv}
 
\end{lem}

\begin{lem} \label{lem:mountain_dec}
 
\end{lem}

% Local Variables:
% mode: yatex
% TeX-master: "main.tex"
% End: