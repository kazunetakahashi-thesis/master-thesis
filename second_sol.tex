%#!platex main.tex
\section{second solutionの存在 1 --- 命題~\ref{prop:second_1}の証明}

\subsection{second solution を求めるための方針}

本節と次節で、定理~\ref{thm:second_solution}を証明する。
本節と次節を通し、$0 < \lambda < \bar{\lambda}$とする。
方程式\ref{eq:prob_sec}を考察するために、以下の記号をおく。

\begin{nota}
 \begin{enumerate}[1.]
  \item $\R \times \R \times \Omega$を定義域とする実数値関数$g, G$を
        以下の通りに定める。
        \begin{align}
         g(t, s, x) &= b(x) \left( (t_+ + s)^p - s^p \right) - a t_+, 
         \label{eq:def_g} \\
         G(t, s, x) &= \int_0^{t_+} g(t, s, x) dt
         \notag \\
         &= b(x) \left( \frac{1}{p+1} (t_+ + s)^{p+1} - \frac{1}{p+1}
         s^{p+1} - s^p t_+ \right) - \frac{1}{2} a(x) t_+^2.
         \label{eq:def_G}
        \end{align}
        $g(v, \underline{u}_\lambda, x)$を$g(v, \underline{u}_\lambda
        )$と表記する。
        $G(v,\underline{u}_\lambda, x)$を$G(v, \underline{u}_\lambda
        )$と表記する。
  \item $I_{\lambda} \colon H_0^1(\Omega) \to \R$を以下の通りに定める。
        \begin{equation}
         I_\lambda (v) = \frac{1}{2} \int_\Omega \lvert Dv \rvert^2 dx
          - \int_\Omega G(v, \underline{u}_\lambda) dx. \label{eq:def_I}
        \end{equation}
        $I_\lambda$のフレッシェ微分を$I_\lambda^\prime$と表記する。
 \end{enumerate}
\end{nota}

\ref{eq:prob_sec}の考察を始める前に、
\ref{eq:prob_main}と\ref{eq:prob_sec}の関係、および、
\ref{eq:prob_sec}と$I_\lambda$の関係を明らかにする。

\begin{lem}
 \begin{enumerate}[1.]
  \item 以下の(1), (2)は同値である。
        \begin{enumerate}[(1)]
         \item \ref{eq:prob_main}の minimal solution $\underline{u}_\lambda$
               以外の弱解$\bar{u}_\lambda \in H_0^1(\Omega)$が存在する。
         \item \ref{eq:prob_sec}の
               弱解$v \in H_0^1(\Omega)$が存在する。
        \end{enumerate}
  \item $v \in H_0^1(\Omega)$は
        \eqref{eq:def_I}で定まる$I_\lambda$の臨界点であると仮定する。
        このとき、$v$は\ref{eq:prob_sec}の弱解である。
 \end{enumerate}
\end{lem}

ここで次の記号を置く。

\begin{nota} \label{nota:S_def}
 $V \subset \R^N$を領域とする。
 \begin{equation}
  S = \inf_{u \in H^1_0(V), u \not \equiv 0}
 \frac{\left\| Du \right\|_{L^2(V)}^2}{\left\| u
                                       \right\|_{L^{p+1}(V)}^2}  
 \label{eq:S_def}
 \end{equation}
 と定める。
\end{nota}

$S$は$V$には依存しないことが知られている。例えば~\cite{田中200808}の定
理@@@@@@を参照されたい。

次の2つの命題を証明することにより、
定理~\ref{thm:second_solution}を証明する。

\begin{prop} \label{prop:second_1}
 $0 < \lambda < \bar{\lambda}$とする。
 $v \geq 0 ~\tin \Omega$、$v_0 \not \equiv 0$、かつ、
 \begin{equation}
  \sup_{t > 0} I_\lambda (tv_0) < \frac{1}{NM_1^{(N-2)/2}} S^{N/2} 
   \label{eq:ineq_S}
 \end{equation}
 をみたす$v_0 \in H_0^1(\Omega)$が存在することを仮定する。
 このとき、\ref{eq:prob_sec}の弱解$v \in H_0^1(\Omega)$が存在する。
\end{prop}

\begin{prop} \label{prop:second_2}
 定理~\ref{thm:second_solution}の仮定のもとで、$v_0 \geq 0 ~\tin
 \Omega$、$v_0 \not \equiv 0$、および、\eqref{eq:ineq_S}をみたす
 $v_0 \in H_0^1(\Omega)$が存在する。
\end{prop}

命題~\ref{prop:second_1}の証明は本節、
命題~\ref{prop:second_2}の証明は次節でおこなう。

\subsection{命題~\ref{prop:second_1}の証明}

本小節では、命題~\ref{prop:second_1}の証明を与える。

まずは、以降の議論で使用する積分の極限について議論する。

\begin{nota}
 関数$H, h, H^\prime, h^\prime, G^\prime, g^\prime$を以下の通りに定める。
 \begin{align*}
  H(t, s, x) &= G(t, s, x) - \frac{1}{p+1}b(x) t_+ ^{p+1}, \\
  h(t, s, x) &= g(t, s, x) - b(x) t_+ ^{p}, \\
  H^\prime (t, s, x) &= H(t, s, x) + \frac{1}{2}a(x) t_+^2, \\
  h^\prime (t, s, x) &= h(t, s, x) + a(x) t_+.
  G^\prime (t, s, x) &= G(t, s, x) + \frac{1}{2}a(x) t_+^2, \\
  g^\prime (t, s, x) &= g(t, s, x) + a(x) t_+.
 \end{align*}
 $H(v, \underline{u}_\lambda, x)$を
 $H(v, \underline{u}_\lambda)$と表記する。
 $h(v, \underline{u}_\lambda)$、
 $H^\prime(v, \underline{u}_\lambda)$、
 $h^\prime(v, \underline{u}_\lambda)$、
 $G^\prime(v, \underline{u}_\lambda)$、
 $g^\prime(v, \underline{u}_\lambda)$も全て同様である。
\end{nota}

\begin{lem} \label{lem:conv}
 $v \in H_0^1(\Omega)$とし、$\{ v_k \}_{k = 0}^\infty$を
 $H_0^1(\Omega)$の有界列とする。$k \to \infty$のとき、
 $v_k \to v ~ae ~tin \Omega$と仮定する。このとき、$k \to \infty$
 とすると、以下が成立する。
 \begin{align}
  \int_\Omega H(v_k, \underline{u}_\lambda)dx &\xrightarrow{ \mbox{ ~
  } } 
  \int_\Omega H(v, \underline{u}_\lambda)dx, \label{eq:conv_H} \\
  \int_\Omega h(v_k, \underline{u}_\lambda)v_k dx &\xrightarrow{
  \mbox{ ~ } } 
  \int_\Omega h(v, \underline{u}_\lambda)v_k dx. \label{eq:conv_h} \\
  \intertext{また、任意の$\psi \in H_0^1(\Omega)$に対し、}
  \int_\Omega g(v_k, \underline{u}_\lambda)\psi dx &\xrightarrow{
  \mbox{ ~ } } 
  \int_\Omega g(v, \underline{u}_\lambda)\psi dx \label{eq:conv_g}
 \end{align}
 が成立する。
\end{lem}

\begin{proof}
 まず、\eqref{eq:conv_H}を証明する。
 $\{ v_k \}$は$H_0^1(\Omega)$の有界列で、$v$に$\Omega$上
 ほとんどいたるところ
 収束するから、
 必要ならば部分列をとることにより、$k \to \infty$とすると、
 以下が成立する。
 \begin{align}
  v_k \xrightharpoonup{ \mbox{ ~ } } v & \ \ \text{weakly~} \tin
  H_0^1(\Omega), \label{eq:minimal_vk_weakly} \\
  v_k \xrightarrow{ \mbox{ ~ } } v & \ \ \tin L^q(\Omega) \ \
   (q < p+1), \label{eq:minimal_vk_Lq} \\
  v_k \xrightarrow{ \mbox{ ~ } } v & \ \ \ae \tin \Omega. 
    \label{eq:minimal_vk_ae}
 \end{align}
 \eqref{eq:minimal_vk_Lq}より、
 \[
  \int_\Omega \frac{1}{2} a v_k^2 dx \xrightarrow{k \to \infty}
 \int_\Omega \frac{1}{2}av^2 dx
 \]
 がわかるので、\eqref{eq:conv_H}を示すためには、
 \begin{equation}
  \int_\Omega H^\prime(v_k, \underline{u}_\lambda)dx \xrightarrow{ k \to \infty } 
  \int_\Omega H^\prime(v, \underline{u}_\lambda)dx \label{eq:conv_Hprime} \\  
 \end{equation}
 を示せば十分である。以下\eqref{eq:conv_Hprime}を示す。
 $t, s \geq 0$のとき、次式が成立する。
 \begin{align}
  H^\prime(t, s, x) &= b(x) \left( \frac{1}{p+1}(t+s)^{p+1} -
  \frac{1}{p+1} s^{p+1} - s^p t - \frac{1}{p+1} t^{p+1} \right) \notag \\
  & \leq b(x) \left( \frac{1}{p+1}(t+s)^{p+1} - \frac{1}{p+1}s^{p+1} -
  \frac{1}{p+1} t^{p+1}\right) \notag \\
  & \leq b(x) \int_0^t \left( (\tau + s)^p - \tau^p \right)
  d\tau. \label{eq:dtau} 
 \end{align}
 ここで、$x \geq 0$に対し、$x \mapsto x^p$は下に凸であるから、
 $(\tau + s)^p - \tau^p \leq p(\tau + s)^{p-1} s$である。
 さらに、
 \begin{equation}
  (\tau + s)^{p-1} \leq (2 \max\{\tau , s\})^{p-1} = 2^{p-1} \max \{
   \tau^{p-1} + s^{p-1} \} \leq C (\tau^{p-1} + s^{p-1}) \label{eq:taus2p-1}
 \end{equation}
 であるから、次が得られる。
 \begin{equation}
  H^\prime(t, s, x) \leq C b \int_0^t (\tau^{p-1} + s^{p-1}) s d \tau
   \leq C b ( t^{p-1} s + s^{p-1} t). \label{eq:fromlemma}
 \end{equation}
 \eqref{eq:fromlemma}の証明は、\cite{MR2317491}~の Lemma~C.4を参考にし
 た。さらにヤングの不等式を適用すると、任意の$\epsilon > 0$に対し、
 $C > 0$が存在し、$s, t \geq 0$に対し、
 $H^\prime(t, s, x) \leq b( \epsilon t^{p+1} + C s^{p+1})$
 が成立する。ゆえに、次式が得られる。
 \begin{equation}
  \left\lvert H^\prime(v_k, \underline{u}_\lambda) - H^\prime(v,
   \underline{u}_\lambda ) \right\rvert \leq b \left( \epsilon
   (v_\epsilon)_+^{p+1}  + \epsilon v_+^{p+1} + C
   \underline{u}_\lambda^{p+1} \right). \label{eq:conv_H_abs}
 \end{equation}
 そこで、
 \begin{equation}
  W_{\epsilon, k} = \left( \left\lvert H^\prime(v_k,
                            \underline{u}_\lambda) - H^\prime(v,
                            \underline{u}_\lambda) \right\rvert
  -\epsilon b (v_k)_+^{p+1}
                    \right)_+ \label{eq:conv_H_W}
 \end{equation}
 とおくと、$k \to \infty$のとき、$W_{\epsilon, k} \to 0 ~\ae ~\tin
 \Omega$である。また、\eqref{eq:conv_H_abs}より、
 $\lvert W_{\epsilon, k} \rvert \leq b \left( \epsilon v_+^{p+1} + C
 \underline{u}_\lambda^{p+1} \right)$であり、この右辺は可積分である。
 したがって、優収束定理により、
 \[
  \lim_{k \to \infty} \int_\Omega  W_{\epsilon, k}(x) dx = 0
 \]
 である。さて、$\{v_k \}$は$H_0^1(\Omega)$の有界列であった。
 $H_0^1(\Omega) \subset L^{p+1}(\Omega)$のソボレフ不等式も
 考慮すると、
 \[
   \int_\Omega b ( v_k )_+^{p+1} dx \leq C
 \]
 をみたす$k$によらない$C > 0$が存在する。
 \eqref{eq:conv_H_W}より、
 \[
  \int_\Omega \left\lvert H(v_k, \underline{u}_\lambda) - H(v,
 \underline{u}_\lambda ) \right\rvert dx \leq \int_\Omega W_{\epsilon,
 k} (x)dx + \epsilon \int_\Omega b(v_k)_+^{p+1} dx
 \]
 であるから、$k \to \infty$の上極限をとると、次式が得られる。
 \[
 \limsup_{k \to \infty} \int_\Omega \left\lvert H(v_k, \underline{u}_\lambda) - H(v,
 \underline{u}_\lambda ) \right\rvert dx \leq C \epsilon
 \]
 $C > 0$は$k, \epsilon$によらず、$\epsilon > 0$は任意であるから、
 このことは
 \[
  \lim_{k \to \infty}
 \int_\Omega \left\lvert H(v_k, \underline{u}_\lambda) - H(v,
 \underline{u}_\lambda ) \right\rvert dx = 0
 \]
 と同値である。ゆえに\eqref{eq:conv_H}が成立する。
 以上の証明は、直接は~\cite{MR2886160}の Lemma~3.1 を参考にしているが、
 ~\cite{MR699419}のアイデアを参考にした。
 
 \eqref{eq:conv_h}も\eqref{eq:conv_H}と同様に証明される。
 \eqref{eq:conv_h}を示すためには、やはり
 \[
  \int_\Omega h^\prime(v_k, \underline{u}_\lambda) v_k dx
 \xrightarrow{k \to \infty} \int_\Omega h^\prime (v,
 \underline{u}_\lambda ) v dx 
 \]
 を示せば十分である。$t, s \geq 0$に対し、
 \[
  h(t, s, x) t \leq C b(x) (t^p s + s^p t)
 \]
 がしたがうため、\eqref{eq:conv_h}と同様に\eqref{eq:conv_H}も得られる。
 
 最後に\eqref{eq:conv_g}を証明する。\eqref{eq:minimal_vk_weakly}より、
 \[
  \int_\Omega a v_k \psi dx \xrightarrow{k \to \infty} \int_\Omega
 av\psi dx
 \]
 であるから、\eqref{eq:conv_g}を示すためには、
 \[
  \int_\Omega g^\prime(v_k, \underline{u}_\lambda)\psi dx \xrightarrow{
  k \to \infty } 
  \int_\Omega g^\prime(v, \underline{u}_\lambda)\psi dx
 \]
 を示せば十分である。
 \eqref{eq:dtau}、\eqref{eq:taus2p-1}と同様にすれば、
 $s, t, r \geq 0$に対し、次式がしたがう。
 \[
  g^\prime(t, s, x)r = b(x) \left( (t+s)^p - s^p \right) r \leq C b(x)
 \left( t^pr + s^p r \right).
 \]
 ヤングの不等式を$2$回使用すると、$\epsilon > 0$に対し、$C, C^\prime >
 0$が存在し、$s, t, r \geq 0$に対し、
 \[
  t^p r + s^p r \leq \epsilon t^{p+1} + C r^{p+1} + s^p r \leq
 \epsilon t^{p+1} + C^\prime (r^{p+1} + s^{p+1})
 \]
 が成立する。ゆえに、$g^\prime$は
 \[
  g^\prime(t, s, x)r \leq b \left( \epsilon t^{p+1} + C ( r^{p+1} +
 s^{p+1}) \right)
 \]
 と評価される。したがって、次式が成立する。
 \[  
  \left\lvert g^\prime(v_k, \underline{u}_\lambda)\psi - g^\prime(v,
   \underline{u}_\lambda)\psi \right\rvert \leq b \left( \epsilon
   (v_\epsilon)_+^{p+1}  + \epsilon v_+^{p+1} + C
   (\underline{u}_\lambda^{p+1} + \lvert \psi \rvert^{p+1})
                                 \right).
 \]
 そこで、$\tilde{W}_{\epsilon, k}$を
\[  \tilde{W}_{\epsilon, k} = 
   \left( \left\lvert g^\prime(v_k, \underline{u}_\lambda)\psi - g^\prime(v,
    \underline{u}_\lambda)\psi \right\rvert
  -\epsilon b (v_k)_+^{p+1}
          \right)_+ \]
 と定める。以降は、\eqref{eq:conv_H}の証明と同様に、\eqref{eq:conv_g}
 が示される。\qedhere
\end{proof}

命題~\ref{prop:second_1}を証明には、
$(\mathrm{PS})$条件を課さない峠の定理~\cite{MR0370183}を使用する。
その後、$I_\lambda$の
$(\mathrm{PS})_c$条件が必要となる。
$I_\lambda$の
$(\mathrm{PS})_c$条件を調べる準備として、
$I_\lambda$についてのパレ・スメイル列が
$H_0^1(\Omega)$の有界列であることを証明する。

\begin{lem} \label{lem:PS_seq}
 $\{ v_k \}_{k=0}^\infty$は、$I_\lambda$についての
 パレ・スメイル列であるとする。すなわち、
\end{lem}

\begin{proof}
 
\end{proof}

補題~\ref{lem:PS_seq}を用いて、
$I_\lambda$の$(\mathrm{PS})_c$条件を調べる。

\begin{lem}
 $0 < c < S^{N/2}/NM_1^{(N-2)/2}$とする。このとき、
 $I_\lambda$は$(\mathrm{PS})_c$条件をみたす。すなわち、
 次の条件(i), (ii)をみたす$H_0^1(\Omega))$の点列
 $\{ v_k \}_{k = 0}^\infty$は、収束する部分列をもつ。
 \begin{enumerate}[(i)]
  \item $\displaystyle \lim_{k \to \infty} I_\lambda (v_k) = c$、
  \item $\displaystyle I^\prime_\lambda (v_k) \xrightarrow{k \to
        \infty} 0 ~\tin H^{-1}(\Omega)$。
 \end{enumerate}
\end{lem}

\begin{proof}
 
\end{proof}

続いて、
$(\mathrm{PS})$条件を課さない峠の定理の仮定がみたされていることを
確認する。

\begin{lem}
 
\end{lem}

命題~\ref{prop:second_1}を証明する最後の準備として、次の補題を証明する。

\begin{lem} \label{lem:mountain_dec}
 
\end{lem}

\begin{proof}
 
\end{proof}

$(\mathrm{PS})$条件を課さない峠の定理を用いて、命題
~\ref{prop:second_1}を証明する。

\begin{proof}[命題~\ref{prop:second_1}]
 
\end{proof}

% Local Variables:
% mode: yatex
% TeX-master: "main.tex"
% End: