%#!platex main.tex
\section{second solutionの存在 1 --- 命題~\ref{prop:second_1}の証明} \label{sec:second_sol}

\subsection{second solution を求めるための方針}

本節と次節で、定理~\ref{thm:second_solution}を証明する。
本節と次節を通し、$0 < \lambda < \bar{\lambda}$とする。
方程式\ref{eq:prob_sec}を考察するために、以下の記号をおく。

\begin{nota}
 \begin{enumerate}[1.]
  \item $\R \times \R \times \Omega$を定義域とする実数値関数$g, G$を
        以下の通りに定める。
        \begin{align}
         g(t, s, x) &= b(x) \left( (t_+ + s)^p - s^p \right) - a t_+, 
         \label{eq:def_g} \\
         G(t, s, x) &= \int_0^{t_+} g(t, s, x) dt
         \notag \\
         &= b(x) \left( \frac{1}{p+1} (t_+ + s)^{p+1} - \frac{1}{p+1}
         s^{p+1} - s^p t_+ \right) - \frac{1}{2} a(x) t_+^2.
         \label{eq:def_G}
        \end{align}
        $g(v, \underline{u}_\lambda, x)$を$g(v, \underline{u}_\lambda
        )$と表記する。
        $G(v,\underline{u}_\lambda, x)$を$G(v, \underline{u}_\lambda
        )$と表記する。
  \item $I_{\lambda} \colon H_0^1(\Omega) \to \R$を以下の通りに定める。
        \begin{equation}
         I_\lambda (v) = \frac{1}{2} \int_\Omega \lvert Dv \rvert^2 dx
          - \int_\Omega G(v, \underline{u}_\lambda) dx. \label{eq:def_I}
        \end{equation}
        $I_\lambda$のフレッシェ微分を$I_\lambda^\prime$と表記する。
 \end{enumerate}
\end{nota}

\ref{eq:prob_sec}の考察を始める前に、
\ref{eq:prob_main}と\ref{eq:prob_sec}の関係、および、
\ref{eq:prob_sec}と$I_\lambda$の関係を明らかにする。

\begin{lem} \label{lem:rel_heart_spade}
 \begin{enumerate}[1.]
  \item 以下の(1), (2)は同値である。
        \begin{enumerate}[(1)]
         \item \ref{eq:prob_main}の minimal solution $\underline{u}_\lambda$
               以外の弱解$\bar{u}_\lambda \in H_0^1(\Omega)$が存在する。
         \item \ref{eq:prob_sec}の
               弱解$v \in H_0^1(\Omega)$が存在する。
        \end{enumerate}
  \item $v \in H_0^1(\Omega)$は
        \eqref{eq:def_I}で定まる$I_\lambda$の臨界点であると仮定する。
        このとき、$v$は\ref{eq:prob_sec}の弱解である。
 \end{enumerate}
\end{lem}

\begin{proof}
 \begin{enumerate}[1.] \sage
  \item \ulinej{(1){$\Rightarrow$}(2)}:
        $v = \bar{u}_\lambda - \underline{u}_\lambda$とする。
        $\underline{u}_\lambda$は\ref{eq:prob_main}のminimal solution
        であるから、$v \geq 0 ~\tin \Omega$である。
        そこで、
        \begin{align*}
         -\Delta \bar{u}_\lambda + a \bar{u}_\lambda &= b
         \bar{u}_\lambda^p + \lambda f, \\
         -\Delta \underline{u}_\lambda + a \underline{u}_\lambda &= b
         \underline{u}_\lambda^p + \lambda f
        \end{align*}
        の両辺を引くと、
        \[
        -\Delta v + a v = b \left( (\underline{u}_\lambda + v)^p -
        \underline{u}_\lambda^p \right)
        \]
        が得られる。この右辺は非負である。$a \geq \kappa > -\kappa_1$であ
        るから、強最大値原理より、$v > 0 ~\tin \Omega$である。
        以上より、$v \in H_0^1(\Omega)$は\ref{eq:prob_sec}の弱解である。

        \ulinej{(2){$\Rightarrow$}(1)}:$\bar{u}_\lambda = v +
        \underline{u}_\lambda$とすれば、$\bar{u}_\lambda$は
        \ref{eq:prob_main}の弱解である。
  \item $I_\lambda$は$C^1$級であり、そのフレッシェ微分は、
        $u \in H_0^1(\Omega)$、$\psi \in H_0^1(\Omega)$として、
        \[
         I^\prime_\lambda(u)\psi = \int_\Omega \left( Dv \cdot D\psi -
        g(v, \underline{u}_\lambda) \psi \right) dx.
        \]
        と表される。$v \in H_0^1(\Omega)$は$I_\lambda$の
        臨界点であるから、$I^\prime_\lambda(v) = 0$である。
        すなわち、
        \begin{equation}
         \int_\Omega \left( Dv \cdot D\psi - g(v,
                      \underline{u}_\lambda)\psi \right) dx = 0
         \label{eq:weak_sol_of_heart}        
        \end{equation}
        が成立する。この$\psi \in H_0^1(\Omega)$は任意であるから、
        $v \in H_0^1(\Omega)$は
        \begin{align}
         \left\{
          \begin{aligned}
           -\Delta v + a v &= b \left( (v_+ + \underline{u}_\lambda)^p -
           (\underline{u}_\lambda)^p \right) 
           & &\text{in~} \Omega, \\
           v &= 0 & &\text{on~} \partial\Omega
          \end{aligned}
         \right.
        \end{align}
        の弱解である。$(v_+ + \underline{u}_\lambda)^p -
           (\underline{u}_\lambda)^p \geq 0 ~\tin \Omega$、
        $a \geq \kappa > -\kappa_1$より、強最大値原理から、
        $v > 0 ~\tin \Omega$が従う。ゆえに$v \in H_0^1(\Omega)$
        は\ref{eq:prob_sec}の
        弱解である。 \qedhere
 \end{enumerate}
\end{proof}

ここで次の記号を置く。

\begin{defn} \label{defn:S_def}
 $V \subset \R^N$を領域とする。
 \begin{equation}
  S = \inf_{u \in H^1_0(V), u \not \equiv 0}
 \frac{\left\| Du \right\|_{L^2(V)}^2}{\left\| u
                                       \right\|_{L^{p+1}(V)}^2}  
 \label{eq:S_def}
 \end{equation}
 を{\bf ソボレフ最良定数}という。
\end{defn}

$S$は$V$には依存しないことが知られている。例えば~\cite{田中200808}の定
理~2.31~(i) を参照されたい。

次の2つの命題を証明することにより、
定理~\ref{thm:second_solution}を証明する。

\begin{prop} \label{prop:second_1}
 $0 < \lambda < \bar{\lambda}$とする。
 $v \geq 0 ~\tin \Omega$、かつ、
 \begin{equation}
  \int_\Omega b v^{p+1} dx > 0, \label{eq:int_omega_bvp+1}  
 \end{equation}
 かつ、
 \begin{equation}
  \sup_{t > 0} I_\lambda (tv_0) < 
   \frac{1}{N\left\| b
             \right\|_{L^\infty(\Omega)} ^{(N-2)/2}} S^{N/2} 
   \label{eq:ineq_S}
 \end{equation}
 をみたす$v_0 \in H_0^1(\Omega)$が存在することを仮定する。
 このとき、\ref{eq:prob_sec}の弱解$v \in H_0^1(\Omega)$が存在する。
\end{prop}

\begin{prop} \label{prop:second_2}
 定理~\ref{thm:second_solution}の仮定のもとで、$v_0 \geq 0 ~\tin
 \Omega$、\eqref{eq:int_omega_bvp+1}、
 および、\eqref{eq:ineq_S}をみたす
 $v_0 \in H_0^1(\Omega)$が存在する。
\end{prop}

命題~\ref{prop:second_1}の証明は本節、
命題~\ref{prop:second_2}の証明は次節でおこなう。

\subsection{命題~\ref{prop:second_1}の証明}

本小節では、命題~\ref{prop:second_1}の証明を与える。

まずは、以降の議論で使用する積分の極限について議論する。

\begin{nota}
 関数$H, h, H^\prime, h^\prime, G^\prime, g^\prime$を以下の通りに定める。
 \begin{align*}
  H(t, s, x) &= G(t, s, x) - \frac{1}{p+1}b(x) t_+ ^{p+1}, \\
  h(t, s, x) &= g(t, s, x) - b(x) t_+ ^{p}, \\
  H^\prime (t, s, x) &= H(t, s, x) + \frac{1}{2}a(x) t_+^2, \\
  h^\prime (t, s, x) &= h(t, s, x) + a(x) t_+, \\
  G^\prime (t, s, x) &= G(t, s, x) + \frac{1}{2}a(x) t_+^2, \\
  g^\prime (t, s, x) &= g(t, s, x) + a(x) t_+.
 \end{align*}
 $H(v, \underline{u}_\lambda, x)$を
 $H(v, \underline{u}_\lambda)$と表記する。
 $h(v, \underline{u}_\lambda)$、
 $H^\prime(v, \underline{u}_\lambda)$、
 $h^\prime(v, \underline{u}_\lambda)$、
 $G^\prime(v, \underline{u}_\lambda)$、
 $g^\prime(v, \underline{u}_\lambda)$も全て同様である。
\end{nota}

\begin{lem} \label{lem:conv}
 $v \in H_0^1(\Omega)$とし、$\{ v_k \}_{k = 0}^\infty$を
 $H_0^1(\Omega)$の有界列とする。$k \to \infty$のとき、
 $v_k \to v ~ae ~tin \Omega$と仮定する。このとき、$k \to \infty$
 とすると、以下が成立する。
 \begin{align}
  \int_\Omega H(v_k, \underline{u}_\lambda)dx &\xrightarrow{ \mbox{ ~
  } } 
  \int_\Omega H(v, \underline{u}_\lambda)dx, \label{eq:conv_H} \\
  \int_\Omega h(v_k, \underline{u}_\lambda)v_k dx &\xrightarrow{
  \mbox{ ~ } } 
  \int_\Omega h(v, \underline{u}_\lambda)v_k dx. \label{eq:conv_h} \\
  \intertext{また、任意の$\psi \in H_0^1(\Omega)$に対し、}
  \int_\Omega g(v_k, \underline{u}_\lambda)\psi dx &\xrightarrow{
  \mbox{ ~ } } 
  \int_\Omega g(v, \underline{u}_\lambda)\psi dx \label{eq:conv_g}
 \end{align}
 が成立する。
\end{lem}

\begin{proof}
 まず、\eqref{eq:conv_H}を証明する。
 $\{ v_k \}$は$H_0^1(\Omega)$の有界列で、$v$に$\Omega$上
 ほとんどいたるところ
 収束するから、
 必要ならば部分列をとることにより、$k \to \infty$とすると、
 以下が成立する。
 \begin{align}
  v_k \xrightharpoonup{ \mbox{ ~ } } v & \ \ \text{weakly~} \tin
  H_0^1(\Omega), \label{eq:minimal_vk_weakly} \\
  v_k \xrightarrow{ \mbox{ ~ } } v & \ \ \tin L^q(\Omega) \ \
   (q < p+1), \label{eq:minimal_vk_Lq} \\
  v_k \xrightarrow{ \mbox{ ~ } } v & \ \ \ae \tin \Omega. 
    \label{eq:minimal_vk_ae}
 \end{align}
 \eqref{eq:minimal_vk_Lq}より、
 \[
  \int_\Omega \frac{1}{2} a v_k^2 dx \xrightarrow{k \to \infty}
 \int_\Omega \frac{1}{2}av^2 dx
 \]
 がわかるので、\eqref{eq:conv_H}を示すためには、
 \begin{equation}
  \int_\Omega H^\prime(v_k, \underline{u}_\lambda)dx \xrightarrow{ k \to \infty } 
  \int_\Omega H^\prime(v, \underline{u}_\lambda)dx \label{eq:conv_Hprime} \\  
 \end{equation}
 を示せば十分である。以下\eqref{eq:conv_Hprime}を示す。
 $t, s \geq 0$のとき、次式が成立する。
 \begin{align}
  H^\prime(t, s, x) &= b(x) \left( \frac{1}{p+1}(t+s)^{p+1} -
  \frac{1}{p+1} s^{p+1} - s^p t - \frac{1}{p+1} t^{p+1} \right) \notag \\
  & \leq b(x) \left( \frac{1}{p+1}(t+s)^{p+1} - \frac{1}{p+1}s^{p+1} -
  \frac{1}{p+1} t^{p+1}\right) \notag \\
  & \leq b(x) \int_0^t \left( (\tau + s)^p - \tau^p \right)
  d\tau. \label{eq:dtau} 
 \end{align}
 ここで、$x \geq 0$に対し、$x \mapsto x^p$は下に凸であるから、
 $(\tau + s)^p - \tau^p \leq p(\tau + s)^{p-1} s$である。
 さらに、
 \begin{equation}
  (\tau + s)^{p-1} \leq (2 \max\{\tau , s\})^{p-1} = 2^{p-1} \max \{
   \tau^{p-1} + s^{p-1} \} \leq C (\tau^{p-1} + s^{p-1}) \label{eq:taus2p-1}
 \end{equation}
 であるから、次が得られる。
 \begin{equation}
  H^\prime(t, s, x) \leq C b \int_0^t (\tau^{p-1} + s^{p-1}) s d \tau
   \leq C b ( t^{p-1} s + s^{p-1} t). \label{eq:fromlemma}
 \end{equation}
 \eqref{eq:fromlemma}の証明は、\cite{MR2317491}~の Lemma~C.4を参考にし
 た。さらにヤングの不等式を適用すると、任意の$\epsilon > 0$に対し、
 $C > 0$が存在し、$s, t \geq 0$に対し、
 $H^\prime(t, s, x) \leq b( \epsilon t^{p+1} + C s^{p+1})$
 が成立する。ゆえに、次式が得られる。
 \begin{equation}
  \left\lvert H^\prime(v_k, \underline{u}_\lambda) - H^\prime(v,
   \underline{u}_\lambda ) \right\rvert \leq b \left( \epsilon
   (v_\epsilon)_+^{p+1}  + \epsilon v_+^{p+1} + C
   \underline{u}_\lambda^{p+1} \right). \label{eq:conv_H_abs}
 \end{equation}
 そこで、
 \begin{equation}
  W_{\epsilon, k} = \left( \left\lvert H^\prime(v_k,
                            \underline{u}_\lambda) - H^\prime(v,
                            \underline{u}_\lambda) \right\rvert
  -\epsilon b (v_k)_+^{p+1}
                    \right)_+ \label{eq:conv_H_W}
 \end{equation}
 とおくと、$k \to \infty$のとき、$W_{\epsilon, k} \to 0 ~\ae ~\tin
 \Omega$である。また、\eqref{eq:conv_H_abs}より、
 $\lvert W_{\epsilon, k} \rvert \leq b \left( \epsilon v_+^{p+1} + C
 \underline{u}_\lambda^{p+1} \right)$であり、この右辺は可積分である。
 したがって、優収束定理により、
 \[
  \lim_{k \to \infty} \int_\Omega  W_{\epsilon, k}(x) dx = 0
 \]
 である。さて、$\{v_k \}$は$H_0^1(\Omega)$の有界列であった。
 $H_0^1(\Omega) \subset L^{p+1}(\Omega)$のソボレフ不等式も
 考慮すると、
 \[
   \int_\Omega b ( v_k )_+^{p+1} dx \leq C
 \]
 をみたす$k$によらない$C > 0$が存在する。
 \eqref{eq:conv_H_W}より、
 \[
  \int_\Omega \left\lvert H(v_k, \underline{u}_\lambda) - H(v,
 \underline{u}_\lambda ) \right\rvert dx \leq \int_\Omega W_{\epsilon,
 k} (x)dx + \epsilon \int_\Omega b(v_k)_+^{p+1} dx
 \]
 であるから、$k \to \infty$の上極限をとると、次式が得られる。
 \[
 \limsup_{k \to \infty} \int_\Omega \left\lvert H(v_k, \underline{u}_\lambda) - H(v,
 \underline{u}_\lambda ) \right\rvert dx \leq C \epsilon
 \]
 $C > 0$は$k, \epsilon$によらず、$\epsilon > 0$は任意であるから、
 このことは
 \[
  \lim_{k \to \infty}
 \int_\Omega \left\lvert H(v_k, \underline{u}_\lambda) - H(v,
 \underline{u}_\lambda ) \right\rvert dx = 0
 \]
 と同値である。ゆえに\eqref{eq:conv_H}が成立する。
 以上の証明は、直接は~\cite{MR2886160}の Lemma~3.1 を参考にしているが、
 ~\cite{MR699419}のアイデアを参考にした。
 
 \eqref{eq:conv_h}も\eqref{eq:conv_H}と同様に証明される。
 \eqref{eq:conv_h}を示すためには、やはり
 \[
  \int_\Omega h^\prime(v_k, \underline{u}_\lambda) v_k dx
 \xrightarrow{k \to \infty} \int_\Omega h^\prime (v,
 \underline{u}_\lambda ) v dx 
 \]
 を示せば十分である。$t, s \geq 0$に対し、
 \[
  h(t, s, x) t \leq C b(x) (t^p s + s^p t)
 \]
 がしたがうため、\eqref{eq:conv_h}と同様に\eqref{eq:conv_H}も得られる。
 
 最後に\eqref{eq:conv_g}を証明する。\eqref{eq:minimal_vk_weakly}より、
 \[
  \int_\Omega a v_k \psi dx \xrightarrow{k \to \infty} \int_\Omega
 av\psi dx
 \]
 であるから、\eqref{eq:conv_g}を示すためには、
 \[
  \int_\Omega g^\prime(v_k, \underline{u}_\lambda)\psi dx \xrightarrow{
  k \to \infty } 
  \int_\Omega g^\prime(v, \underline{u}_\lambda)\psi dx
 \]
 を示せば十分である。
 \eqref{eq:dtau}、\eqref{eq:taus2p-1}と同様にすれば、
 $s, t, r \geq 0$に対し、次式がしたがう。
 \[
  g^\prime(t, s, x)r = b(x) \left( (t+s)^p - s^p \right) r \leq C b(x)
 \left( t^pr + s^p r \right).
 \]
 ヤングの不等式を$2$回使用すると、$\epsilon > 0$に対し、$C, C^\prime >
 0$が存在し、$s, t, r \geq 0$に対し、
 \[
  t^p r + s^p r \leq \epsilon t^{p+1} + C r^{p+1} + s^p r \leq
 \epsilon t^{p+1} + C^\prime (r^{p+1} + s^{p+1})
 \]
 が成立する。ゆえに、$g^\prime$は
 \[
  g^\prime(t, s, x)r \leq b \left( \epsilon t^{p+1} + C ( r^{p+1} +
 s^{p+1}) \right)
 \]
 と評価される。したがって、次式が成立する。
 \[  
  \left\lvert g^\prime(v_k, \underline{u}_\lambda)\psi - g^\prime(v,
   \underline{u}_\lambda)\psi \right\rvert \leq b \left( \epsilon
   (v_\epsilon)_+^{p+1}  + \epsilon v_+^{p+1} + C
   (\underline{u}_\lambda^{p+1} + \lvert \psi \rvert^{p+1})
                                 \right).
 \]
 そこで、$\tilde{W}_{\epsilon, k}$を
\[  \tilde{W}_{\epsilon, k} = 
   \left( \left\lvert g^\prime(v_k, \underline{u}_\lambda)\psi - g^\prime(v,
    \underline{u}_\lambda)\psi \right\rvert
  -\epsilon b (v_k)_+^{p+1}
          \right)_+ \]
 と定める。以降は、\eqref{eq:conv_H}の証明と同様に、\eqref{eq:conv_g}
 が示される。\qedhere
\end{proof}

命題~\ref{prop:second_1}を証明には、
$(\mathrm{PS})$条件を課さない峠の定理~\cite{MR0370183}を使用する。
その後、$I_\lambda$の
$(\mathrm{PS})_c$条件が必要となる。
$I_\lambda$の
$(\mathrm{PS})_c$条件を調べる準備として、
$I_\lambda$についてのパレ・スメイル列が
$H_0^1(\Omega)$の有界列であることを証明する。

\begin{lem} \label{lem:PS_seq}
 $\{ v_k \}_{k=0}^\infty$は、$I_\lambda$についての
 パレ・スメイル列であると仮定する。すなわち、
 \begin{enumerate}[(i)]
  \item $\{ I_\lambda(v_k) \}$は有界列。
  \item $\displaystyle I^\prime_{\lambda} (v_l) \xrightarrow{k \to
        \infty} 0 ~\tin H^{-1}(\Omega)$。
 \end{enumerate}
 と仮定する。このとき、$\{ v_k \}$は$H_0^1(\Omega)$の有界列である。
\end{lem}

\begin{proof}
 (i)より、
 \begin{equation}
  \frac{1}{2} \left\| v_k \right\|^2 - \int_\Omega G(v_k,
   \underline{u}_\lambda )dx \leq M \label{eq:ps_1}
 \end{equation}
 となる$k \in \N$によらない$M > 0$が存在する。
 $\epsilon > 0$とする。(ii)より、$K \in \N$が存在し、
 $k \geq K$、$\psi \in H_0^1(\Omega)$に対し、
 \[
  \int_\Omega (D v_k \cdot D \psi) dx - \int_\Omega g(v_k,
 \underline{u}_\lambda)\psi dx \leq \epsilon \left\| \psi
 \right\|_{H_0^1(\Omega) }
 \]
 が成立する。$\psi = v_k$とすると、次式が得られる。
 \begin{equation}
  \left\| v_k \right\|^2 \geq \int_\Omega g(v_k,
   \underline{u}_\lambda)v_k dx - \epsilon \left\| v_k
   \right\|_{H_0^1(\Omega)}. \label{eq:ps_2}
 \end{equation}
 $\alpha > 0$とする。
 \eqref{eq:ps_1}、\eqref{eq:ps_2}より、以下がしたがう。
 \begin{align}
  \alpha M &\geq  \frac{\alpha}{2} \left\| v_k \right\|^2 - \alpha
  \int_\Omega G(v_k, \underline{u}_\lambda) dx \notag \\
  & \geq \frac{\alpha-2}{2} \left\| v_k \right\|^2 + \int_\Omega 
  \left( g(v_k, \underline{u}_\lambda)v_k - \alpha G(v_k,
  \underline{u}_\lambda)  \right) dx - \epsilon \left\| v_k
  \right\|_{H^1_0(\Omega)}. \label{eq:alpha_M}
 \end{align}
 右辺の積分の中身を考察する。$t, s \geq 0$に対し、
 \[
   g(t, s)t - \alpha G(t, s)
  = b \left( (t+s)^p t - s^p t - \frac{\alpha}{p+1}(t+s)^{p+1} +
  \frac{\alpha}{p+1}s^{p+1} + \alpha s^p t \right) - a \left( t^2 -
  \frac{\alpha}{2} t^2 \right)
 \]
 である。ここで
 \[
  F(t) = (t+s)^p t - \frac{\alpha}{p+1}(t+s)^{p+1}
 \]
 の$t = 0$のまわりの$2$次のテイラー多項式は、
 \[
 s^pt + ps^{p-1} t^2 - \frac{\alpha}{p+1} s^{p+1} - \alpha s^p t -
 \frac{\alpha p}{2} s^{p-1} t^2 
 \]
 と計算される。$F$の$3$階の導関数は、
 \[
  F^{\prime\prime\prime}(t) = p(p-1)(p-2)(t+s)^{p-3} t + (3 - \alpha)
 p(p-1) (t+s)^{p-2}
 \]
 と計算される。テイラーの定理より、
 $3$次の剰余項$R_3$は、$0 < \theta < 1$を用いて、
 \[
  R_3 = \frac{F^{\prime\prime\prime}(\theta t)}{3!}t^3 = \frac{t^3}{6}
 \left( p(p-1)(p-2)(\theta t + s)^{p-3} \theta t + (3-\alpha) p(p-1)
 (\theta t + s)^{p-2} \right)
 \]
 とかける。以下では、$\alpha$を$p$に応じて定め、$R_3 \geq 0$となるように
 する。$p$の値に応じて場合分けをする。

 \ulinej{{$p \geq 2$}のとき}:$\alpha = 3$とすると、$R_3 \geq 0$が従う。

 \ulinej{{$1 < p \geq 2$}のとき}:$\alpha = p + 1$とすると、以下の通り
 $R_3 \geq 0$が従う。
 \begin{align*}
  R_3 &= \frac{t^3}{6} p(p-1)(\theta t + s)^{p-3} \left( (p-2)\theta t
  + (2-p) (\theta t + s)\right) \\
  &= \frac{t^3}{6}p(p-1)(2-p) s (\theta t + s)^{p-3} \geq 0.
 \end{align*}
 以上より、
 \begin{align*}
  g(t, s, x)t - \alpha G(t, s, x) &= b\left( ps^{p-1}t^2 -
  \frac{\alpha p}{2} s^{p-1} t^2 + R_3 \right) - a \left( t^2 -
  \frac{\alpha}{2} t^2 \right) \\
  & \geq b\left( ps^{p-1}t^2 -
  \frac{\alpha p}{2} s^{p-1} t^2 \right) - a \left( t^2 -
  \frac{\alpha}{2} t^2 \right) \\
  & = \left( \frac{\alpha}{2} - 1 \right) \left( at^2 - bps^{p-1} t^2 \right)
 \end{align*}
 と下から評価される。これを
 \eqref{eq:alpha_M}に適用すると、$a >
 \kappa$、及び、
 $\dnorm_\kappa$と$\dnorm_{H_0^1(\Omega)}$が
 同値であることから、以下の式変形が進む。
 \begin{align*}
  \alpha M &\geq \frac{\alpha-2}{2} \left( \left\| v_k \right\|^2 -
  \int_\Omega bp \underline{u}_\lambda^{p-1} v_k^2 dx + \int_\Omega
  av_k^2 dx \right) - \epsilon \left\| v_k \right\|_{H_0^1(\Omega)} \\
  & =\frac{\alpha -2}{2} \left( 1 - \frac{1}{\mu_1(\lambda)} \right)
  \left( \left\| v_k \right\|^2 + \int_\Omega av_k^2 dx \right) \\
  & \geq \frac{\alpha -2}{2} \left( 1 - \frac{1}{\mu_1(\lambda)}
  \right) \left\| v_k \right\|_\kappa^2 \\
  & \geq  \frac{\alpha -2}{2} \left( 1 - \frac{1}{\mu_1(\lambda)}
  \right) C \left\| v_k \right\|_{H_0^1(\Omega)}^2 - \epsilon \left\|
  v_k \right\|_{H_0^1(\Omega)}.
 \end{align*}
 $C >0$はポアンカレの不等式から決まる$k$によらない定数である。
 $\alpha$の定め方から$\alpha > 2$、補題~\ref{lem:lin_1}から
 $1 - 1/\mu_1(\lambda) > 0$であるから、結局$k \geq K$に対し、
 \[
  \alpha M \geq C^\prime \left\| v_k \right\|_{H_0^1(\Omega)}^2 - \epsilon
 \left\| v_k \right\|_{H_0^1(\Omega)}
 \]
 が成立する。ゆえに、$\{ v_k \}$は$H_0^1(\Omega)$の有界列である。 \qedhere
\end{proof}

補題~\ref{lem:PS_seq}を用いて、
$I_\lambda$の$(\mathrm{PS})_c$条件を調べる。

\begin{lem} \label{lem:PS_c}
 $0 < c < S^{N/2}/NM_1^{(N-2)/2}$とする。このとき、
 $I_\lambda$は$(\mathrm{PS})_c$条件をみたす。すなわち、
 次の条件(i), (ii)をみたす$H_0^1(\Omega))$の点列
 $\{ v_k \}_{k = 0}^\infty$は、収束する部分列をもつ。
 \begin{enumerate}[(i)]
  \item $\displaystyle \lim_{k \to \infty} I_\lambda (v_k) = c$。
  \item $\displaystyle I^\prime_\lambda (v_k) \xrightarrow{k \to
        \infty} 0 ~\tin H^{-1}(\Omega)$。
 \end{enumerate}
\end{lem}

\begin{proof}
 仮定(i), (ii)と補題~\ref{lem:PS_seq}より、$\{v_k \}$は
 $H_0^1(\Omega)$の有界列である。
 したがって、必要ならば部分列をとることにより、$v \in H_0^1(\Omega)$
 が存在し、$k \to \infty$とすると、
 以下が成立する。
 \begin{align}
  v_k \xrightharpoonup{ \mbox{ ~ } } v & \ \ \text{weakly~} \tin
  H_0^1(\Omega), \label{eq:vk_weakly} \\
  v_k \xrightarrow{ \mbox{ ~ } } v & \ \ \tin L^q(\Omega) \ \
   (q < p+1), \label{eq:vk_Lq} \\
  v_k \xrightarrow{ \mbox{ ~ } } v & \ \ \ae \tin \Omega. 
    \label{eq:vk_ae}
 \end{align}
 (ii)より、任意の$\psi \in H_0^1(\Omega)$に対し、
 \[
 \int_\Omega (Dv_k \cdot D\psi) dx - \int_\Omega g(v_k,
 \underline{u}_\lambda) \psi dx = o(1) \ \ (k \to \infty)
 \]
 が成り立つ。
 \eqref{eq:vk_weakly}と補題~\ref{lem:conv}より、
 $k \to \infty$とすると、次式がしたがう。
 \begin{equation}
  \int_\Omega (Dv \cdot D\psi) dx - \int_\Omega g(v,
   \underline{u}_\lambda) \psi dx = 0. \label{eq:intvpsi}
 \end{equation}
 つまり、$v \in H_0^1(\Omega)$は、$I_\lambda$の臨界点である。
 よって補題~\ref{lem:rel_heart_spade}.2により、$v$は
 \ref{eq:prob_sec}の弱解である。

 ここで、
 \begin{equation}
  I_\lambda(v) \geq 0 \label{eq:Ilambdageq0}
 \end{equation}
 であることを示す。
 \eqref{eq:intvpsi}で$\psi = v$とすると、
 \begin{equation}
  \int_\Omega \lvert Dv \rvert^2 dx = \int_\Omega g(v,
   \underline{u}_\lambda)v dx \label{eq:v_kankei}
 \end{equation}
 という関係式が導かれる。ゆえに、$v$における$I_\lambda$の値は
 \begin{equation}
  I_\lambda (v) = \frac{1}{2} \int_\Omega g(v, \underline{u}_\lambda)
   v dx - \int_\Omega G(v, \underline{u}_\lambda) dx \label{eq:Ilambda_v}
 \end{equation}
 と書けることがわかる。
 そこで、$t, s \geq 0$、$x \in \Omega$に対し、
 \begin{align*}
  \frac{1}{2} g(t, s, x) t - G(t, s, x) &= \frac{1}{2} \left(b \left(
  (t+s)^p - s^p \right) at \right)t - \left( b \left(
  \frac{1}{p+1}(t+s)^{p+1} - \frac{1}{p+1}s^{p+1} - s^p t -
  \frac{1}{2}at^2  \right) \right) \\
  &= b \left( \frac{1}{2} \left( (t+s)^pt - s^pt \right) -\left(
  \frac{1}{p+1} (t+s)^{p+1} - \frac{1}{p+1} s^{p+1} - s^p t \right) \right)
 \end{align*}
 を考える。
 \[
  \alpha(t) = \frac{1}{2} (t+s)^p t - \frac{1}{p+1}(t+s)^{p+1}
 \]
 の$1$次のテイラー多項式は、
 \[
  \frac{1}{2} s^p t - \frac{1}{p+1}s^{p+1} - s^p t
 \]
 である。$\alpha$の$2$階の導関数は、
 \[
  \alpha^{\prime\prime}(t) = \frac{p(p-1)}{2}(t+s)^{p-2}t
 \]
 と計算されるから、$2$次の剰余項は、$0 < \theta < 1$を用いて、
 \[
  \frac{\alpha^{\prime\prime}(\theta t)}{2} t^2 =
 \frac{p(p-1)}{2}(\theta t + s)^{p-2} \theta t^3
 \]
 と表すことができる。ゆえに、
 \[
  \frac{1}{2} g(t, s, x)t - G(t, s, x) = b(x) \frac{p(p-2)}{2} (\theta
 t + s)^{p-2} \theta t^3 \geq 0
 \]
 とわかる。\eqref{eq:Ilambda_v}と合わせ、
 \eqref{eq:Ilambdageq0}が得られる。
 \eqref{eq:Ilambdageq0}は、本証明の最後で重要な役割を担う。

 以下では、$k \to \infty$のとき
 $v_k \to v ~\tin H_0^1(\Omega)$であることを示す。
 これが示されれば、証明が完了する。
 $w_k = v_k - v$とおく。$H_0^1(\Omega)$の点列
 $\{ w_k \}_{k=0}^\infty$について、以下が成立する。
 \begin{align}
  w_k \xrightharpoonup{ \mbox{ ~ } } 0 & \ \ \text{weakly~} \tin
  H_0^1(\Omega), \label{eq:wk_weakly} \\
  w_k \xrightarrow{ \mbox{ ~ } } 0 & \ \ \tin L^q(\Omega) \ \
  (q < p+1), \label{eq:wk_Lq} \\
  w_k \xrightarrow{ \mbox{ ~ } } 0 & \ \ \ae \tin \Omega. 
  \label{eq:wk_ae}
 \end{align}
 $I_\lambda(v)$と$I_\lambda(v_k)$の差を、$w_k$を用いて評価する。
 \eqref{eq:wk_weakly}より、以下が成立する。
 \begin{equation}
  \int_\Omega \lvert D v_k \rvert^2 dx = \int_\Omega \lvert Dw_k + Dv
   \rvert^2 dx = \int_\Omega \lvert Dv \rvert^2 dx + \int_\Omega \lvert
   Dw_k \rvert^2 dx + o(1). \label{eq:Dvk_Dwk}
 \end{equation}
 ここで、$\tilde{w}_k = (v_k)_+ - v$とおく。
 $v > 0 ~\tin \Omega$より、$\lvert \tilde{w}_k \rvert \leq \lvert w_k
 \rvert ~\tin \Omega$である。また、
 $k \to \infty$とすると、$\tilde{w}_k \to 0 ~\ae \tin \Omega$となる。
 ゆえに、ブレジス・リーブの補題~\cite{MR699419}より、次式が成立する。
 \begin{equation}
  \int_\Omega b(v_k)_{+}^{p+1} dx = \int_\Omega bv^{p+1} dx +
   \int_\Omega \lvert b\tilde{w}_k \rvert^{p+1} dx + o(1) \ \ (k \to
   \infty).
   \label{eq:bzlblem}
 \end{equation}
 \eqref{eq:bzlblem}と補題~\ref{lem:conv}、および、$k \to \infty$のとき
 $(v_k)_+ \to v ~\ae ~\tin \Omega$より、次式が成立する。
 \begin{align}
  \int_\Omega g(v_k, \underline{u}_\lambda) v_k dx &= \int_\Omega
  h(v_k, \underline{u}_\lambda) v_k dx + \int_\Omega b (v_k)_+^{p+1}
  dx \notag \\
  &= \int_\Omega
  h(v, \underline{u}_\lambda) v dx + \int_\Omega b v^{p+1}
  dx + \int_\Omega b \lvert \tilde{w}_k \lvert^{p+1} dx + o(1) \notag \\
  &= \int_\Omega g(v, \underline{u}_\lambda) v dx + 
  \int_\Omega b \lvert \tilde{w}_k \lvert^{p+1} dx + o(1). \label{eq:w_k_1}
 \end{align}
 同様にして、次式も得られる。
 \begin{align}
  \int_\Omega G(v_k, \underline{u}_\lambda) dx &= 
  \int_\Omega H(v_k, \underline{u}_\lambda) dx + \frac{1}{p+1}
  \int_\Omega b (v_k)_+^{p+1} dx \notag \\
  &= \int_\Omega G(v, \underline{u}_\lambda) dx + \frac{1}{p+1}
  \int_\Omega b \lvert \tilde{w}_k \rvert^{p+1} dx + o(1). \label{eq:w_k_2}
 \end{align}
 \eqref{eq:Dvk_Dwk}、\eqref{eq:w_k_2}より、
 $I_\lambda(v_k)$と$I_\lambda(v)$の差は以下の通りに書ける。
 \[
  I_\lambda(v_k) = I_\lambda(v) + \frac{1}{2} \int_\Omega \lvert Dw_k
 \rvert^2 dx - \frac{1}{p+1} \int_\Omega \lvert \tilde{w}_k
 \rvert^{p+1} dx + o(1).
 \]
 ここで(i)より、次式が得られる。
 \begin{equation}
  I_\lambda(v) + \frac{1}{2} \int_\Omega \lvert Dw_k
   \rvert^2 dx - \frac{1}{p+1} \int_\Omega \lvert \tilde{w}_k
   \rvert^{p+1} dx = c + o(1). \label{eq:w_k_c}
 \end{equation}
 (ii)より、任意の$\epsilon > 0$に対し、$K \in \N$が存在し、
 $k > K$に対し、
 \[
 \left\lvert \int_\Omega \lvert Dv_k \rvert^2 dx - \int_\Omega g(v_k,
 \underline{u}_\lambda) v_k dx \right\rvert \leq \epsilon \left\| v_k
 \right\|_{H_0^1(\Omega)} 
 \]
 が成立する。$\{ v_k \}$は$H_0^1(\Omega)$の有界列であるから、
 $\epsilon$、$k$によらない$C > 0$が存在し、$k > K$のとき、
 \[
 \left\lvert \int_\Omega \lvert Dv_k \rvert^2 dx - \int_\Omega g(v_k,
 \underline{u}_\lambda) v_k dx \right\rvert \leq \epsilon C
 \]
 が成り立つ。ゆえに、次式が従う。
 \[
 \lim_{k \to \infty} \left( \int_\Omega \lvert Dv_k \rvert^2 dx -
 \int_\Omega g(v_k, \underline{u}_\lambda) v_k dx \right) = 0
 \]
 \eqref{eq:v_kankei}、\eqref{eq:Dvk_Dwk}、および、\eqref{eq:w_k_1}より、
 \[
 \lim_{k \to \infty} \left( \int_\Omega \lvert Dw_k \rvert^2 dx -
 \int_\Omega b \lvert \tilde{w}_k \rvert^{p+1} dx \right) = 0 
 \]
 が成立する。この事実と、$\{ v_k \}$が$H_0^1(\Omega)$の有界列であるこ
 とから、実数列
 \begin{equation}
  \left\{ \int_\Omega \lvert Dw_k \rvert^2 dx \right\}_k , \ 
  \left\{ \int_\Omega b \lvert \tilde{w}_k \rvert^{p+1} dx \right\}_k
  \label{eq:real_seq}
 \end{equation}
 は両方共ある有界閉区間上の数列である。
 したがって、必要ならば$\{ w_k \}$の部分列を取ると、
 \eqref{eq:real_seq}は収束する。
 収束先を$l \geq 0$とすると、
 \[
  \lim_{k \to \infty} \int_\Omega \lvert Dw_k \rvert^2 dx = \lim_{k
 \to \infty} \int_\Omega b \lvert \tilde{w}_k \rvert^{p+1} dx = l
 \]
 をみたす。\eqref{eq:S_def}により、
 \[
  \int_\Omega \lvert Dw_k \rvert^2 dx \geq S \left( \int_\Omega \lvert
 \tilde{w}_k \rvert^{p+1} dx \right)^{2/(p+1)} \geq S \left(
 \frac{1}{\left\| b \right\|_{L^\infty(\Omega)}} \int_\Omega b \lvert
 \tilde{w}_k \rvert^{p+1} dx \right)^{2/(p+1)}
 \]
 が成立する。$k \to \infty$として、
 \begin{equation}
  S l^{2/(p+1)} \leq \left\| b \right\|_{L^\infty(\Omega)}^{2/(p+1)} l 
   \label{eq:PS_c_S}
 \end{equation}
 が得られる。ここで$l = 0$であることを背理法を用いて示す。
 $l > 0$と仮定する。\eqref{eq:PS_c_S}より、
 $l \geq S^{N/2}/ \left\| b \right\|_{L^\infty(\Omega)}^{(N-2)/2}$
 である。一方、\eqref{eq:w_k_c}で$k \to \infty$とすると、
 \[
  I_\lambda(v = c - \frac{1}{N}l \leq c - \frac{S^{N/2}}{N \left\| b
 \right\|_{L^\infty(\Omega)}^{(N-2)/2}} < 0
 \]
 が得られる。これは\eqref{eq:Ilambdageq0}に反する。
 したがって、$l = 0$である。
 以上より、$k \to \infty$のとき、
 $w_k \to 0 ~\tin H_0^1(\Omega)$である。すなわち、
 $v_k \to v ~\tin H_0^1(\Omega)$である。これが示すべきことであった。\qedhere
\end{proof}

続いて、
$(\mathrm{PS})$条件を課さない峠の定理の仮定がみたされていることを
確認する。

\begin{lem} \label{lem:delta_rho}
 以下の条件をみたす$\delta > 0$、$\rho > 0$が存在する。
 \begin{equation}
  \text{$\left\|v \right\|_{H^1_0(\Omega)} = \delta$をみたす$v \in
   H_0^1(\Omega)$に対し、$I_\lambda(v) \geq \rho$が成立する。} 
   \label{eq:delta_rho}
 \end{equation}
\end{lem}

\begin{proof}
 $v \in H_0^1(\Omega)$は任意のものとする。
 \begin{align*}
  I_\lambda(v) &= \frac{1}{2} \left\| v \right\|^2 - \int_\Omega G(v,
  \underline{u}_\lambda)dx \\
  &= \frac{1}{2} \left( \int_\Omega \left( \lvert Dv \rvert^2 + a v^2
  \right) dx -\int_\Omega pb\underline{u}_\lambda^{p-1} v^2 dx
  \right)
  - \int_\Omega b \left( \frac{1}{p+1}(v +
  \underline{u}_\lambda)^{p-1} - \frac{1}{p+1}
  \underline{u}_\lambda^{p+1} - \underline{u}_\lambda^p v -
  \frac{p}{2} \underline{u}_\lambda^{p-1} v^2 \right) dx.
 \end{align*}
 第$1$項を$J_1$とおき、第$2$項の積分を$J_2$とおく。
 補題~\ref{lem:lin_p}.2 より、
 \begin{equation}
  J_1 \geq \frac{1}{2} \left( 1 - \frac{1}{\mu_1(\lambda)} \right)
   \int_\Omega \left( \lvert Dv \rvert^2 + a v^2 \right) dx \label{eq:J1}
 \end{equation}
 と下から評価される。補題~\ref{lem:lin_1}より、この括弧の中は正である。
 次に、$t, s \geq 0$に対し、$\alpha(t) = (t+s)^{p+1}/(p+1)$と定めると、
 $\alpha$の$t = 0$の周りの$2$次のテイラー多項式は、
 \[
 \frac{1}{p+1} (t+s)^{p+1} + s^p t + \frac{p}{2}s^{p-1}t^2
 \]
 である。$\alpha$の$3$階の導関数は
 $\alpha^{\prime\prime\prime}(t) = p(p-1)(t+s)^{p-2}$であるか
 ら、$3$次の剰余項
 \begin{equation}
  R_3 = \frac{1}{p+1}(t+s)^{p+1} - 
 \frac{1}{p+1} (t+s)^{p+1} - s^p t - \frac{p}{2}s^{p-1}t^2 \label{eq:R_3_1}
 \end{equation}
 は、テイラーの定理より、$0 < \theta < 1$を用いて、
 \[
  R_3 = \frac{\alpha^{\prime\prime\prime}(\theta t)}{3!} t^3 =
 \frac{p(p-1)}{6} (\theta t + s)^{p-2} t^3
 \]
 とかける。この$R_3$は、$p - 2 \geq 0$のとき、
 \begin{equation}  
 R_3  \leq C (2 \max \{ t, s \})^{p-2} t^3 
 = C 2^{p-2} ( \max \{ t^{p-2}, s^{p-2} \})t^3
 \leq  C ( t^{p-2} + s^{p-2} )t^3
 = C ( t^{p+1} + s^{p-2} t^3) \label{eq:R_3_2_pow}
 \end{equation}
 と評価される。$C > 0$は$s, t \geq 0$によらない。
 $p - 2 \leq 0$のときは、$R_3 \leq p(p-1)s^{p-1}t^3 / 6$であるから、
 やはり\eqref{eq:R_3_2_pow}に帰着する。
 さらに、ヤングの不等式より、任意の$\epsilon > 0$に対し、
 $C > 0$が存在し、$s, t \geq 0$に対し、
 $s^{p-2} t^3 \leq \epsilon s^{p-1} t^2 + C t^{p+1}$
 となる。ゆえに、$R_3$は
 \begin{equation}
  R_3 \leq \epsilon s^{p-1}t^2 + C t^{p+1} \label{eq:R_3_2}
 \end{equation}
 と評価される。\eqref{eq:R_3_1}と\eqref{eq:R_3_2}より、
 $J_2$の評価
 \begin{equation}
  J_2 \leq \epsilon \int_\Omega b \underline{u}_\lambda^{p-1} v^2 dx +
   C \int_\Omega b v^{p+1} dx \label{eq:J2}
 \end{equation}
 が得られる。\eqref{eq:J2}の$2$つの項は、それぞれ
 補題~\ref{lem:lin_p}.2、ソボレフ不等式より、
 \begin{align*}
  \int_\Omega b\underline{u}_\lambda^{p-1} v^2 dx &\leq
  \frac{1}{p\mu_1(\lambda)} \int_\Omega \left( \lvert Dv \rvert^2 +
  av^2 \right) dx, \\
  \int_\Omega bv^{p+1} dx &\leq \left\| b \right\|_{L^\infty(\Omega)}
  \left\| v \right\|_{L^{p+1}(\Omega)}^{p+1} \leq C \left\| v
  \right\|_{H_0^1(\Omega)}^{p+1}
 \end{align*}
 と更に評価が進む。これらと\eqref{eq:J1}より、$I_\lambda(v)$は
 \[
  I_\lambda(v) = J_1 - J_2 \geq C \int_\Omega \left( \lvert Dv
 \rvert^2 + a v^2 \right) dx - \epsilon C^\prime \int_\Omega \left(
 \lvert Dv \rvert^2 + av^2 \right) dx - C^{\prime\prime} \left\| v \right\|^{p+1}_{H_0^1(\Omega)}
 \]
 と下から評価される。必要ならば$\epsilon > 0$を小さくすれば、
 次式が得られる。
 \[
  I_\lambda (v) \geq C \int_\Omega \left( \lvert Dv \rvert^2 + av^2
 \right) dx - C^\prime \left\| v \right\|_{H_0^1(\Omega)}^{p+1} \leq C
 \left\| v \right\|_{\kappa}^p - C^\prime \left\| v
 \right\|^{p+1}_{H_0^1(\Omega)}.
 \]
 $\dnorm_\kappa$と$\dnorm_{H_0^1(\Omega)}$が同値なノルムであることを考
 慮すると、
 \[
  I_\lambda \geq C^{\prime\prime} \left\| v \right\|_{H_0^1(\Omega)}^2
 - C^{\prime} \left\| v \right\|_{H_0^1(\Omega)}^{p+1}
 \]
 と導かれる。$C^{\prime\prime}, C^\prime > 0$は$v$によらない。
 $2 < p+1$であるから、$\delta > 0$を十分小さくとれば、$\rho =
 C^{\prime\prime} \delta^2 - C^{\prime} \delta^{p+1} > 0$とできる。
 つまり、\eqref{eq:delta_rho}が成立する。 \qedhere
\end{proof}

命題~\ref{prop:second_1}を証明する最後の準備として、次の補題を証明する。

\begin{lem} \label{lem:mountain_dec}
 $v \geq 0 ~\tin \Omega$および
 \eqref{eq:int_omega_bvp+1}をみたす
 $v \in H_0^1(\Omega)$について、次式が成立する。
 \begin{equation}
  \lim_{t \to \infty} I_\lambda(tv) =  -\infty. \label{eq:mountain_dec}
 \end{equation}
\end{lem}

\begin{proof}
 $t, s \geq 0$、$x \in \Omega$とする。
 \[
  G^\prime(t, s, x) - \frac{b(x)}{p+1} t^{p+1} = b(x) \left(
 \frac{1}{p+1}(t+s)^{p+1} - \frac{1}{p+1} s^{p+1} - s^p t -
 \frac{1}{p+1} t^{p+1} \right)
 \]
 である。右辺の括弧の中を$\alpha(s)$とおく。$\alpha$の$1$階導関数は、
 \[
  \alpha^\prime(s) = (t+s)^p - s^p - p s^{p-1} t
 \]
 である。
 右辺を$t$の関数とみると、テイラーの定理より、
 \[
  \alpha^\prime(s) = \frac{p(p-1)}{2}(s+ \theta t)^{p-1} t^2
 \]
 をみたす$0 < \theta < 1$が存在する。したがって、$\alpha^\prime(s)
 \geq 0$である。すなわち、$\alpha$は$s$についての単調増加関数である。
 ゆえに、
 \[
 G^\prime(t, s, x) - \frac{b(x)}{p+1} t^{p+1} \geq 
 G^\prime(t, 0, x) - \frac{b(x)}{p+1} t^{p+1} = 0
 \]
 が成立する。したがって、
 \[
  \int_\Omega G(tv, \underline{u}_\lambda) dx \geq
 \frac{t^{p+1}}{p+1} \int_\Omega b v^{p+1} dx
 \]
 である。ゆえに、$I_\lambda(tv)$は、次の通りに上から評価される。
 \[
  I_\lambda(tv) \leq \frac{t^2}{2} \int_\Omega \left( \lvert Dv
 \rvert^2 + av^2 \right) dx - \frac{t^{p+1}}{p+1} \int_\Omega bv^{p+1} dx.
 \]
 \eqref{eq:int_omega_bvp+1}より、\eqref{eq:mountain_dec}が成立する。
\end{proof}

$(\mathrm{PS})$条件を課さない峠の定理を用いて、命題
~\ref{prop:second_1}を証明する。

\begin{proof}[命題~\ref{prop:second_1}]
 
\end{proof}

% Local Variables:
% mode: yatex
% TeX-master: "main.tex"
% End: