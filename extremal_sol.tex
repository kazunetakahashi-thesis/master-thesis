%#!platex main.tex
\section{extremal solution の存在と一意性} \label{sec:extremal_sol}

本節では、\ref{eq:prob_main}の extremal solution について考察する。
$\lambda = \bar{\lambda}$における\ref{eq:prob_main}を考察する。

\begin{defn}
 $\bar{\lambda}$を記号~\ref{nota:ext}のものとする。
 $\lambda = \bar{\lambda}$における\ref{eq:prob_main}の弱解を
 \ref{eq:prob_main}の{\bf extremal solution }という。
\end{defn}

\subsection{extremal solution の存在}

本小節では、\ref{eq:prob_main}の extremal solution が存在することを示
す。このために、まず以下の集合を考察する。
\begin{equation}
 K = \{ \underline{u}_\lambda \in H_0^1(\Omega) \mid 0 < \lambda <
  \bar{\lambda} \}.
  \label{eq:set_min_sol}
\end{equation}

\begin{lem} \label{lem:set_min_sol}
 \eqref{eq:set_min_sol}の$K$は$H_0^1(\Omega)$の有界集合である。
\end{lem}

\begin{proof}
 $g_0 \in H_0^1(\Omega)$を記号~\ref{nota:g_0}のものとする。
 $v_\lambda = \underline{u}_\lambda - \lambda g_0$と定める。
 すると、次が成立する。
 \[
  -\Delta v_\lambda = \underline{u}_\lambda - \lambda g_0 \ \ ~\tin \Omega.
 \]
 ゆえに、$\psi \in H_0^1(\Omega)$とすると、次が成立する。
 \[
  \int_\Omega (Dv_\lambda \cdot D\psi + a v_\lambda \psi) dx =
 \int_\Omega b(v_\lambda + \lambda g_0)^p \psi dx.
 \]
 $\psi = v_\lambda$とおくと、次を得る。
 \begin{equation}
  \int_\Omega \left( \lvert Dv_\lambda \rvert^2 + a \lvert v_\lambda
               \rvert^2 \right) dx =
  \int_\Omega b(v_\lambda + \lambda g_0)^p v_\lambda dx. \label{eq:v_lambda}
 \end{equation}

 ここで、次の事実を示す。任意の$\epsilon > 0$に対し、$C > 0$が
 存在し、任意の$s, t \geq 0$に対し、次式が成立する。
 \begin{equation}
  (t+s)^p \leq (1 + \epsilon) (t+s)^{p-1} t + C s^p. \label{eq:tspleq}
 \end{equation}
 まず、$(t+s)^{p-1}s$にヤングの不等式を用いる。
 $q, r > 1$は、$q^{-1} + r^{-1} = 1$をみたすものとする。
 任意の$0 < \tilde{\epsilon} < 1$に対し、$\tilde{C} > 0$が存在し、
 次が成立する。
 \[
  (t+s)^{p-1} s \leq \tilde{\epsilon} \left( (t+s)^{p-1} \right)^q +
 \tilde{C} s^r.
 \]
 ここで$q = p/(p-1)$とおくと、$r = p$である。
 ゆえに、以下が成立する。
 \begin{align*}
  (t+s)^{p-1} s & \leq \tilde{\epsilon} (t+s)^p + \tilde{C} s^p \\
  &= \tilde{\epsilon} (t+s)^{p-1} t + \tilde{\epsilon} (t+s)^{p-1} s 
  + \tilde{C} s^p, \\
  (t+s)^{p-1} s &\leq \frac{\tilde{\epsilon}}{1 - \tilde{\epsilon}}
  (t+s)^{p-1} t + \frac{\tilde{C}}{1 - \tilde{\epsilon}} s^p.
 \end{align*}
 任意の$\epsilon > 0$に対し、
 $\epsilon = \tilde{\epsilon} / (1 - \tilde{\epsilon})$となる
 $0 < \tilde{\epsilon} < 1$は存在する。この$\tilde{\epsilon}$に対し、
 $C = \tilde{C}/ (1 - \tilde{\epsilon})$とすると、次が成立する。
 \[
  (t+s)^{p-1} s \leq \epsilon (t+s)^{p-1} t + Cs^p.
 \]
 $(t+s)^{p} = (t+s)^{p-1} s + (t+s)^{p-1}t$より、
 \eqref{eq:tspleq}が得られる。以上の\eqref{eq:tspleq}の証明は
 \cite{MR2317491}~の Lemma 4.1 によった。

 \eqref{eq:v_lambda}の左辺を$I$とおく。
 \eqref{eq:tspleq}より、次式が成立する。
 \begin{equation}
  \int_\Omega b (v_\lambda + \lambda g_0)^p v_\lambda dx 
   \leq (1 + \epsilon) \int_\Omega b \underline{u}_\lambda ^{p-1}
   v_\lambda^2 dx + C \lambda^p \int_\Omega bg_0^p v_\lambda dx.
   \label{eq:v_lambda_I}
 \end{equation}
 ここで、補題~\ref{lem:lin_p}.2、補題~\ref{lem:lin_1}から、次式を得る。
 \[
  I \leq \mu_1 p \int_\Omega b (\underline{u}_\lambda)^{p-1}
 \underline{v}_\lambda^2 dx > p \int_\Omega b (\underline{u}_\lambda)^{p-1}
 \underline{v}_\lambda^2 dx.
 \]
 すなわち、次を得る。
 \begin{equation}
  \int_\Omega b(\underline{u}_\lambda)^{p-1} v_\lambda^2 dx <
   \frac{I}{p} \label{eq:I/p}
 \end{equation}
 また、$g_0, v_\lambda \in H_0^1(\Omega) \subset L^{p+1}(\Omega)$、及び、
 ヘルダーの不等式、ソボレフの不等式から、次式を得る。
 \begin{equation}
  \int_\Omega b g_0^p v_\lambda dx \leq 
   \left\| b  \right\|_{L^\infty(\Omega)}
   \left\| g_0 \right\|_{L^{p+1}(\Omega)}^p 
   \left\| v_\lambda \right\|_{L^{p+1}(\Omega)}
   \leq C \left\| v_\lambda \right\|_{H_0^1(\Omega)}
   \leq C^\prime \left\| v_\lambda \right\|_{\kappa}.
   \label{eq:bg_0v_lambda}
 \end{equation}
 ここで$C, C^\prime > 0$は$\lambda$によらない。

 \eqref{eq:v_lambda}、\eqref{eq:I/p}、\eqref{eq:bg_0v_lambda}
 から、次式がしたがう。
 \[
  I \leq \frac{1 + \epsilon}{p} I + \bar{\lambda}^p C \left\|
 v_\lambda \right\|_{\kappa}.
 \]
 $\epsilon > 0$を$(1+\epsilon)/p < 1$となるよう小さくとれば、
 $I \leq C \left\| v_\lambda \right\|_\kappa$となる。
 ここで$I \geq \left\| v_\lambda \right\|_\kappa^2$、$v_\lambda \not
 \equiv 0$であるから、$\left\| v_\lambda \right\|_{\kappa} \leq C$であ
 る。$\dnorm_\kappa$と$\dnorm_{H_0^1(\Omega)}$は同値であるから、
 $\{ v_\lambda \in H_0^1(\Omega) \mid 0 < \lambda < \bar{\lambda} \}$
 は$H_0^1(\Omega)$の有界集合である。
 $v_\lambda$の定め方から
 $\underline{u}_\lambda = v_\lambda + \lambda g_0$であるため、
 次の式が成立する。
 \[
  \left\| \underline{u}_\lambda \right\|_{H_0^1(\Omega)} 
 \leq \left\| v_\lambda \right\|_{H_0^1(\Omega)}
 + \bar{\lambda} \left\| g_0 \right\|_{H_0^1(\Omega)}.
 \]
 右辺は$\lambda$によらない定数で抑えられる。従って、
 \eqref{eq:set_min_sol}の$K$は、$H_0^1(\Omega)$の有界集合である。\qedhere
\end{proof}

$\lambda \nearrow \bar{\lambda}$のときの
$\underline{u}_\lambda$の極限をとることで、
\ref{eq:prob_main}の extremal solution を構成する。

\begin{prop} \label{prop:ext_exi}
 \begin{enumerate}[1.] \sage
  \item \ref{eq:prob_main}の extremal solution が存在する。とくに、
        $\lambda = \bar{\lambda}$における\ref{eq:prob_main}の
        minimal solution $\underline{u}_{\bar{\lambda}}$が存在する。
  \item $\lambda > 0$とする。$\lambda \nearrow \bar{\lambda}$のとき、
        $\underline{u}_\lambda \nearrow
        \underline{u}_{\bar{\lambda}} ~\ae
        \tin
        \Omega$となる。
 \end{enumerate}
\end{prop}

\begin{proof}
 \begin{enumerate}[1.] \sage
  \item 正の数の列$\{ \lambda_n \}_{n=0}^\infty$は、
        $\lambda_n \nearrow \bar{\lambda}$をみたすものとする。
        $u_n = \underline{u}_{\lambda_n}$とかく。
        $u_n$は$\lambda = \lambda_n$における
         \ref{eq:prob_main}の弱解であるから、
        任意の$\psi \in H_0^1(\Omega)$
        に対し、次が成立する。
        \begin{equation}
         \int_\Omega (Du_n \cdot D\psi + a u_{n} \psi) dx 
          = \int_\Omega bu_n^p \psi dx + \lambda \int_\Omega f\psi dx.
          \label{eq:u_n_weaksol}
        \end{equation}
        補題~\ref{lem:set_min_sol} より、$\{ u_n \}$は$H_0^1(\Omega)$の
         有界列である。
        ゆえに、必要ならば部分列をとることにより、
        $u \in H_0^1(\Omega)$が存在して、$n \to \infty$とすると、
        以下が成立する。
        \begin{align}
         u_n \xrightharpoonup{ \mbox{ ~ } } u & \ \ \text{weakly~} \tin
         H_0^1(\Omega), \label{eq:extremal_u_n_weakly} \\
         u_n \xrightarrow{ \mbox{ ~ } } u & \ \ \tin L^q(\Omega) \ \
          (q < p+1), \notag \\
         u_n \xrightarrow{ \mbox{ ~ } } u & \ \ \ae \tin \Omega. 
         \label{eq:extremal_u_n_ae}
        \end{align}
        $u$が\ref{eq:prob_main}の extremal solution であることを示す。
        \eqref{eq:extremal_u_n_weakly}により、次が成立する。
        \[
        \int_\Omega (Du_n \cdot D\psi + a u_n \psi) dx
        \xrightarrow{n \to \infty}
        \int_\Omega (Du \cdot D\psi + a u \psi) dx.
        \]
         補題~\ref{lem:minimal_va}.3 と
        \eqref{eq:extremal_u_n_ae}により、$u_n \leq u ~\tin \Omega$となる。
         とくに、$u > 0 ~\tin \Omega$である。
        また、$b \in L^\infty(\Omega)$、
        $u, \psi \in H_0^1(\Omega) \subset L^{p+1}(\Omega)$より、
        \[
        \left\lvert bu_n^p \psi \right\rvert \leq b \hat{u}^p
        \lvert\psi\rvert \ \ \ae
        \tin \Omega
        \]
         の右辺は可積分である。\eqref{eq:extremal_u_n_ae}より、
        優収束定理から、次を得る。
        \[
        \int_\Omega bu_n^p \psi dx \xrightarrow{n \to \infty} 
        \int_\Omega bu^p \psi dx.
        \]
        したがって、\eqref{eq:u_n_weaksol}で$n \to \infty$とすると次を得る。
        \[ 
         \int_\Omega (Du \cdot D\psi + a u \psi) dx 
          = \int_\Omega bu^p \psi dx + \bar{\lambda} \int_\Omega f\psi dx.
        \]
        $\psi \in H_0^1(\Omega)$は任意であるから、
        $u \in H_0^1(\Omega)$は\ref{eq:prob_main}の extremal solution
        である。
        すなわち、\ref{eq:prob_main}の extremal solution
        が存在する。補題~\ref{lem:minimal_itt}.2 より、
        特に$\lambda = \bar{\lambda}$における
        \ref{eq:prob_main}の minimal solution 
        $\underline{u}_{\bar{\lambda}}$
        が存在する。
  \item 補題~\ref{lem:minimal_itt}.3 より、
        $u_n = \underline{u}_{\lambda_n} <
        \underline{u}_{\bar{\lambda}} ~\tin \Omega$
        である。$n \to \infty$とすると、
        $u \leq \underline{u}_{\bar{\lambda}} ~\tin \Omega$
        を得る。$u \in S_{\bar{\lambda}}$であり、
        $\underline{u}_{\bar{\lambda}}$は$\lambda = \bar{\lambda}$
        における \ref{eq:prob_main}の minimal solution 
        であるから、$u = \underline{u}_{\bar{\lambda}}$である。
        したがって、$n \to \infty$のとき、
        $\underline{u}_{\lambda_n} \nearrow
        \underline{u}_{\bar{\lambda}} ~\ae \tin \Omega$
        となる。$\{ \lambda_n \}$の任意性により、
        $\lambda \nearrow \bar{\lambda}$のとき、
        $\underline{u}_{\lambda} \nearrow
        \underline{u}_{\bar{\lambda}} ~\ae \tin \Omega$となる。\qedhere
 \end{enumerate}
\end{proof}

\subsection{extremal solution の一意性}

前小節では、\ref{eq:prob_main}の extremal solution の存在を示した。
本小節では、\ref{eq:prob_main}の extremal solution が
$b > 0 ~\tin \Omega$のときは唯一つに限る
ことを示す。

鍵となるのは、\eqref{eq:lin_prob_min_sol}の
第$1$固有値$\mu_1(\lambda)$である。補題~\ref{lem:lin_1}では、
$0 < \lambda < \bar{\lambda}$において
$\mu_1(\lambda) > 1$となることを示した。
$b > 0 ~\tin \Omega$のときは、
$\lambda = \bar{\lambda}$において、この不等式が成立しなくなることを示
す。その準備として、
まずは$\lambda$が大きくなると$\mu_1(\lambda)$が小さくなることを示す。

\begin{lem} \label{rem:mu1_dec}
 $\lambda_1, \lambda_2$は、$0 < \lambda_1 \leq \lambda_2 \leq
 \bar{\lambda}$とする。このとき、$\mu_1(\lambda_1) \geq
 \mu_1(\lambda_2)$が成立する。
\end{lem}

\begin{proof}
 $\phi_1$を$\lambda = \lambda_1$における
 \ref{eq:prob_main}のminimal solutionに関する
 線形化固有値問題\eqref{eq:lin_prob_min_sol}の
 第$1$固有関数とする。補題~\ref{lem:lin_p}.2より、$\mu_1(\lambda_1)$、
 $\mu_1(\lambda_2)$は以下の通りに書ける。
 \begin{align*}
  \mu_1(\lambda_1) &= \frac{\displaystyle 
  \int_\Omega \left( \lvert D\phi_1 \rvert^2
  + a \phi_1^2 \right) dx}{\displaystyle \int_\Omega p b
  \underline{u}_{\lambda_1}^{p-1} \phi_1^2 dx}, \\
  \mu_1(\lambda_2) &= \inf_{\psi \in H_0^1(\Omega), \ \psi \not\equiv
  0} \frac{\displaystyle 
  \int_\Omega \left( \lvert D\psi \rvert^2
  + a \psi^2 \right) dx}{\displaystyle \int_\Omega p b
  \underline{u}_{\lambda_2}^{p-1} \psi^2 dx}.
 \end{align*}
 補題~\ref{lem:minimal_va}.3より、
 $\underline{u}_{\lambda_1} \leq \underline{u}_{\lambda_2} ~\ae ~\tin
 \Omega$であるから、
 \[
  \mu_1(\lambda_2) \leq 
  \inf_{\phi_1 \in H_0^1(\Omega), \ \phi_1 \not\equiv
  0} \frac{\displaystyle 
  \int_\Omega \left( \lvert D\phi_1 \rvert^2
  + a \phi_1^2 \right) dx}{\displaystyle \int_\Omega p b
  \underline{u}_{\lambda_2}^{p-1} \phi_1^2 dx} \leq
  \frac{\displaystyle 
  \int_\Omega \left( \lvert D\phi_1 \rvert^2
  + a \phi_1^2 \right) dx}{\displaystyle \int_\Omega p b
  \underline{u}_{\lambda_1}^{p-1} \phi_1^2 dx} = \mu_1(\lambda_1)
 \]
 が得られる。 \qedhere
\end{proof}

\begin{lem} \label{lem:mu_1_bar_lambda}
 \begin{enumerate}[1.] \sage
  \item $\lambda \nearrow \bar{\lambda}$のとき、$\mu_1(\lambda)
        \searrow \mu_1(\bar{\lambda})$である。
  \item $b > 0 ~\tin \Omega$ならば、$\mu_1(\bar{\lambda}) = 1$である。
 \end{enumerate}
\end{lem}
\begin{proof}
 \begin{enumerate}[1.] \sage
  \item $\phi_1 \in H_0^1(\Omega)$を、$\mu_1(\bar{\lambda})$に付随する
        $\phi_1 > 0 ~\tin \Omega$をみたす固有関数とする。
        正の実数列
        $\{\lambda_n \}_{n=0}^\infty$を$\lambda_n$を、$\lambda_n
        \nearrow \bar{\lambda}$をみたすものとする。
        単調収束定理より、次式が成立する。
        \[
         \int_\Omega bp ( \underline{u}_{\lambda_n} )^{p-1} \phi_1^2
          dx \nearrow \int_\Omega bp
          (\underline{u}_{\bar{\lambda}})^{p-1} \phi_1^2 dx  \ \ (n
        \to \infty).        
        \]
        $\{\lambda_n \}$の任意性より、次式が成立する。
        \begin{equation}
         \int_\Omega bp ( \underline{u}_{\lambda} )^{p-1} \phi_1^2
          dx \nearrow \int_\Omega bp
          (\underline{u}_{\bar{\lambda}})^{p-1} \phi_1^2 dx
           \ \ (\lambda \nearrow \bar{\lambda}).
          \label{eq:int_lambda_barlambda}
        \end{equation}
        $\epsilon > 0$とする。\eqref{eq:int_lambda_barlambda}より、
        $\delta > 0$が存在し、$0 < \bar{\lambda} - \lambda < \delta$な
        らば、
        \begin{equation}
         0 < \frac{\displaystyle \int_\Omega \left( \lvert D\phi_1
                                              \rvert^2
                                              + a
                                             \phi_1^2\right) dx}
         {\displaystyle 
         \int_\Omega bp (\underline{u}_{\bar{\lambda}})^{p-1} \phi_1^2 dx}
         -
         \frac{\displaystyle \int_\Omega \left( \lvert D\phi_1
                                              \rvert^2
                                              + a
                                             \phi_1^2\right) dx}
         {\displaystyle 
         \int_\Omega bp (\underline{u}_{\lambda})^{p-1} \phi_1^2
         dx} < \epsilon \label{eq:zero_int_epsilon}
        \end{equation}
        が成立する。ここで、$\tilde{\mu}(\lambda)$を
        \[
         \tilde{\mu}(\lambda) = 
        \frac{\displaystyle \int_\Omega \left( \lvert D\phi_1
                                              \rvert^2
                                              + a
                                             \phi_1^2\right) dx}
         {\displaystyle 
         \int_\Omega bp (\underline{u}_{\bar{\lambda}})^{p-1} \phi_1^2 dx}
        \]
        と定めると、\eqref{eq:zero_int_epsilon}は
        $0 < \tilde{\mu}(\lambda) - \mu_1(\bar{\lambda}) < \epsilon$
        と書き直される。\eqref{eq:mu1_quotient}より、
        $\mu_1(\lambda) \leq \tilde{\mu}(\lambda)$である。
        補題~\ref{rem:mu1_dec}より$\mu_1(\bar{\lambda}) \leq
        \mu_1(\lambda)$
        である。したがって、$0 < \bar{\lambda} - \lambda < \delta$なら
        ば、$0 \leq \mu_1 (\lambda) - \mu_1 (\bar{\lambda}) \leq
        \tilde{\mu}(\lambda) - \mu_1 (\bar{\lambda}) < \epsilon$となる。
        以上より、$\lambda \nearrow \bar{\lambda}$のとき、
        $\mu_1(\lambda) \searrow \mu_1(\bar{\lambda})$である。
  \item 補題~\ref{lem:lin_1} および 1.~より、$\mu_1(\bar{\lambda})
        \geq 1$である。$\mu_1(\bar{\lambda}) = 1$を背理法を用いて示す。
        
        $\mu_1(\bar{\lambda}) > 1$であると仮定する。
        $\Phi \colon [0,\infty) \times H^1_0 (\Omega) \to
        H^{-1}(\Omega)$を
        \eqref{eq:def_of_Phi}の通りに定める。
        \eqref{eq:Phi_dr}より、$w \in H^1_0(\Omega)$に対し
        \begin{equation}
         \Phi_u (\bar{\lambda}, \underline{u}_{\bar{\lambda}})
          w = -\Delta w + aw - b
          p(\underline{u}_{\bar{\lambda}})^{p-1} w.
          \label{eq:Phi_dr_barlambda}
        \end{equation}
        となる。
        ここで、
        $\Phi_u (\bar{\lambda}, \underline{u}_{\bar{\lambda}}) \colon
        H_0^1(\Omega) \to H^{-1}(\Omega)$
        が可逆であることを示す。

        まずは$\Phi_u (\bar{\lambda}, \underline{u}_{\bar{\lambda}})$
        が
        全射であることを示す。$f \in H^{-1}(\Omega)$とする。
        $I_f \colon H_0^1(\Omega) \to \R$を次式で定める。
        \begin{equation}
         I_f (w) = \int_\Omega \left( \lvert Dw \rvert + aw^2 \right)
          dx - \int_\Omega bp \underline{u}_{\bar{\lambda}}^{p-1} w^2
          dx - \int_\Omega fw dx. \label{eq:def_I_f}
        \end{equation}
        $I_f$の$H_0^1(\Omega)$における下限を達成する元$u \in
        H_0^1(\Omega)$が
        存在するとすれば、$u$は$\Phi(\bar{\lambda},
        \underline{u}_\lambda)u = f$をみたす。
        以下、この$u$の存在を示す。$w \in H_0^1(\Omega)$とする。
        補題~\ref{lem:lin_p}.1より、次式が従う。
        \begin{equation}
         \int_\Omega \left( \lvert Dw \rvert^2 + aw^2 \right) dx \geq
          \mu_1(\bar{\lambda}) \int_\Omega bp
          \underline{u}_{\bar{\lambda}}^{p-1} w^2 dx.
          \label{eq:mu_1_bar_lambda_r} 
        \end{equation}
        ゆえに、\eqref{eq:def_I_f}は以下の通りに下から評価される。
        \[
         I_f(w) \geq \left( 1 - \frac{1}{\mu_1(\bar{\lambda})} \right)
        \int_\Omega \left( \lvert Dw \rvert^2 + aw^2\right) dx -
        \int_\Omega fw dx \geq \left( 1 -
        \frac{1}{\mu_1(\bar{\lambda})} \right) \left\| w
        \right\|_\kappa^2 - \left\| f \right\|_{H^{-1}(\Omega)}
        \left\| w \right\|_{H_0^1(\Omega)}.
        \]
        仮定より、$\mu_1(\bar{\lambda}) > 1$であるから、
        $1 - 1/\mu_1(\bar{\lambda}) > 0$である。また、
        $\dnorm_\kappa$と$\dnorm_{H_0^1(\Omega)}$は
        同値なノルムである。ゆえに、$C, C^\prime > 0$が存在し、
        $w \in H_0^1(\Omega)$に対し、
        \[
         I_f (w) \geq C \left\| w \right\|^2_{H_0^1(\Omega)} -
          C^\prime \left\| w \right\|_{H_0^1(\Omega)}          
        \]
        が成立する。ゆえに、$\left\| w \right\|_{H_0^1(\Omega)} \to
        \infty$のとき、$I_f(w) \to \infty$となる。
        すなわち、$I_f$は強圧的であり、下に有界である。
        ここで$w \mapsto \langle f, w \rangle$は弱連続であり、
        $L \colon \R^N \times \R \times \Omega$を
        \[
         L(t, s, x) = t^2 + a(x)s^2 - b(x)p
        \underline{u}_{\bar{\lambda}}^{p-1}(x) s^2
        \]
        と定めると、\eqref{eq:def_I_f}より
        \[
         I_f (w) = \int_\Omega L(Dw, w, x) dx - \langle f , w \rangle
        \]
        である。
        $L$は$t, s$について連続であり、
        $t$について下に凸である
        から、$I_f$は弱下半連続である。
        したがって、$I_f$の$H_0^1(\Omega)$における下限を達成する元$u
        \in H_0^1(\Omega)$が存在する。
        この部分は、例えば~\cite{MR2597943}の\S~8.2のTHEOREM 1, 2を
        参照されたい。以上より、
        $\Phi_u (\bar{\lambda}, \underline{u}_{\bar{\lambda}}) \colon
        H_0^1(\Omega) \to H^{-1}(\Omega)$は全射である。
        
        次に、$\Phi_u (\bar{\lambda}, \underline{u}_{\bar{\lambda}}) \colon
        H_0^1(\Omega) \to H^{-1}(\Omega)$が単射であることを示す。
        $\Phi_u (\bar{\lambda}, \underline{u}_{\bar{\lambda}})$は
        線形写像であるから、
        $\Phi_u (\bar{\lambda}, \underline{u}_{\bar{\lambda}})w = 0$
        となる
        $w \in H_0^1(\Omega)$が$w = 0$に限ることを示せば良い。
        いま、$f = 0$であるから、
        $\langle \Phi_u (\bar{\lambda},
        \underline{u}_{\bar{\lambda}})w , w \rangle = 0$である。つまり、
        \[
         \int_\Omega \left( \lvert Dw \rvert^2 + aw^2 \right)dx =
        \int_\Omega p b\underline{u}_\lambda^{p-1} w^2 dx
        \]
        である。\eqref{eq:mu_1_bar_lambda_r}と合わせて、
        \[
         \left( \mu_1(\bar{\lambda}) - 1 \right) p \int_\Omega b
        \underline{u}_{\bar{\lambda}}^{p-1} w^2 dx \leq 0
        \]
        が従う。$\mu_1 ( \bar{\lambda}) > 1$より、
        \[
         \int_\Omega b
        \underline{u}_{\bar{\lambda}}^{p-1} w^2 dx \leq 0 
        \]
        となるが、$b , \underline{u}_{\bar{\lambda}} > 0 ~\tin \Omega$
        であるから、$w \equiv 0$と結論付けられる。
        したがって、
        $\Phi_u (\bar{\lambda}, \underline{u}_{\bar{\lambda}})$
        は単射である。

        以上より、
        $\Phi_u (\bar{\lambda}, \underline{u}_{\bar{\lambda}}) \colon
        H_0^1(\Omega) \to H^{-1}(\Omega)$は全単射である。
        ゆえに、陰関数定理より、$\epsilon > 0$が存在して、
        $\lambda \in (\bar{\lambda} - \epsilon, \bar{\lambda} +
        \epsilon)$
        に対し、$\Phi (\lambda, \underline{u}_{\lambda}) = 0$をみたす
        $\underline{u}_\lambda \in H_0^1(\Omega)$が存在する。
        補題~\ref{lem:imp}.1の証明と同様にすると、
        $\underline{u}_\lambda$は\ref{eq:prob_main}の弱解である。
        ゆえに、$S_\lambda \neq \emptyset$をみたす
        $\lambda > \bar{\lambda}$が存在する。しかし
        これは、定義~\ref{defn:ext}に反する。したがって、
        背理法により、$\mu_1(\bar{\lambda}) = 1$と結論付けられる。
 \end{enumerate}
\end{proof}

$\mu_1(\bar{\lambda}) = 1$であることから、
\ref{eq:prob_main}の extremal solution の一意性が証明される。

\begin{prop} \label{prop:ext_uni}
 $b > 0 ~\tin \Omega$と仮定する。
 \ref{eq:prob_main}の extremal solution は、
 $\lambda = \bar{\lambda}$における\ref{eq:prob_main}の minimal
 solution $\underline{u}_{\bar{\lambda}}$に限る。
\end{prop}

\begin{proof}
 $u \in S_{\bar{\lambda}}$とする。$z = u -
 \underline{u}_{\bar{\lambda}}$とする。$\underline{u}_{\bar{\lambda}}$
 は$\lambda = \bar{\lambda}$における\ref{eq:prob_main}の
 minimal solution であるから、$z \geq 0 ~\tin \Omega$である。また、
 \begin{align*}
  -\Delta u + au &= bu^p + \bar{\lambda} f, \\
  -\Delta \underline{u}_{\bar{\lambda}} + a
  -\underline{u}_{\bar{\lambda}}
  &= b \underline{u}_{\bar{\lambda}}^p + \bar{\lambda} f
 \end{align*}
 の両辺を引くと、
 \begin{equation}
  -\Delta z + az = b \left( u^p - \underline{u}_{\bar{\lambda}}^p
 \right) ~\tin \Omega \label{eq:unique_1}
 \end{equation}
 が得られる。
 
 $\phi_1 \in H_0^1(\Omega)$を、$\mu_1(\bar{\lambda})$に付随する
 $\phi_1 > 0 ~\tin \Omega$をみたす固有関数とする。
 補題~\ref{lem:lin_p}.1、及び、
 補題~\ref{lem:mu_1_bar_lambda}.2より、
 \begin{equation}
  -\Delta \phi_1 + a\phi_1 = pb \underline{u}_{\bar{\lambda}}^{p-1}
   \phi_1 ~\tin \Omega \label{eq:unique_2}
 \end{equation}
 が成立する。$\text{\eqref{eq:unique_1}} \times \phi_1 -
 \text{\eqref{eq:unique_2}} \times z$を$\Omega$上積分すると、
 \begin{equation}
  \int_\Omega b F(u, \underline{u}_{\bar{\lambda}}) \phi_1 dx = 0
   \label{eq:unique_eq_zero}
 \end{equation}
 が得られる。ここで$F \colon \R^2 \to \R$は、
 $F(t, s) = t^p - s^p - ps^{p-1}(t-s)$である。
 $\alpha(t) = t^p$とする。
 $\alpha$の$t = s$の周りの$1$次のテイラー多項式は、
 $s^p + ps^{p-1}(t-s)$である。
 ゆえに、
 $2$次の剰余項は$F(t, s)$に一致する。
 テイラーの定理より、ある$0 < \theta < 1$を用いて、
 \[
  F(t, s) = \frac{\alpha^{\prime\prime}(s + \theta t)}{2!} (t-s)^2 = \frac{1}{2}
 p(p-1) (s+ \theta t)^{p-2} (t-s)^2
 \]
 と書ける。ここで次の(i), (ii)がわかる。
 \begin{enumerate}[(i)]
  \item $t \geq s \geq 0$ならば、$F(t, s) \geq 0$である。
  \item $t \geq s \geq 0$のとき、$F(t, s) = 0$であることは、
        $t = s$であることと同値である。
 \end{enumerate}
 $u \geq \underline{u}_{\bar{\lambda}} ~\tin \Omega$であることに注意する。
 (i)、$b, \phi_1 > 0 ~\tin \Omega$、および、\eqref{eq:unique_eq_zero}
 より、$F(u, \underline{u}_{\bar{\lambda}}) = 0 ~\ae ~\tin \Omega$であ
 る。さらに(ii)より、$u = \underline{u}_{\bar{\lambda}} ~\ae ~\tin
 \Omega$である。つまり、$H_0^1(\Omega)$の元として、
 $u = \underline{u}_{\bar{\lambda}}$である。すなわち、
 \ref{eq:prob_main}の extremal solution は、
 $\underline{u}_{\bar{\lambda}}$に限る。 \qedhere
\end{proof}

\begin{proof}[定理~\ref{thm:extremal_solution}]
 命題~\ref{prop:ext_exi}.1 と命題~\ref{prop:ext_uni} からしたがう。\qedhere
\end{proof}

% Local Variables:
% mode: yatex
% TeX-master: "main.tex"
% End: