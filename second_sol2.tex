%#!platex main.tex
\section{second solutionの存在 2 --- 命題~\ref{prop:second_2}の証明}

本節では、命題~\ref{prop:second_2}を証明する。
本節を通し、定理~\ref{thm:second_solution}の仮定をおく。
必要ならば$\Omega$を平行移動することにより、$p = 0$としてよい。
以降$p = 0$とする。

\subsection{タレンティー関数の考察}

本小節では、命題~\ref{prop:second_2}の証明の鍵となる
タレンティー関数を考察する。
命題~\ref{prop:second_2}の$v_0$は、タレンティー関数を
加工することにより得られる。そこで本小節では、
次小節で必要となる具体的計算を実行する。

まずは、タレンティー関数を定義する。
\begin{defn}
 {\bf タレンティー関数}$U \colon \R^N \to \R$を
 \[
   U(x) = \frac{1}{(1 + \lvert x \rvert^2)^{(N-2)/2}}
 \]
 と定める。
\end{defn}

$U$について、
以下の事実が知られている。

\begin{lem}
 タレンティー関数$U$について、次式が成立する。
 \begin{equation}
  S = \frac{\left\| DU \right\|_{L^2(\R^N)}^2}
   {\left\| U \right\|_{L^{p+1}(\R^N)}^2}. 
   \label{eq:S_ach}
 \end{equation}
 すなわち、\eqref{eq:S_def}の右辺の下限は、$V = \R^N$のとき、
 $U$により達成される。
\end{lem}

$\Omega$上の cut off function $\eta$を、
$\eta \in C^\infty_c (\Omega)$、$0 \leq \eta \leq 1 ~\tin
\Omega$、
$\{ \lvert x \rvert \leq r_0 \}$上$\eta \equiv 1$、
$\{ \lvert x \rvert \geq 2r_0 \}$上$\eta \equiv 0$となるものとする。
$\epsilon > 0$とする。
$\Omega$上の関数$u_\epsilon$を
\[
 u_\epsilon (x) = \frac{\eta(x)}{(\epsilon + \lvert x \rvert^2)^{(N-2)/2}}
\]
と定める。\cite{MR709644}~より、次式が成立する。
\begin{equation}
 \left\| Du_\epsilon \right\|_{L^2(\Omega)}^2 
  = \left\| DU \right\|_{L^2(\Omega)}^2 \epsilon^{-(N-2)/2} + O(1).
  \label{eq:Duepsilon}
\end{equation}

次に、$\left\| b^{1/(p+1)} u_\epsilon \right\|_{L^{p+1}(\Omega)}^2$
を考察する。
\[
 \int_\Omega b u_{\epsilon}^{p+1} dx
 = \int_\Omega \frac{b(x) \eta(x)^{p+1}}{(\epsilon + \lvert x
 \rvert^2)^N} dx 
 = O(1) + \int_{ \{ \lvert x \rvert < r_0 \} } 
 \frac{b(x)}{(\epsilon + \lvert x
 \rvert^2)^N} dx.
\]
最左辺の積分を$I$とおく。ここで$q$と$N$の大小により
場合分けをする。

\ulinej{{$q < N$}のとき}:変数変換により、
\[
 I = \int_{ \{ \lvert x \rvert < r_0 \} } 
 \frac{M_1 - M_2 \lvert x \rvert^q}{(\epsilon + \lvert x
 \rvert^2)^N} dx
 = \frac{M_1}{\epsilon^{N/2}} \int_{ \left\{ \lvert x \rvert <
 \frac{r_0}{\sqrt{\epsilon}} \right\}} \frac{1}{(1 + \lvert x
 \rvert^2)^N}
 - \frac{M_2}{\epsilon^{(N-q)/2}} \int_{ \left\{ \lvert x \rvert <
 \frac{r_0}{\sqrt{\epsilon}} \right\}}
 \frac{ \lvert x \rvert^q}{(1 + \lvert x
 \rvert^2)^N}
\]
である。第$1$項の積分を$I_1(\epsilon)$、第$2$項の積分を$I_2(\epsilon)$
とおく。$\epsilon \searrow 0$のとき、
$I_1(\epsilon) \to \left\| U \right\|_{L^{p+1}(\Omega)}^{p+1}$である。
$q < N$であるから、$I_2(\epsilon)$は有限の値に収束する。

\subsection{命題~\ref{prop:second_2}の証明}