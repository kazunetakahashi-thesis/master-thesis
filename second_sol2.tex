\section{second solutionの存在 2 --- 命題~\ref{prop:second_2}の証明} \label{sec:second_sol2}

本節では,命題~\ref{prop:second_2}を証明する.
本節を通し,定理~\ref{thm:second_solution}の仮定をおく.
必要ならば$\Omega$を平行移動することにより,$x_0 = 0$としてよい.
以降$x_0 = 0$とする.

\subsection{タレンティー関数の考察}

本小節では,命題~\ref{prop:second_2}の証明の鍵となる
タレンティー関数を考察する.
命題~\ref{prop:second_2}の$v_0$は,タレンティー関数を
加工することにより得られる.そこで本小節では,
次小節で必要となる具体的計算を実行する.

まずは,タレンティー関数を定義する.
\begin{defn}
 {\bf タレンティー関数}$U \colon \R^N \to \R$を
 \[
   U(x) = \frac{1}{(1 + \lvert x \rvert^2)^{(N-2)/2}}
 \]
 と定める.
\end{defn}

$U$について,
以下の事実が知られている.

\begin{lem}[\cite{MR0463908}]
 タレンティー関数$U$について,次式が成立する.
 \begin{equation}
  S = \frac{\left\| DU \right\|_{L^2(\R^N)}^2}
   {\left\| U \right\|_{L^{p+1}(\R^N)}^2}. 
   \label{eq:S_ach}
 \end{equation}
 すなわち,\eqref{eq:S_def}の右辺の下限は,$V = \R^N$のとき,
 $U$により達成される.
\end{lem}

\begin{nota}
$\Omega$上の cut off function $\eta$を,
$\eta \in C^\infty_c (\Omega)$,$0 \leq \eta \leq 1 ~\tin
\Omega$,
$\{ \lvert x \rvert \leq r_0 \}$上$\eta \equiv 1$,
$\{ \lvert x \rvert \geq 2r_0 \}$上$\eta \equiv 0$となるものとする.
$\epsilon > 0$とする.
$\Omega$上の関数$u_\epsilon, v_\epsilon$を,
\begin{align}
 u_\epsilon (x) &= \frac{\eta(x)}{(\epsilon + \lvert x
 \rvert^2)^{(N-2)/2}}, \label{eq:def_uepsilon} \\
 v_\epsilon (x) &= \frac{u_\epsilon(x)}{\left\| b^{1/(p+1)} u_\epsilon \right\|_{L^{p+1}(\Omega)}} \label{eq:def_vepsilon}
\end{align}
と定める.
\end{nota}

$m_1 < 0$,または,$m_1 = 0$かつ$m_2 < 0$のとき,
必要なら$r_0 > 0$を小さくすると,$a \leq 0 ~\tin
~\{ \lvert x \rvert < 2r_0\}$が成立する.ゆえに,
\begin{equation}
 \int_\Omega av_\epsilon^2 dx \leq 0 \label{eq:intav^2_leq_0}
\end{equation}
が成立する.他方,
$m_1 > 0$,または,$m_1 = 0$かつ$m_2 > 0$のとき,
必要なら$r_0 > 0$を小さくすると,$a \geq 0 ~\tin
~\{ \lvert x \rvert < 2r_0\}$が成立する.ゆえに,
\begin{equation}
 \int_\Omega av_\epsilon^2 dx \geq 0 \label{eq:intav^2_geq_0}
\end{equation}
が成立する.

さて,\cite{MR709644}~の p.~444 より,次式が成立する.
\begin{equation}
 \left\| Du_\epsilon \right\|_{L^2(\Omega)}^2 
  = \left\| DU \right\|_{L^2(\R^N)}^2 \epsilon^{-(N-2)/2} + O(1).
  \label{eq:Duepsilon}
\end{equation}

次に,$\left\| b^{1/(p+1)} u_\epsilon \right\|_{L^{p+1}(\Omega)}^2$
を考察する.
\[
 \int_\Omega b u_{\epsilon}^{p+1} dx
 = \int_\Omega \frac{b(x) \eta(x)^{p+1}}{(\epsilon + \lvert x
 \rvert^2)^N} dx 
 = O(1) + \int_{ \{ \lvert x \rvert < r_0 \} } 
 \frac{b(x)}{(\epsilon + \lvert x
 \rvert^2)^N} dx.
\]
最左辺の積分を$I$とおく.
$e > 0$とする.$\delta > 0$が存在し,$\lvert x \rvert < \delta$ならば,
$M - e \leq b(x) \leq M$が成立する.必要ならば,$\delta > 0$を
小さくとりなおすと,$\delta < r_0$となる.
この$e, \delta$は,$\epsilon$とは無関係であることに
注意されたい.ここで$\tilde{I}$,$\hat{I}$を
\begin{align*}
 \tilde{I} &= \int_{ \{ \lvert x \rvert < \delta \} } 
 \frac{b(x)}{(\epsilon + \lvert x
 \rvert^2)^N} dx, \\
 \hat{I} &= \int_{ \{ \lvert x \rvert < \delta \} } 
 \frac{1}{(\epsilon + \lvert x
 \rvert^2)^N} dx
\end{align*}
とおく.$I = O(1) + \tilde{I}$,$(M - e) \hat{I}
 \leq \tilde{I} \leq M \hat{I}$である.変数変換により,
\[
 \hat{I} 
 = \frac{1}{\epsilon^{N/2}} \int_{ \left\{ \lvert x \rvert <
 \delta/\sqrt{\epsilon} \right\}} \frac{1}{(1 + \lvert x
 \rvert^2)^N}dx
\]
が得られる.
\[
 I(\epsilon) = \int_{ \left\{ \lvert x \rvert <
 \delta/\sqrt{\epsilon} \right\}} \frac{1}{(1 + \lvert x
 \rvert^2)^N} dx
\]
とすると,単調収束定理より,
\[
 \lim_{\epsilon \searrow 0} I(\epsilon) =
 \int_{\R^N} \frac{1}{(1 + \lvert x
 \rvert^2)^N} dx = \left\| U \right\|_{L^{p+1}(\Omega)}^{p+1}
\]
がわかる.$\tilde{I}$と$I(\epsilon)$の関係は,
\[
 \frac{(M - e)I(\epsilon)}{\epsilon^{N/2}} \leq \tilde{I} \leq \frac{M
 I(\epsilon)}{\epsilon^{N/2}}
\]
である.ここで$J(\epsilon) = I - \tilde{I}$とおく.
$J(\epsilon) = O(1)$であり,
\[
 \frac{(M - e)I(\epsilon)}{\epsilon^{N/2}} + J(\epsilon)
 \leq I \leq \frac{M
 I(\epsilon)}{\epsilon^{N/2}} + J(\epsilon)
\]
両辺を$2/(p+1)$乗すると,次式が得られる.
\begin{multline}
 \frac{(M -e)^{2/(p+1)}}{\epsilon^{(N-2)/2}} I(\epsilon)^{2/(p+1)}
 \left( 1 + \frac{\epsilon^{N/2}}{(M-e)I(\epsilon)} J(\epsilon)
 \right)^{2/(p+1)}  \\  \leq  \left\| b^{1/(p+1)} u_\epsilon
 \right\|^2_{L^{p+1}(\Omega)}  \\ \leq
 \frac{M^{2/(p+1)}}{\epsilon^{(N-2)/2}} I(\epsilon)^{2/(p+1)}
 \left( 1 + \frac{\epsilon^{N/2}}{MI(\epsilon)} J(\epsilon)
 \right)^{2/(p+1)} \label{eq:2/p+1pow} 
\end{multline}
最右辺,最左辺の括弧の中は,$\epsilon \searrow 0$のとき$1$に
収束する.$\{ \epsilon_n \}_{n=0}^\infty$を$n \to \infty$のとき
$\epsilon_n \searrow 0$となる任意の数列とする.
\eqref{eq:2/p+1pow},\eqref{eq:Duepsilon},および,
\eqref{eq:S_ach}より,
\begin{align}
 \liminf_{n \to \infty} \frac{\left\| Du_{\epsilon_n}
 \right\|_{L^2(\Omega)}^2}{\left\| b^{1/(p+1)} u_{\epsilon_n}
 \right\|_{L^{p+1}(\Omega)}^2} &\geq 
 \frac{\left\| DU \right\|_{L^2(\R^N)}^2}
   {M^{2/(p+1)} \left\| U \right\|_{L^{p+1}(\R^N)}^2} 
 = \frac{S}{M^{2/(p+1)}}, \label{eq:liminf_quo} \\ 
 \limsup_{n \to \infty} \frac{\left\| Du_{\epsilon_n}
 \right\|_{L^2(\Omega)}^2}{\left\| b^{1/(p+1)} u_{\epsilon_n}
 \right\|_{L^{p+1}(\Omega)}^2} &\leq 
 \frac{\left\| DU \right\|_{L^2(\R^N)}^2}
   {(M-e)^{2/(p+1)} \left\| U \right\|_{L^{p+1}(\R^N)}^2} 
 = \frac{S}{(M-e)^{2/(p+1)}} \label{eq:limsup_quo}
\end{align}
が得られる.$e > 0$は任意であるから,
\eqref{eq:limsup_quo}から
\begin{equation}
 \limsup_{n \to \infty} \frac{\left\| Du_{\epsilon_n}
 \right\|_{L^2(\Omega)}^2}{\left\| b^{1/(p+1)} u_{\epsilon_n}
 \right\|_{L^{p+1}(\Omega)}^2} \leq 
 \frac{S}{M^{2/(p+1)}} \label{eq:limsup_quo2}
\end{equation}
が従う.\eqref{eq:liminf_quo},\eqref{eq:limsup_quo2}より,
\[
 \lim_{n \to \infty} \frac{\left\| Du_{\epsilon_n}
 \right\|_{L^2(\Omega)}^2}{\left\| b^{1/(p+1)} u_{\epsilon_n}
 \right\|_{L^{p+1}(\Omega)}^2} =
 \frac{S}{M^{2/(p+1)}}
\]
であるとわかる.$\{ \epsilon_n \}$の任意性から,
\[
 \lim_{\epsilon \searrow 0} \frac{\left\| Du_{\epsilon}
 \right\|_{L^2(\Omega)}^2}{\left\| b^{1/(p+1)} u_{\epsilon}
 \right\|_{L^{p+1}(\Omega)}^2} =
 \frac{S}{M^{2/(p+1)}}
\]
が成立する.すなわち,次式が従う.
\begin{equation}
 \left\| v_\epsilon \right\|^2 = \left\| Dv_\epsilon
                                 \right\|^2_{L^2(\Omega)} 
 = \frac{S}{M^{2/(p+1)}} + O(\epsilon^{(N-2)/2}). \label{eq:vepsilon}
\end{equation}
また,議論の途中で次式も判明している.
\begin{equation}
 \left\| b^{1/(p+1)} u_\epsilon \right\|_{L^{p+1}(\Omega)}^2 = 
 O(\epsilon^{-(N-2)/2}). \label{eq:buepsilon}
\end{equation}

次に,
\[
 \int_\Omega au_\epsilon^2 dx = O(1) + \int_{ \{ \lvert x \rvert <
 r_0 \} } \frac{a(x)}{(\epsilon +
 \lvert x \rvert^2)^{N-2}} dx 
\]
を考察する.
$a(x)$の$o(\lvert x \rvert^q)$の項を$m_2 \neq 0$で割ったものを
$\theta(x)$と書く.つまり,$a$は,
\[
 a(x) = m_1 + m_2 \lvert x \rvert^q + m_2 \theta(x) 
 = m_1 + m_2 \lvert x \rvert^q \left( 1 + \frac{\theta(x)}{\lvert
 x \rvert^q } \right)
\]
と表される.
$I_1, I_2$を
\begin{align*}
 I_1 &= 
 \int_{ \{ \lvert x \rvert <
 r_0 \} } \frac{1}{(\epsilon +
 \lvert x \rvert^2)^{N-2}} dx, \\
 I_2 &= 
 \int_{ \{ \lvert x \rvert <
 r_0 \} } \frac{ \lvert x \rvert^q \left( 1 + \theta(x) / \lvert x
 \rvert^q \right)}{(\epsilon +
 \lvert x \rvert^2)^{N-2}} dx
\end{align*}
とおく.\cite{MR709644}~の p.~444 より,次式が成立する.
\[
 I_1 = \begin{cases}
        O(\epsilon^{-(N-4)/2}) & (N \geq 5),\\
        O(\lvert \log \epsilon \rvert) & (N = 4), \\ 
        O(1) & (N = 3).
       \end{cases}
\]
$0 < e < 1$とする.$\theta (x) = o(\lvert x \rvert^q)$より,
$\delta > 0$が存在し,$\lvert x \rvert < \delta$ならば,
$\lvert \theta (x) \rvert \leq e \lvert x \rvert^q$となる.
必要ならば$\delta > 0$を小さくし,$\delta < r_0$とする.
$\tilde{I}_2, \hat{I}_2$を
\begin{align*}
 \tilde{I}_2 &= 
 \int_{ \{ \lvert x \rvert <
 \delta \} } \frac{ \lvert x \rvert^q \left( 1 + \theta(x) / \lvert x
 \rvert^q \right)}{(\epsilon +
 \lvert x \rvert^2)^{N-2}} dx, \\ 
 \hat{I}_2 &= 
 \int_{ \{ \lvert x \rvert <
 \delta \} } \frac{ \lvert x \rvert^q }{(\epsilon +
 \lvert x \rvert^2)^{N-2}} dx
\end{align*}
と定める.$I_2 = O(1) + \tilde{I}_2$,
$(1 - e)\hat{I} \leq \tilde{I}_2 \leq (1 + e)\hat{I}_2$,
$e$は$\epsilon$と無関係であるから,
$O(I_2) = O(\tilde{I}_2) = O(\hat{I}_2)$である.
$\hat{I}_2$を,$N$と$q + 4$の大小で場合分けして計算する.

\ulinej{{$N > q + 4$}のとき}:変数変換により,
\[
 \hat{I}_2 = \frac{1}{\epsilon^{(N-q - 4)/2}} \int_{ \{ \lvert x
 \rvert < \delta / \sqrt{\epsilon}\}} 
 \frac{ \lvert x \rvert^{q}}{(1 +
 \lvert x \rvert^2)^{N-2}} dx
\]
である.
\[
 I(\epsilon) = \int_{ \{ \lvert x
 \rvert < \delta / \sqrt{\epsilon}\}} 
 \frac{ \lvert x \rvert^{q}}{(1 +
 \lvert x \rvert^2)^{N-2}} dx
\]
とおく.$N > q + 4$であるから,$\epsilon \searrow 0$のとき,
$I(\epsilon)$は収束する.よって,$I_1 =
O(\epsilon^{-(N-q-4)/2})$である.

\ulinej{{$N = q + 4$}のとき}:極座標変換により,
\[
 \hat{I}_2 = \mathrm{vol}(S^{N-1}) \int_0^{\delta} \frac{
 r^{N-4}}{(\epsilon + r^2)^{N-2}} r^{N-1} dr =
 O(\lvert \log \epsilon \rvert)
\]
と計算される.ここで$ \mathrm{vol}(S^{N-1})$は半径$1$の
$(N-1)$次元球面の標準計量における$(N-1)$次元体積である.

\ulinej{{$N < q + 4$}のとき}:
\[
 \hat{I}_2 < \int_{ \{ \lvert x \rvert < \delta \}} \lvert x \rvert^{q -
 2(N-2)} dx < \infty
\]
であるから,$\hat{I}_2 = O(1)$である.

以上より,$\epsilon \searrow 0$のときの$I_2$の挙動は次の通りにまとめられる.
\[
 I_2 = \begin{cases}
        O(\epsilon^{-(N-q-4)/2}) & (N > q + 4),\\
        O(\lvert \log \epsilon \rvert) & (N = q + 4), \\ 
        O(1) & (N < q + 4).
       \end{cases}
\]
以上の結果と,\eqref{eq:buepsilon}より,以下が成立する.
\begin{equation}
 \left\{ 
 \begin{aligned}
  \int_\Omega a v_\epsilon^2 dx &= O(\epsilon^{(N-2)/2})
  + m_1 I_1^\prime + m_2
  I_2^\prime, \\
  I_1^\prime &= \begin{cases}
                 O(\epsilon) & (N \geq 5), \\
                 O(\epsilon \lvert \log \epsilon \rvert) & (N = 4), \\
                 O(\epsilon^{1/2}) & (N = 3),
                \end{cases} \\
  I_2^\prime &= \begin{cases}
                 O(\epsilon^{1 + q/2 }) & (N > q + 4), \\
                 O(\epsilon^{(N-2)/2} \lvert \log \epsilon \rvert) & (N =
                 q + 4), \\
                 O(\epsilon^{(N-2)/2}) & (N < q + 4).
                \end{cases}
 \end{aligned} \right. \label{eq:av_epsilon}
\end{equation}
全ての$N, q$に対し,$i = 1, 2$それぞれ,
$I^{\prime}_i \gg \epsilon^{(N-2)/2}$または
$I^{\prime}_i = O(\epsilon^{(N-2)/2})$が成立していることに
注意されたい.また,$I^\prime_1 \gg I^\prime_2$である.
したがって,次式が成立する.
\[
 \int_\Omega av_\epsilon^2 dx = \begin{cases}
                                 O(I_1^\prime) & (m_1 \neq 0), \\ 
                                 O(I_2^\prime) & (m_1 = 0).
                                \end{cases}
\]

\subsection{命題~\ref{prop:second_2}の証明}

本小節では,補題を積み重ね,
命題~\ref{prop:second_2}に証明を与える.

\begin{lem} \label{lem:tauepsilon}
 $\tau_\epsilon = \left\| v_\epsilon \right\|^{2/(p-1)}$とする.
 このとき,次式が成立する.
 \begin{equation}
  \lim_{\epsilon \searrow 0} I_\lambda (\tau_\epsilon v_\epsilon) = 
   \frac{1}{NM^{(N-2)/2}} S^{N/2}. \label{eq:limI}
 \end{equation}
\end{lem}

\begin{proof}
 \eqref{eq:def_vepsilon}より,
 \[
  \int_\Omega bv_\epsilon^{p+1} dx = 1
 \]
 であるから,$t \geq 0$に対し,次式が得られる.
 \[
  I_\lambda(tv_\epsilon) = \frac{1}{2} t^2 \left\| v_\epsilon
 \right\|^2 - \frac{1}{p+1} t^{p+1} - \int_\Omega H(tv_\epsilon,
 \underline{u}_\lambda) dx.
 \]
 したがって,次式が成立する.
 \[
  I_\lambda(\tau_\epsilon v_\epsilon) = \frac{1}{N} \left( \left\|
 v_\epsilon \right\|^2 \right)^{N/2} - \int_\Omega H(\tau_\epsilon
 v_\epsilon, \underline{u}_\lambda) dx.
 \]
 \eqref{eq:vepsilon}より,以下が従う.
 \begin{align*}
  \lim_{\epsilon \searrow 0} \tau_\epsilon
  &= \frac{S^{2/(p-1)}}{M^{2/(p+1)(p-1)}}, \\
  \lim_{\epsilon \searrow 0} \frac{1}{N} \left( \left\|
 v_\epsilon \right\|^2 \right)^{N/2} &= \frac{1}{NM^{(N-2)/2}}
  S^{N/2}.
 \end{align*}
 $\epsilon \searrow 0$のとき,$\tau_\epsilon v_\epsilon \to 0 ~\ae
 \tin \Omega$である.ゆえに,補題~\ref{lem:conv}より,次式が成立する.
 \[
  \lim_{\epsilon \searrow 0} \int_\Omega H(\tau_\epsilon
 v_\epsilon, \underline{u}_\lambda) dx = 0.
 \]
 以上より,\eqref{eq:limI}が得られる.\qedhere
\end{proof}

\begin{lem} \label{lem:t_epsilon}
$\sup_{t > 0} I_\lambda (tv_\epsilon)$を達成する$t > 0$が存在する.
\end{lem}

\begin{proof}
補題~\ref{lem:mountain_dec}より,$t \to \infty$のとき,
$I_\lambda(t v_\epsilon) \to -\infty$となる.
したがって,ある$K>0$が存在し,$K < t$においては,
$I_\lambda(t v_\epsilon) < 0$となる.
また,$I_\lambda(0) = 0$である.
ゆえに,$\sup_{t > 0} I_\lambda (tv_\epsilon) = 
\sup_{t \in [0, K]} I_\lambda (tv_\epsilon)$となる.
$I_\lambda(tv_\epsilon)$は$t$についての連続関数であるから,
$\sup_{t > 0} I_\lambda (tv_\epsilon)$を達成する$t > 0$が存在する.\qedhere
\end{proof}

\begin{nota}
 補題~\ref{lem:t_epsilon}
 の$t$を$t_\epsilon$とかく.$t_\epsilon > 0$であり,
 次式が成立する.
 \begin{equation}
  I_\lambda(t_\epsilon v_\epsilon) = \sup_{t > 0} I_\lambda(t
   v_\epsilon).
   \label{eq:t_epsilon}
 \end{equation}
\end{nota}

\begin{lem} \label{lem:intHprime}
 $\epsilon_0 > 0$と$C > 0$が存在し,$0 < \epsilon < \epsilon_0$に対し,
 \begin{equation}
  \int_\Omega H^\prime(t_\epsilon v_\epsilon, \underline{u}_\lambda)
   dx \geq C\epsilon^{(N-2)/4} \label{eq:int_Hprime}
 \end{equation}
 が成立する.
\end{lem}

\begin{proof}
 まず,次式を背理法を用いて証明する.
 \begin{equation}
  \liminf_{\epsilon \searrow 0} t_\epsilon > 0. \label{eq:liminf_tepsilon}
 \end{equation}
 \eqref{eq:liminf_tepsilon}の意味は,$0$に収束する
 任意の正の実数列$\{ \epsilon_n \}_{n=0}^\infty$に対し,
 $\liminf_{n \to \infty} t_{\epsilon_n} > 0$であるということである.
 \eqref{eq:liminf_tepsilon}を否定し,
 $\liminf_{\epsilon \searrow 0} t_\epsilon = 0$であることを仮定する.
 $s, t \geq 0$,$x \in \Omega$に対し,
 $G^\prime(s, t, x) \geq 0$であることを確かめる.
 テイラーの定理より,
 \[
  G^\prime(t, s, x) = b(x) \left( \frac{1}{p+1}(t+s)^{p+1} -
 \frac{1}{p+1}s^{p+1} - s^p t \right)
 = b(x) p (s + \theta t)^{p-1} t^2
 \]
 となる$0 < \theta < 1$が存在する.最右辺は$0$以上である.
 ゆえに$G^\prime(s, t, x) \geq 0$が成立する.
 そこで,
 \begin{equation}
  I_\lambda(t_\epsilon v_\epsilon) = \frac{1}{2} t_\epsilon^2 
   \int_\Omega \left( \lvert Dv_\epsilon \rvert^2 +  a v_\epsilon^2
               \right) dx
  - \int_\Omega G^\prime(t_\epsilon v_\epsilon, \underline{u}_\lambda) dx \leq
  \frac{1}{2} t_\epsilon^2 
  \int_\Omega \left( \lvert Dv_\epsilon \rvert^2 + a v_\epsilon^2
              \right) dx \leq  C t_\epsilon^2 \left\|
              v_\epsilon \right\|^2 
 \end{equation}
 において,$\epsilon \searrow 0$における下極限をとると,
 仮定と\eqref{eq:vepsilon}により
 \begin{equation}
  \liminf_{\epsilon \searrow 0} I_\lambda (t_\epsilon v_\epsilon) \leq
   0 \label{eq:liminf_cont_1}
 \end{equation}
 が従う.一方で,\eqref{eq:t_epsilon}より
 $I_\lambda(t_\epsilon v_\epsilon) \geq I_\lambda
 (\tau_\epsilon v_\epsilon)$で
 あるから,補題~\ref{lem:tauepsilon}より,
 \begin{equation}
  \liminf_{\epsilon \searrow 0} I_\lambda(t_\epsilon v_\epsilon) \geq
   \lim_{\epsilon \searrow 0} I_\lambda (\tau_\epsilon v_\epsilon) =
   \frac{1}{NM^{(N-2)/2}} S^{N/2} > 0. \label{eq:lininf_cont_2}
 \end{equation}
 \eqref{eq:liminf_cont_1}と\eqref{eq:lininf_cont_2}は同時に
 成立しない.
 よって,背理法により,\eqref{eq:liminf_tepsilon}が従う. 

さて,\eqref{eq:liminf_tepsilon}と\eqref{eq:buepsilon}より,
$\epsilon_0 > 0$,$C > 0$が存在し,$0 <\epsilon < \epsilon_0$のとき,
$\lvert x \rvert < r_0$に対し,
\begin{equation}
 t_\epsilon v_\epsilon (x) = t_\epsilon \frac{\eta(x)}{(\epsilon +
 \lvert x \rvert^2)^{(N-2)/2} \left\| b^{1/(p+1)} u_\epsilon
 \right\|_{L^{p+1}(\Omega)}} \geq \frac{C
 \epsilon^{(N-2)/4}}{(\epsilon + \lvert x \rvert^2)^{(N-2)/2}} 
 \label{eq:teve_p}
\end{equation}
が成立する.必要ならば$\epsilon_0 > 0$を小さくとりなおし,
$\sqrt{\epsilon_0} < r_0$が成立するとして良い.すると,
$\lvert x \lvert < \sqrt{\epsilon}$に対し,
\begin{equation}
 t_\epsilon v_\epsilon (x) \geq C_0 \epsilon^{-(N-2)/2} \label{eq:teve}
\end{equation}
となる.$C_0 > 0$は$\epsilon$によらない.
この$C_0$について,
\begin{equation}
 t_0 = C_0 \epsilon_0^{-(N-2)/2} \label{eq:t_0_epsilon}
\end{equation}
と定める.
また,強最大値原理より,必要なら$\epsilon_0 > 0$を小さくすると,
\begin{equation}
 \underline{u}_\lambda (x) > s_0 \ \ \tin \{ \lvert x \rvert \leq
  \sqrt{\epsilon_0} \}
  \label{eq:uus_0}
\end{equation}
を成立させる$x$によらない$s_0 > 0$が存在する.

ここで,$x \in \Omega$,$t \geq t_0$,$s \geq s_0$に対し,
\begin{equation}
 H^\prime (t, s, x) \geq C b(x) t^p \label{eq:Hprime_lb}
\end{equation}
を成り立たせる$t, s, x$によらない定数$C>0$が存在することを示す.
$s, t \geq 0$に対し,
\[
 H^\prime(t, s, x) = b(x) \left( \frac{1}{p+1}(t+s)^{p+1} -
 \frac{1}{p+1} t^{p+1} - \frac{1}{p+1} s^{p+1} - s^p t \right)
\]
である.そこで$s$についての偏導関数は,
\[
 H^\prime_s (t, s, x) = b(x) \left( (t+s)^p - s^p - ps^{p-1}t \right)
\]
である.右辺はテイラーの定理より,$0 < \theta < 1$を用いて
$p(p-1)(s + \theta t)^{p-2}t^2/2$と表される.これは非負であるから,
$H^\prime_s(t, s, x) \geq 0$である.すなわち,$H^\prime$は
$s$についての増加関数である.
したがって,$s \geq s_0$,$t \geq 0$に対し,
\begin{equation}
 H^\prime(t, s, x) \geq H^\prime(t, s_0, x) \label{eq:Hprime_lb_pf1}
\end{equation}
である.また,$s \geq 0$,$t \geq 0$に対し,
\begin{equation}
 H^\prime(t, s, x) \geq H^\prime(t, 0, x) = 0 \label{eq:Hprime_lb_pf3}
\end{equation}
もわかる.ここでテイラーの定理より,
\[
 \frac{1}{p+1} (t + s_0)^{p+1} - \frac{1}{p+1} t^{p+1} = (t + \theta
 s_0)^p s_0
\]
 をみたす$0 < \theta < 1$が存在する.ゆえに$t \geq t_0$に対し,
 以下が従う.
 \begin{align*}
  H^\prime(t, s_0, x) &\geq b(x) \left( (t+\theta s_0)^p s_0 -
  \frac{1}{p+1} s_0^{p+1} - s_0^p t\right) \\
  & \geq b(x) \left( t^p s_0 - \frac{1}{p+1} s_0^{p+1} - s_0^p t \right) \\
  & = t^p b(x) \left( s_0 - \frac{s_0^{p-1}}{p+1} \frac{1}{t^p} - s_0^p
  \frac{1}{t^{p-1}}  \right) \\
  & \geq t^p b(x) \left( s_0 - \frac{s_0^{p-1}}{p+1} \frac{1}{t_0^p} - s_0^p
  \frac{1}{t_0^{p-1}}  \right).
 \end{align*}
 ここで最右辺の括弧の中が正となるよう,
 必要ならば$\epsilon_0 > 0$を小さくとりなおす.
 $s_0$と$C_0$は$\epsilon_0$によっているが,
 $\lvert x \rvert < \sqrt{\epsilon_0}$に対して
 \eqref{eq:teve}および\eqref{eq:uus_0}を成り立たせるために
 $s_0$と$C_0$は変更する必要がないことに注意されたい.
 \eqref{eq:t_0_epsilon}により,
 最右辺の括弧の中が正となるよう,$t_0$を大きくすることができる.
 以上により,$t \geq t_0$に対し,
 \begin{equation}
  H^\prime (t, s_0, x) \geq C b(x) t^p \label{eq:Hprime_lb_pf2}
 \end{equation}
 が成立する.
 \eqref{eq:Hprime_lb_pf1}と\eqref{eq:Hprime_lb_pf2}より,
 $t \geq t_0$,$s \geq s_0$に対し,
 \eqref{eq:Hprime_lb}が従う.
 
 $b$は$\{ \lvert x \rvert < \sqrt{\epsilon_0}\}$上連続であるから,
 $\delta > 0$が存在し,$\lvert x \rvert < \sqrt{\epsilon_0}$ならば,
 $b(x) \geq M - \delta$となる.必要ならば$\epsilon_0 > 0$を
 小さく取り直せば,$M_1 - \delta > 0$とできる.
 \eqref{eq:Hprime_lb_pf3},\eqref{eq:Hprime_lb},\eqref{eq:teve_p}を
 順に使うと,
 $0 < \epsilon < \epsilon_0$に対し,以下が成立する.
 \begin{align*}
  \int_\Omega H^\prime(t_\epsilon v_\epsilon, \underline{u}_\lambda )
  dx & \geq \int_{\{ \lvert x \rvert \leq \sqrt{\epsilon} \}} 
  H^\prime(t_\epsilon v_\epsilon, \underline{u}_\lambda ) dx \\
  & \geq C \int_{\{ \lvert x \rvert \leq \sqrt{\epsilon} \}} 
  b(x) (t_\epsilon v_\epsilon)^p dx \\
  & \geq C \int_{\{ \lvert x \rvert \leq \sqrt{\epsilon} \}} 
  (M - \delta) \left(
  \frac{\epsilon^{(N-2)/4}}{(\epsilon + \lvert x \rvert^2)^{(N-2)/2}}
  \right)^p dx \\
  & \geq C \int_{\{ \lvert x \rvert \leq \sqrt{\epsilon} \}} 
  \left(
  \frac{\epsilon^{(N-2)/4}}{(\epsilon + \lvert x \rvert^2)^{(N-2)/2}}
  \right)^p dx \\
  & = C \epsilon^{(N-2)/4} \int_{ \{ \lvert y \rvert \leq 1 \}}
  \frac{1}{(1 + \lvert y \rvert^2)^{(N+2)/2}} dy \\
  & = C^\prime \epsilon^{(N-2)/4}.
 \end{align*}
 $C^\prime > 0$
 は$\epsilon$によらない.所望の\eqref{eq:int_Hprime}が得られた.\qedhere
\end{proof}

\begin{proof}[命題~\ref{prop:second_2}]
 \ulinej{{$m_1 < 0$},または,{$m_1 = 0$かつ$m_2 < 0$}のとき}:
 \eqref{eq:intav^2_leq_0}より,以下が成立する.
 \begin{align*}
  \sup_{t > 0} I_\lambda (t v_\epsilon) & = I_\lambda (t_\epsilon
  v_\epsilon) \\ 
  & = \frac{1}{2} t_\epsilon^2 \left\| v_\epsilon \right\|^2 -
  \frac{1}{p+1} t_\epsilon^{p+1} - \int_\Omega H^\prime(t_\epsilon
  v_\epsilon, \underline{u}_\lambda) dx + t_\epsilon^2
  \int_\Omega a v_\epsilon^2
  dx \\
  & \leq \frac{1}{2} t_\epsilon^2 \left\| v_\epsilon \right\|^2 -
  \frac{1}{p+1} t_\epsilon^{p+1} - \int_\Omega H^\prime(t_\epsilon
  v_\epsilon, \underline{u}_\lambda) dx \\
  & \leq \sup_{t > 0} \left( \frac{1}{2} t^2 \left\| v_\epsilon
  \right\|^2 - \frac{1}{p+1} t^{p+1} \right)
  - \int_\Omega H^\prime(t_\epsilon
  v_\epsilon, \underline{u}_\lambda) dx \\
  & = \frac{1}{N} \left( \left\| v_\epsilon \right\|^2 \right)^{N/2}
  - \int_\Omega H^\prime(t_\epsilon
  v_\epsilon, \underline{u}_\lambda) dx.
 \end{align*}
 ここで,最後の変形では,$t > 0$の関数
 \[
   \alpha(t) = \frac{1}{2} t^2 \left\| v_\epsilon \right\|^2 -
 \frac{1}{p+1} t^{p+1}
 \]
 が,$t = \left\| v_\epsilon \right\|^{2/(p-1)}$に
 おいて最大値をとることに注意した.
 \eqref{eq:vepsilon}より,$\epsilon \searrow 0$のとき,次式が成立する.
 \[
 \frac{1}{N} \left( \left\| v_\epsilon
 \right\|^2 \right)^{N/2} = \frac{1}{NM^{(N-2)/2}} S^{N/2}
 + O(\epsilon^{(N-2)/2}).
 \]
 補題~\ref{lem:intHprime}により,$\epsilon_0 > 0$,$C, C^\prime > 0$が
 存在し,$0 < \epsilon < \epsilon_0$に対し,
 \begin{equation}
  \sup_{t > 0} I_\lambda (tv_\epsilon) \leq \frac{1}{NM^{(N-2)/2}}
   S^{N/2} + \left( C \epsilon^{(N-2)/2} - C^\prime \epsilon^{(N-2)/4}
             \right) \label{eq:sup_least}
 \end{equation}
 が成立する.
 全ての$N \geq 3$に対し,$(N-2)/2 > (N-2)/4$であるから,
 \eqref{eq:sup_least}の右辺の括弧の中が負となる
 $\epsilon > 0$が存在する.
 この$\epsilon$を用いて$v_0 = v_\epsilon$とすると,$v_0 \geq 0 ~\tin
 \Omega$,\eqref{eq:int_omega_bvp+1},および,
 \eqref{eq:ineq_S}が成立する.すなわち,
 命題~\ref{prop:second_2}の帰結は成立する.
 
 \ulinej{{$m_1 > 0$},または,{$m_1 = 0$かつ$m_2 > 0$}のとき}:
 \eqref{eq:intav^2_geq_0}より,$\epsilon \searrow 0$のとき,
 以下が成立する.
 \begin{align*}
  \sup_{t > 0} I_\lambda (t v_\epsilon) & = I_\lambda (t_\epsilon
  v_\epsilon) \\ 
  & = \frac{1}{2} t_\epsilon^2 \left\| v_\epsilon \right\|^2 -
  \frac{1}{p+1} t_\epsilon^{p+1} - \int_\Omega H^\prime(t_\epsilon
  v_\epsilon, \underline{u}_\lambda) dx + t_\epsilon^2
  \int_\Omega a v_\epsilon^2
  dx \\
  & = \frac{1}{2} t_\epsilon^2 \left(
  \left\| v_\epsilon \right\|^2 + 2 \int_\Omega av_\epsilon^2 dx \right) -
  \frac{1}{p+1} t_\epsilon^{p+1} - \int_\Omega H^\prime(t_\epsilon
  v_\epsilon, \underline{u}_\lambda) dx \\
  & \leq \sup_{t > 0} \left( \frac{1}{2} t^2 \left(\left\| v_\epsilon
  \right\|^2 + 2 \int_\Omega av_\epsilon^2 dx \right)
  - \frac{1}{p+1} t^{p+1} \right)
  - \int_\Omega H^\prime(t_\epsilon
  v_\epsilon, \underline{u}_\lambda) dx \\
  & = \frac{1}{N} \left( \left\| v_\epsilon \right\|^2
  + 2 \int_\Omega av_\epsilon^2 dx \right)^{N/2}
  - \int_\Omega H^\prime(t_\epsilon
  v_\epsilon, \underline{u}_\lambda) dx.
 \end{align*}
 ここで,$A(\epsilon)$を,以下で定める.
 \[
  A(\epsilon) = \frac{1}{N}
 \left( \left\| v_\epsilon \right\|^2
 + 2 \int_\Omega av_\epsilon^2 dx \right)^{N/2} - 
 \frac{1}{NM^{(N-2)/2}} S^{N/2}.
 \]
\eqref{eq:vepsilon},\eqref{eq:av_epsilon}より,$\epsilon \searrow 0$
のとき,次式が成立する.
 \begin{equation}
  \begin{aligned}
   A(\epsilon) &=
   \begin{cases}
    O(\epsilon) & (m_1 > 0, N \geq 5), \\
    O(\epsilon \lvert \log \epsilon \rvert) & (m_1 > 0, N = 4), \\
    O(\epsilon^{1/2}) & (m_1 > 0, N = 3), \\
    O(\epsilon^{1 + q/2 }) & (m_1 = 0, m_2 > 0,  N > q + 4), \\
    O(\epsilon^{(N-2)/2} \lvert \log \epsilon \rvert) & (m_1 = 0, m_2 > 0,
    N = q + 4), \\
    O(\epsilon^{(N-2)/2}) & (m_1 = 0, m_2 > 0, N < q + 4).
   \end{cases}
  \end{aligned} \label{eq:Aepsilon}
 \end{equation}
 補題~\ref{lem:intHprime}により,$\epsilon_0 > 0$,$C^\prime > 0$が
 存在し,$0 < \epsilon < \epsilon_0$に対し,
 \begin{equation}
  \sup_{t > 0} I_\lambda (tv_\epsilon) \leq \frac{1}{NM^{(N-2)/2}}
   S^{N/2} + \left( A(\epsilon) - C^\prime \epsilon^{(N-2)/4}
             \right) \label{eq:sup_least_2}
 \end{equation}
 が成立する.先ほどと同様に,
 \eqref{eq:sup_least_2}の右辺の括弧の中を負とする
 $\epsilon > 0$が存在するならば,
 命題~\ref{prop:second_2}の帰結は成立する.
 $m_1 > 0$のとき,\eqref{eq:Aepsilon}より,$N = 3, 4, 5$ならば,
 $A(\epsilon) \ll \epsilon^{(N-2)/4}$となるため,
 所望の$\epsilon > 0$が存在する.
 また,$m_1 = 0$かつ$m_2 > 0$のとき,
 \eqref{eq:Aepsilon}より,
 $N \leq q+4$のときは,$A(\epsilon) \ll \epsilon^{(N-2)/4}$が成立する.
 $N > q+4$のときは,
 $A(\epsilon) \ll \epsilon^{(N-2)/4}$となるための条件は,
 $1+q/2 > (N-2)/4$である.これを変形すると,$N < 2q+6$が得られる.
 すなわち,$3 \leq N < 2q+6$のとき,
 所望の$\epsilon > 0$が存在する.\qedhere
\end{proof}

\begin{proof}[定理~\ref{thm:second_solution}]
 命題~\ref{prop:second_1}と命題~\ref{prop:second_2}より
 \ref{eq:prob_sec}は弱解$v$を持つ.
 これと補題~\ref{lem:rel_heart_spade}.2より,\ref{eq:prob_main}は
 minimal solution 以外の弱解を持つ.
 \qedhere
\end{proof}

%#!platex main.tex
% Local Variables:
% mode: yatex
% coding: utf-8
% TeX-master: "main.tex"
% End: