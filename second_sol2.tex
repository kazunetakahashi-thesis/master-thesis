%#!platex main.tex
\section{second solutionの存在 2 --- 命題~\ref{prop:second_2}の証明} \label{sec:second_sol2}

本節では、命題~\ref{prop:second_2}を証明する。
本節を通し、定理~\ref{thm:second_solution}の仮定をおく。
必要ならば$\Omega$を平行移動することにより、$x_0 = 0$としてよい。
以降$x_0 = 0$とする。

\subsection{タレンティー関数の考察}

本小節では、命題~\ref{prop:second_2}の証明の鍵となる
タレンティー関数を考察する。
命題~\ref{prop:second_2}の$v_0$は、タレンティー関数を
加工することにより得られる。そこで本小節では、
次小節で必要となる具体的計算を実行する。

まずは、タレンティー関数を定義する。
\begin{defn}
 {\bf タレンティー関数}$U \colon \R^N \to \R$を
 \[
   U(x) = \frac{1}{(1 + \lvert x \rvert^2)^{(N-2)/2}}
 \]
 と定める。
\end{defn}

$U$について、
以下の事実が知られている。

\begin{lem}[\cite{MR0463908}]
 タレンティー関数$U$について、次式が成立する。
 \begin{equation}
  S = \frac{\left\| DU \right\|_{L^2(\R^N)}^2}
   {\left\| U \right\|_{L^{p+1}(\R^N)}^2}. 
   \label{eq:S_ach}
 \end{equation}
 すなわち、\eqref{eq:S_def}の右辺の下限は、$V = \R^N$のとき、
 $U$により達成される。
\end{lem}

\begin{nota}
$\Omega$上の cut off function $\eta$を、
$\eta \in C^\infty_c (\Omega)$、$0 \leq \eta \leq 1 ~\tin
\Omega$、
$\{ \lvert x \rvert \leq r_0 \}$上$\eta \equiv 1$、
$\{ \lvert x \rvert \geq 2r_0 \}$上$\eta \equiv 0$となるものとする。
$\epsilon > 0$とする。
$\Omega$上の関数$u_\epsilon, v_\epsilon$を、
\begin{align}
 u_\epsilon (x) &= \frac{\eta(x)}{(\epsilon + \lvert x
 \rvert^2)^{(N-2)/2}}, \label{eq:def_uepsilon} \\
 v_\epsilon (x) &= \frac{u_\epsilon(x)}{\left\| b^{1/(p+1)} u_\epsilon \right\|_{L^{p+1}(\Omega)}} \label{eq:def_vepsilon}
\end{align}
と定める。
\end{nota}

さて、\cite{MR709644}~の p.~444 より、次式が成立する。
\begin{equation}
 \left\| Du_\epsilon \right\|_{L^2(\Omega)}^2 
  = \left\| DU \right\|_{L^2(\R^N)}^2 \epsilon^{-(N-2)/2} + O(1).
  \label{eq:Duepsilon}
\end{equation}

次に、$\left\| b^{1/(p+1)} u_\epsilon \right\|_{L^{p+1}(\Omega)}^2$
を考察する。
\[
 \int_\Omega b u_{\epsilon}^{p+1} dx
 = \int_\Omega \frac{b(x) \eta(x)^{p+1}}{(\epsilon + \lvert x
 \rvert^2)^N} dx 
 = O(1) + \int_{ \{ \lvert x \rvert < r_0 \} } 
 \frac{b(x)}{(\epsilon + \lvert x
 \rvert^2)^N} dx.
\]
最左辺の積分を$I$とおく。ここで$q$と$N$の大小により
場合分けをする。

\ulinej{{$q < N$}のとき}:変数変換により、
\[
 I = \int_{ \{ \lvert x \rvert < r_0 \} } 
 \frac{M_1 - M_2 \lvert x \rvert^q}{(\epsilon + \lvert x
 \rvert^2)^N} dx
 = \frac{M_1}{\epsilon^{N/2}} \int_{ \left\{ \lvert x \rvert <
 r_0/\sqrt{\epsilon} \right\}} \frac{1}{(1 + \lvert x
 \rvert^2)^N}dx
 - \frac{M_2}{\epsilon^{(N-q)/2}} \int_{ \left\{ \lvert x \rvert <
 r_0/\sqrt{\epsilon} \right\}}
 \frac{ \lvert x \rvert^q}{(1 + \lvert x
 \rvert^2)^N}dx
\]
である。第$1$項の積分を$I_1(\epsilon)$、第$2$項の積分を$I_2(\epsilon)$
とおく。$\epsilon \searrow 0$のとき、
$I_1(\epsilon) \to \left\| U \right\|_{L^{p+1}(\Omega)}^{p+1}$である。
$q < N$であるから、$I_2(\epsilon)$は有限の値に収束する。
\[
 \left\| b^{1/(p+1)} u_\epsilon \right\|_{L^{p+1}(\Omega)}^2 = 
 \frac{M_1^{2/(p+1)}}{\epsilon^{(N-2)/2}} I_1(\epsilon)^{1/(p+1)} 
 - \frac{M_2^{2/(p+1)}}{\epsilon^{(N-2)(N-q)/2N}}
 I_2(\epsilon)^{1/(p+1)} + O(1)
\]
であるから、\eqref{eq:Duepsilon}および\eqref{eq:S_ach}より、
\begin{equation}
 \lim_{\epsilon \searrow 0} 
  \frac{\left\| Du_\epsilon
        \right\|_{L^2(\Omega)}^2}
  {\left\| b^{1/(p+1)} u_\epsilon \right\|_{L^{p+1}(\Omega)}^2}
  = \frac{\left\| DU \right\|_{L^2(\R^N)}^2}{M_1^{2/(p+1)} 
  \left\| U \right\|_{L^{p+1}(\R^N)}^2} = \frac{S}{M_1^{2/(p+1)}}
  \label{eq:vepsilon_S}
\end{equation}
と計算される。すなわち、次式がしたがう。
\begin{equation}
 \left\| v_\epsilon \right\|^2 = \left\| Dv_\epsilon
                                 \right\|^2_{L^2(\Omega)} 
 = \frac{S}{M^{2/(p+1)}} + O(\epsilon^{(N-2)/2}). \label{eq:vepsilon}
\end{equation}

\ulinej{{$q = N$}のとき}:極座標変換をすると、次式が得られる。
\[
 \int_{ \{ \lvert x \rvert < r_0 \} } 
 \frac{\lvert x \rvert^q}{(\epsilon + \lvert x
 \rvert^2)^N} dx
 = \mathrm{vol}(S^{N-1}) \int_0^{r_0} \frac{r^N}{(\epsilon + r^2)^N}
 r^{N-1} dr = O(\lvert \log \epsilon \rvert).
\]
ここで$\mathrm{vol}(S^{N-1})$は半径$1$の$(N-1)$次元球面の体積である。
ゆえに、
\eqref{eq:vepsilon_S}、
\eqref{eq:vepsilon}が同様にしたがう。

\ulinej{{$q > N$}のとき}:
\[
 \int_{ \{ \lvert x \rvert < r_0 \} } 
 \frac{\lvert x \rvert^q}{(\epsilon + \lvert x
 \rvert^2)^N} dx < 
 \int_{ \{ \lvert x \rvert < r_0 \} } 
 \lvert x \rvert^{q-2N} dx < \infty
\]
であるから、最右辺は$O(1)$である。ゆえに、やはり
\eqref{eq:vepsilon_S}、
\eqref{eq:vepsilon}がしたがう。
いずれの場合でも、
\begin{equation}
 \left\| b^{1/(p+1)} u_\epsilon \right\|_{L^{p+1}(\Omega)}^2 = 
 O(\epsilon^{-(N-2)/2}) \label{eq:buepsilon}
\end{equation}
である。

次に、
\[
 \int_\Omega au_\epsilon^2 dx = O(1) + \int_{ \{ \lvert x \rvert <
 r_0 \} } \frac{m_1 + m_2 \lvert x \rvert^{q^\prime}}{(\epsilon +
 \lvert x \rvert^2)^{N-2}} dx 
\]
を考察する。$I_1, I_2$を
\begin{align*}
 I_1 &= 
 \int_{ \{ \lvert x \rvert <
 r_0 \} } \frac{1}{(\epsilon +
 \lvert x \rvert^2)^{N-2}} dx, \\
 I_2 &= 
 \int_{ \{ \lvert x \rvert <
 r_0 \} } \frac{ \lvert x \rvert^{q^\prime}}{(\epsilon +
 \lvert x \rvert^2)^{N-2}} dx
\end{align*}
とおく。\cite{MR709644}~の p.~444 より、次式が成立する。
\[
 I_1 = \begin{cases}
        O(\epsilon^{-(N-4)/2}) & (n \geq 5),\\
        O(\lvert \log \epsilon \rvert) & (n = 4), \\ 
        O(1) & (n = 3).
       \end{cases}
\]
$I_2$を、$N$と$q^\prime + 4$の大小で場合分けして計算する。

\ulinej{{$N > q^\prime + 4$}のとき}:変数変換により、
\[
 I_2 = \frac{1}{\epsilon^{(N-q^\prime - 4)/2}} \int_{ \{ \lvert x
 \rvert < r_0 / \sqrt{\epsilon}\}} 
 \frac{ \lvert x \rvert^{q^\prime}}{(1 +
 \lvert x \rvert^2)^{N-2}} dx
\]
である。右辺の積分を$I(\epsilon)$とおく。
$N > q^\prime + 4$であるから、$\epsilon \searrow 0$のとき、
$I(\epsilon)$は収束する。よって、$I_1 =
O(\epsilon^{-(N-q^\prime-4)/2})$である。

\ulinej{{$N = q^\prime + 4$}のとき}:極座標変換により、
\[
 I_2 = \mathrm{vol}(S^{N-1}) \int_0^{r_0} \frac{ \lvert x
 \rvert^{N-4}}{(\epsilon + \lvert x \rvert^2)^{N-2}} r^{N-1} dr =
 O(\lvert \log \epsilon \rvert)
\]
と計算される。

\ulinej{{$N < q^\prime + 4$}のとき}:
\[
 I_2 < \int_{ \{ \lvert x \rvert < r_0 \}} \lvert x \rvert^{q^\prime -
 2(N-2)} dx < \infty
\]
であるから、$I_2 = O(1)$である。

よって、$\epsilon \searrow 0$のときの$I_2$の挙動は次の通りにまとめられる。
\[
 I_2 = \begin{cases}
        O(\epsilon^{-(N-q^\prime-4)/2}) & (N > q^\prime + 4),\\
        O(\lvert \log \epsilon \rvert) & (N = q^\prime + 4), \\ 
        O(1) & (N < q^\prime + 4).
       \end{cases}
\]
以上の結果と、\eqref{eq:buepsilon}より、以下が成立する。
\begin{equation}
 \left\{ 
 \begin{aligned}
  \int_\Omega a v_\epsilon^2 dx &= O(1) + m_1 I_1^\prime + m_2
  I_2^\prime, \\
  I_1^\prime &= \begin{cases}
                 O(\epsilon) & (N \geq 5), \\
                 O(\epsilon \lvert \log \epsilon \rvert) & (N = 4), \\
                 O(\epsilon^{1/2}) & (N = 3),
                \end{cases} \\
  I_2^\prime &= \begin{cases}
                 O(\epsilon^{1 + q^\prime/2 }) & (N > q^\prime + 4), \\
                 O(\epsilon^{(N-2)/2} \lvert \log \epsilon \rvert) & (N =
                 q^\prime + 4), \\
                 O(\epsilon^{(N-2)/2}) & (N < q^\prime + 4).
                \end{cases}
 \end{aligned} \right. \label{eq:av_epsilon}
\end{equation}

\subsection{命題~\ref{prop:second_2}の証明}

本小節では、補題を積み重ね、
命題~\ref{prop:second_2}に証明を与える。

\begin{lem} \label{lem:tauepsilon}
 $\tau_\epsilon = \left\| v_\epsilon \right\|^{2/(p-1)}$とする。
 このとき、次式が成立する。
 \begin{equation}
  \lim_{\epsilon \searrow 0} I_\lambda (\tau_\epsilon v_\epsilon) = 
   \frac{1}{NM_1^{(N-2)/2}} S^{N/2}. \label{eq:limI}
 \end{equation}
\end{lem}

\begin{proof}
 \eqref{eq:def_vepsilon}より、
 \[
  \int_\Omega bv_\epsilon^{p+1} dx = 1
 \]
 であるから、$t \geq 0$に対し、次式が得られる。
 \[
  I_\lambda(tv_\epsilon) = \frac{1}{2} t^2 \left\| v_\epsilon
 \right\|^2 - \frac{1}{p+1} t^{p+1} - \int_\Omega H(tv_\epsilon,
 \underline{u}_\lambda) dx.
 \]
 したがって、次式が成立する。
 \[
  I_\lambda(\tau_\epsilon v_\epsilon) = \frac{1}{N} \left( \left\|
 v_\epsilon \right\|^2 \right)^{N/2} - \int_\Omega H(\tau_\epsilon
 v_\epsilon, \underline{u}_\lambda) dx.
 \]
 \eqref{eq:vepsilon}より、以下がしたがう。
 \begin{align*}
  \lim_{\epsilon \searrow 0} \tau_\epsilon
  &= \frac{S^{1/(p-1)}}{M_1^{2/(p+1)(p-1)}}, \\
  \lim_{\epsilon \searrow 0} \frac{1}{N} \left( \left\|
 v_\epsilon \right\|^2 \right)^{N/2} &= \frac{1}{NM_1^{(N-2)/2}}
  S^{N/2}.
 \end{align*}
 $\epsilon \searrow 0$のとき、$\tau_\epsilon v_\epsilon \to 0 ~\ae
 \tin \Omega$である。ゆえに、補題~\ref{lem:conv}より、次式が成立する。
 \[
  \lim_{\epsilon \searrow 0} \int_\Omega H(\tau_\epsilon
 v_\epsilon, \underline{u}_\lambda) dx = 0.
 \]
 以上より、\eqref{eq:limI}を得る。\qedhere
\end{proof}

\begin{lem} \label{lem:t_epsilon}
$\sup_{t > 0} I_\lambda (tv_\epsilon)$を達成する$t > 0$が存在する。
\end{lem}

\begin{proof}
補題~\ref{lem:mountain_dec}より、$t \to \infty$のとき、
$I_\lambda(t v_\epsilon) \to -\infty$となる。
したがって、ある$K>0$が存在し、$K < t$においては、
$I_\lambda(t v_\epsilon) < 0$となる。
また、$I_\lambda(0) = 0$である。
ゆえに、$\sup_{t > 0} I_\lambda (tv_\epsilon) = 
\sup_{t \in [0, K]} I_\lambda (tv_\epsilon)$となる。
$I_\lambda(tv_\epsilon)$は$t$についての連続関数であるから、
$\sup_{t > 0} I_\lambda (tv_\epsilon)$を達成する$t > 0$が存在する。\qedhere
\end{proof}

\begin{nota}
 補題~\ref{lem:t_epsilon}
 の$t$を$t_\epsilon$とかく。$t_\epsilon > 0$であり、
 次式が成立する。
 \begin{equation}
  I_\lambda(t_\epsilon v_\epsilon) = \sup_{t > 0} I_\lambda(t
   v_\epsilon).
   \label{eq:t_epsilon}
 \end{equation}
\end{nota}

\begin{lem} \label{lem:intHprime}
 $\epsilon_0 > 0$と$C > 0$が存在し、$0 < \epsilon < \epsilon_0$に対し、
 \begin{equation}
  \int_\Omega H^\prime(t_\epsilon v_\epsilon, \underline{u}_\lambda)
   dx \geq C\epsilon^{(N-2)/4} \label{eq:int_Hprime}
 \end{equation}
 が成立する。
\end{lem}

\begin{proof}
まず、次式を背理法を用いて証明する。
\begin{equation}
 \liminf_{\epsilon \searrow 0} t_\epsilon > 0. \label{eq:liminf_tepsilon}
\end{equation}
\eqref{eq:liminf_tepsilon}を否定し、
$\liminf_{\epsilon \searrow 0} t_\epsilon = 0$であることを仮定する。
$s, t \geq 0$に対し、$G(s, t) \geq 0$である。そこで、
\begin{equation}
 I_\lambda(t_\epsilon v_\epsilon) = \frac{1}{2} t_\epsilon^2 \left\|
                                                              v_\epsilon
                                                             \right\|^2
 - \int_\Omega G(t_\epsilon v_\epsilon, \underline{u}_\lambda) dx \leq
 \frac{1}{2} t_\epsilon^2 \left\| v_\epsilon \right\|^2
\end{equation}
において、$\epsilon \searrow 0$における下極限をとると、
\begin{equation}
 \liminf_{\epsilon \searrow 0} I_\lambda (t_\epsilon v_\epsilon) \leq
  0 \label{eq:liminf_cont_1}
\end{equation}
がしたがう。一方で、\eqref{eq:t_epsilon}より
$I_\lambda(t_\epsilon v_\epsilon) \geq I(\tau_\epsilon v_\epsilon)$で
あるから、補題~\ref{lem:tauepsilon}より、
\begin{equation}
  \liminf_{\epsilon \searrow 0} I_\lambda(t_\epsilon v_\epsilon) \geq
 \lim_{\epsilon \searrow 0} I_\lambda (\tau_\epsilon v_\epsilon) =
 \frac{1}{NM_1^{(N-2)/2}} S^{N/2} > 0. \label{eq:lininf_cont_2}
\end{equation}
\eqref{eq:liminf_cont_1}と\eqref{eq:lininf_cont_2}は同時に
成立しない。
よって、背理法により、\eqref{eq:liminf_tepsilon}がしたがう。 

さて、\eqref{eq:liminf_tepsilon}と\eqref{eq:buepsilon}より、
$\epsilon_0 > 0$、$C > 0$が存在し、$0 <\epsilon < \epsilon_0$のとき、
$\lvert x \rvert < r_0$に対し、
\begin{equation}
 t_\epsilon v_\epsilon (x) = t_\epsilon \frac{\eta(x)}{(\epsilon +
 \lvert x \rvert^2)^{(N-2)/2} \left\| b^{1/(p+1)} u_\epsilon
 \right\|_{L^{p+1}(\Omega)}} \geq \frac{C
 \epsilon^{(N-2)/4}}{(\epsilon + \lvert x \rvert^2)^{(N-2)/2}} 
 \label{eq:teve_p}
\end{equation}
が成立する。必要ならば$\epsilon_0 > 0$を小さくとりなおし、
$\sqrt{\epsilon_0} < r_0$が成立するとして良い。すると、
$\lvert x \lvert < \sqrt{\epsilon}$に対し、
\begin{equation}
 t_\epsilon v_\epsilon (x) \geq C_0 \epsilon^{-(N-2)/2} \label{eq:teve}
\end{equation}
となる。$C_0 > 0$は$\epsilon$によらない。
この$C_0$について、
\begin{equation}
 t_0 = C_0 \epsilon_0^{-(N-2)/2} \label{eq:t_0_epsilon}
\end{equation}
と定める。
$\underline{u}_\lambda > 0 ~\tin \Omega$であるから、
$\{ \lvert x \rvert \leq \sqrt{\epsilon_0} \}$における
$\underline{u}_\lambda$の最小値より小さい正の数$s_0$が存在する。
すなわち、$\lvert x \rvert < \sqrt{\epsilon_0}$に対し、
\begin{equation}
 \underline{u}_\lambda (x) > s_0 \label{eq:uus_0}
\end{equation}
となる。

ここで、$x \in \Omega$、$t \geq t_0$、$s \geq s_0$に対し、
\begin{equation}
 H^\prime (t, s, x) \geq C b(x) t^p \label{eq:Hprime_lb}
\end{equation}
を成り立たせる$t, s, x$によらない定数$C>0$が存在することを示す。
$s, t \geq 0$に対し、
\[
 H^\prime(t, s, x) = b(x) \left( \frac{1}{p+1}(t+s)^{p+1} -
 \frac{1}{p+1} t^{p+1} - \frac{1}{p+1} s^{p+1} - s^p t \right)
\]
である。そこで$s$についての偏導関数は、
\[
 H^\prime_s (t, s, x) = b(x) \left( (t+s)^p - s^p - ps^{p-1}t \right)
\]
である。右辺はテイラーの定理より、$0 < \theta < 1$を用いて
$p(p-1)(s + \theta t)^{p-2}t^2/2$と表される。これは非負であるから、
$H^\prime_s(t, s, x) \geq 0$である。すなわち、$H^\prime$は
$s$についての増加関数である。
したがって、$s \geq s_0$、$t \geq 0$に対し、
\begin{equation}
 H^\prime(t, s, x) \geq H^\prime(t, s_0, x) \label{eq:Hprime_lb_pf1}
\end{equation}
である。また、$s \geq 0$、$t \geq 0$に対し、
\begin{equation}
 H^\prime(t, s, x) \geq H^\prime(t, 0, x) = 0 \label{eq:Hprime_lb_pf3}
\end{equation}
もわかる。ここでテイラーの定理より、
\[
 \frac{1}{p+1} (t + s_0)^{p+1} - \frac{1}{p+1} t^{p+1} = (t + \theta
 s_0)^p s_0
\]
 をみたす$0 < \theta < 1$が存在する。ゆえに$t \geq t_0$に対し、
 以下がしたがう。
 \begin{align*}
  H^\prime(t, s_0, x) &\geq b(x) \left( (t+\theta s_0)^p s_0 -
  \frac{1}{p+1} s^{p+1} - s^p t\right) \\
  & \geq b(x) \left( t^p s_0 - \frac{1}{p+1} s^{p+1} - s^p t \right) \\
  & = t^p b(x) \left( s_0 - \frac{1}{p+1} \frac{1}{t^p} - s^p
  \frac{1}{t^{p-1}}  \right) \\
  & \geq t^p b(x) \left( s_0 - \frac{1}{p+1} \frac{1}{t_0^p} - s^p
  \frac{1}{t_0^{p-1}}  \right).
 \end{align*}
 ここで最右辺の括弧の中が正となるよう、
 必要ならば$\epsilon_0 > 0$を小さくとりなおす。
 $s_0$と$C_0$は$\epsilon_0$によっているが、
 $\lvert x \rvert < \sqrt{\epsilon_0}$に対して
 \eqref{eq:teve}および\eqref{eq:uus_0}を成り立たせるために
 $s_0$と$C_0$は変更する必要がないことに注意されたい。
 \eqref{eq:t_0_epsilon}により、
 最右辺の括弧の中が正となるよう、$t_0$を大きくすることができる。
 以上により、$t \geq t_0$に対し、
 \begin{equation}
  H^\prime (t, s_0, x) \geq C b(x) t^p \label{eq:Hprime_lb_pf2}
 \end{equation}
 が成立する。
 \eqref{eq:Hprime_lb_pf1}と\eqref{eq:Hprime_lb_pf2}より、
 $t \geq t_0$、$s \geq s_0$に対し、
 \eqref{eq:Hprime_lb}がしたがう。

 \eqref{eq:Hprime_lb_pf3}、\eqref{eq:Hprime_lb}、\eqref{eq:teve_p}を
 順に使うと、
 $0 < \epsilon < \epsilon_0$に対し、以下が成立する。
 \begin{align*}
  \int_\Omega H^\prime(t_\epsilon v_\epsilon, \underline{u}_\lambda )
  dx & \geq \int_{\{ \lvert x \rvert \leq \sqrt{\epsilon} \}} 
  H^\prime(t_\epsilon v_\epsilon, \underline{u}_\lambda ) dx \\
  & \geq C \int_{\{ \lvert x \rvert \leq \sqrt{\epsilon} \}} 
  b(x) (t_\epsilon v_\epsilon)^p dx \\
  & \geq C \int_{\{ \lvert x \rvert \leq \sqrt{\epsilon} \}} 
  (M_1 + M_2 (\sqrt{\epsilon})^q) \left(
  \frac{\epsilon^{(N-2)/4}}{(\epsilon + \lvert x \rvert^2)^{(N-2)/2}}
  \right)^p dx \\
  & \geq C \int_{\{ \lvert x \rvert \leq \sqrt{\epsilon} \}} 
  \left(
  \frac{\epsilon^{(N-2)/4}}{(\epsilon + \lvert x \rvert^2)^{(N-2)/2}}
  \right)^p dx \\
  & = C \epsilon^{(N-2)/4} \int_{ \{ \lvert y \rvert \leq 1 \}}
  \frac{1}{(1 + \lvert x \rvert^2)^{(N+2)/2}} dx \\
  & \geq C \epsilon^{(N-2)/4}.
 \end{align*}
 $C>0$は$\epsilon$によらない。所望の\eqref{eq:int_Hprime}が得られた。\qedhere
\end{proof}

\begin{proof}[命題~\ref{prop:second_2}]
 \eqref{eq:t_epsilon}より、以下が成立する。
 \begin{align*}
  \sup_{t > 0} I_\lambda (t v_\epsilon) & = I_\lambda (t_\epsilon
  v_\epsilon) \\ 
  & = \frac{1}{2} t_\epsilon^2 \left\| v_\epsilon \right\|^2 -
  \frac{1}{p+1} t_\epsilon^p - \int_\Omega H^\prime(t_\epsilon
  v_\epsilon, \underline{u}_\lambda) dx + \int_\Omega a v_\epsilon^2
  dx \\
  & \leq \sup_{t > 0} \left( \frac{1}{2} t^2 \left\| v_\epsilon
  \right\|^2 - \frac{1}{p+1} t^{p+1} \right)
  - \int_\Omega H^\prime(t_\epsilon
  v_\epsilon, \underline{u}_\lambda) dx + \int_\Omega a v_\epsilon^2
  dx \\
  & = \frac{1}{N} \left( \left\| v_\epsilon \right\|^2 \right)^{N/2}
  - \int_\Omega H^\prime(t_\epsilon
  v_\epsilon, \underline{u}_\lambda) dx + \int_\Omega a v_\epsilon^2
  dx. \\
 \end{align*}
 ここで、最後の変形では、$t > 0$の関数
 \[
   \alpha(t) = \frac{1}{2} t^2 \left\| v_\epsilon \right\|^2 -
 \frac{1}{p+1} t^{p+1}
 \]
 が、$t = \left\| v_\epsilon \right\|^{2/(p-1)}$に
 おいて最大値をとることに注意した。\eqref{eq:vepsilon}、
 \eqref{eq:av_epsilon}、補題~\ref{lem:tauepsilon}、補
 題~\ref{lem:intHprime}により、$\epsilon_0 > 0$、$C, C^\prime > 0$が
 存在し、$0 < \epsilon < \epsilon_0$に対し、
 \begin{equation}
  \sup_{t > 0} I_\lambda (tv_\epsilon) \leq \frac{1}{NM_1^{(N-2)/2}}
   S^{N/2} + \left( C \epsilon^{(N-2)/2} - C^\prime \epsilon^{(N-2)/4}
             + m_1 I_1^\prime + m_2 I_2^\prime \right) \label{eq:sup_least}
 \end{equation}
 が成立する。ここで$I_1^\prime$、$I_2^\prime$は、\eqref{eq:av_epsilon}
 のものである。以下の条件を考える。
 \begin{equation}
  \text{\eqref{eq:sup_least}の右辺の括弧の中が負となる
   $\epsilon > 0$が存在する。} \label{eq:condition}
 \end{equation}
 \eqref{eq:condition}が成立するならば、
 その$\epsilon$を用いて$v_0 = v_\epsilon$とすると、$v_0 \geq 0 ~\tin
 \Omega$、$v_0 \not \equiv 0$、および
 \eqref{eq:ineq_S}が成立する。すなわち、
 \eqref{eq:condition}は、命題~\ref{prop:second_2}の帰結の
 十分条件である。
 以下、$m_1, m_2$の正負で場合分けして検証する。
 すべての$N \geq 3$、$q^\prime > 0$に対し、$\epsilon \searrow 0$のとき
 $I_1^\prime \gg I_2^\prime$であることに注意されたい。

 \begin{enumerate}[(i)]
  \item \ulinej{{$m_1 < 0$}のとき}:$\epsilon \searrow 0$のとき
        $\epsilon^{(N-2)/2} \ll \epsilon^{(N-2)/4}$であるから、
        すべての$N \geq 3$について、\eqref{eq:condition}はみたされる。
  \item \ulinej{{$m_1 > 0$}のとき}:$\epsilon \searrow 0$のとき
        $I_1^\prime \ll \epsilon^{(N-2)/4}$となれば、
        \eqref{eq:condition}はみたされる。
        \eqref{eq:av_epsilon}より、
        $N = 3, 4, 5$であれば\eqref{eq:condition}はみたされる。
  \item \ulinej{{$m_1 = 0$、$m_2 < 0$}のとき}:(i)と同様に、
        すべての$N \geq 3$について、\eqref{eq:condition}はみたされる。
  \item \ulinej{{$m_1 = 0$、$m_2 > 0$}のとき}:$\epsilon \searrow 0$のとき
        $I_2^\prime \ll \epsilon^{(N-2)/4}$となれば、
        \eqref{eq:condition}はみたされる。
        \eqref{eq:av_epsilon}より、
        $N \leq q^\prime + 4$のときは、この式は成立している。
        $N > q^\prime + 4$のとき、この式が成立する条件は、
        \[
         1 + \frac{q^\prime}{2} > \frac{N-2}{4}
        \]
        である。これを変形して、$N < 2q^\prime + 6$を得る。
        以上により、$3 \leq N < 2q^\prime + 6$のとき、
        \eqref{eq:condition}はみたされる。 \qedhere
 \end{enumerate}
\end{proof}

\begin{proof}[定理~\ref{thm:second_solution}]
 命題~\ref{prop:second_1}と命題~\ref{prop:second_2}より成立する。 \qedhere
\end{proof}

% Local Variables:
% mode: yatex
% TeX-master: "main.tex"
% End: