%#!platex main.tex
 \section{extremal value 付近での second solution の存在} \label{sec:exist}
 
 本節では,定理~\ref{thm:second_solution_nonex}.1を証明する.
 
 $0 < \lambda \leq \bar{\lambda}$に対し,$\phi_\lambda$を
 \eqref{eq:lin_prob_min_sol}の第$1$固有関数とする.
 補題~\ref{lem:lin_p}.1より,$\Omega$上正値であるものが
 存在するので,$\phi_\lambda > 0 ~\tin \Omega$と仮定して良い.また,
 $\left\| \phi_\lambda \right\|_{L^{p+1}(\Omega)} = 1$と正規化されたも
 のとする.
 次の命題を示せば,命題~\ref{prop:second_1}と合わせて,証明が完了する.
 
\begin{prop} \label{prop:exist}
 定理~\ref{thm:second_solution_nonex}.1の仮定のもとで,
 $0 < \lambda^* < \bar{\lambda}$が存在し,
 $\lambda^* \leq \lambda < \bar{\lambda}$に対し,
 \begin{equation}
  \sup_{t > 0} I_\lambda(t \phi_\lambda) < \frac{1}{N \left\| b
                                                                                            \right\|_{L^\infty(\Omega)}^{(N-2)/2}}
  S^{N/2} \label{eq:exist_S}
 \end{equation}
 が成立する.
\end{prop}

\begin{proof}
 $v \in H^1_0(\Omega)$とする.
 \begin{equation}
  I_\lambda (v) = \frac{1}{2} \left( \int_\Omega \left( \lvert Dv
                                                  \rvert^2 + a v^2
                                                 \right) dx - p
  \int_\Omega b \underline{u}_\lambda^{p-1} v^2 dx \right) -
  \frac{1}{p+1} \int_\Omega b v^{p+1} dx + \frac{p}{2} \int_\Omega b
  \underline{u}_\lambda ^{p-1} v^2 dx - \int_\Omega H^\prime (v,
  \underline{u}_\lambda) dx \label{eq:I_lambda_exist}
 \end{equation}
 である.ここで,$t, s \geq 0$,$x \in \Omega$とし,
 \[
 A = \frac{p}{2} b(x) s^{p-1} t^2 - H^\prime(t, s) = b(x) \left(
 \frac{p}{2} s^{p-1} t^2 - \left(\frac{1}{p+1} (t+s)^{p+1} -
 \frac{1}{p+1}t^p - \frac{1}{p+1} s^{p+1} - s^p t \right) \right)
 \]
 を考察する.ここで$0 < \epsilon < 1$を,
 後述の\eqref{eq:epsilon_gutaiteki}をみたすよう定める.
 補題~\ref{lem:ineq_naito_sato}より,
 \[
 A \leq b \left( \left( \frac{1}{p+1} - \frac{p}{2} \epsilon \right)
 t^{p+1} + \frac{p}{2}\epsilon s^{p-1} t^2 \right)
 \]
 が成立する.$p > 1$であるから,$p/2 > 1/(p+1)$である.ゆえに,
 \[
 A \leq b \left( \frac{1 - \epsilon}{p+1} t^{p+1} +
 \frac{p\epsilon}{2} s^{p-1} t^2 \right)
 \]
 が得られる.したがって,
 \[
 \frac{p}{2} \int_\Omega b \underline{u}_\lambda^{p-1} v^2 dx -
 \int_\Omega H^\prime (v, \underline{u}_\lambda) dx \leq \int_\Omega b
 \left( \frac{1-\epsilon}{p+1} v^{p+1} + \frac{p \epsilon}{2}
 \underline{u}_\lambda ^{p-1} v^2 \right) dx
 \]
 である.\eqref{eq:I_lambda_exist}より,$I_\lambda$は
 \[
 I_\lambda(v) \leq \frac{1}{2} \left( \int_\Omega \left( \lvert Dv
 \rvert^{2} + a v^2\right)dx - (1 -\epsilon)p \int_\Omega b
 \underline{u}_\lambda^{p-1} v^2 dx \right) - \frac{\epsilon}{p+1}
 \int_\Omega bv^{p+1} dx
\]
 と評価される.
 ここで右辺の括弧の中を$P_\epsilon(v)$とおく.
 補題~\ref{lem:lin_p}.2,補題~\ref{lem:lin_1}より,
 \[
  P_\epsilon(v) \geq \left(\mu_1(\lambda) - (1 - \epsilon) \right) p
 \int_\Omega b
 \underline{u}_\lambda^{p-1} v^2 dx \geq \epsilon p 
 \int_\Omega b
 \underline{u}_\lambda^{p-1} v^2 dx \geq 0
 \]
 とわかる.
 また,
 \begin{equation}
  K = \int_\Omega b \phi_\lambda^{p+1} dx \label{eq:intbphi}
 \end{equation}
 とおく.$b \geq 0 ~\tin \Omega$,$b \not\equiv 0$,$\phi_\lambda > 0
 ~\tin \Omega$であるから,$K > 0$である.
 $t > 0$に対し,
 \[
 I_\lambda (t \phi_\lambda) \leq \frac{t^2}{2} P_\epsilon
 (\phi_\lambda) - \frac{\epsilon K t^{p+1}}{p+1}
 \]
 である.ゆえに,
\begin{equation}
 \sup_{t > 0} I_\lambda (t \phi_\lambda) \leq \sup_{t > 0} \left(
                                                            \frac{t^2}{2} P_\epsilon 
 (\phi_\lambda) - \frac{\epsilon K t^{p+1}}{p+1} \right) =  
 \frac{P_\epsilon(\phi_\lambda)^{N/2}}{\left( \epsilon K
                                         \right)^{(N-2)/2}}
 \label{eq:supIlambdatphi1} 
\end{equation}
と抑えられる.最後の変形は,中辺の括弧の中を$t$の関数と見たとき,
$t = ( P_\epsilon(\phi_\lambda) / \epsilon K )^{1/(p-1)}$において
最大値をとることに注意した.ここで,補題~\ref{lem:lin_p}.2より,
\[
 \int_\Omega \left( \lvert D\phi_\lambda \rvert^2 + a \phi_\lambda^2
 \right) dx = \mu_1(\lambda) p \int_\Omega b
 \underline{u}_\lambda^{p-1} \phi_\lambda^2 dx
\]
が成立することを用いれば,$P_\epsilon(\phi_\lambda)$は
\[
 P_\epsilon(\phi_\lambda) = \left( \mu_1(\lambda) - 1 + \epsilon
 \right) p \int_\Omega b \underline{u}_\lambda^{p-1} \phi_\lambda^2 dx
\]
と変形される.ヘルダーの不等式より,次式が成立する.
\[
 \int_\Omega b \underline{u}_\lambda^{p-1} \phi_\lambda^2 dx \leq
 \left\| b \right\|_{L^\infty(\Omega)} \left\| \underline{u}_\lambda
 \right\|_{L^{p+1}(\Omega)}^{p-1} \left\| \phi_\lambda
 \right\|_{L^{p+1}(\Omega)}^2 = \left\| b \right\|_{L^\infty(\Omega)}
 \left\| \underline{u}_\lambda 
 \right\|_{L^{p+1}(\Omega)}^{p-1}.
\]
命題~\ref{prop:ext_exi}.2より,$\left\| \underline{u}_\lambda 
 \right\|_{L^{p+1}(\Omega)} \leq \left\| \underline{u}_{\bar{\lambda}} 
 \right\|_{L^{p+1}(\Omega)}$が成立する.ゆえに,
次式が成り立つ.
\[
 P_\epsilon(\phi_\lambda) \leq (\mu_1(\lambda) - 1 + \epsilon) \left\|
 b \right\|_{L^{p+1}(\Omega)} p \left\| \underline{u}_{\bar{\lambda}}
 \right\|_{L^{p+1}(\Omega)}^{p-1}.
\]
補題~\ref{lem:mu_1_bar_lambda}より,$0 < \lambda^{*} < \bar{\lambda}$
が存在し,$\lambda^{*} \leq \lambda < \bar{\lambda}$ならば,
$\mu_1(\lambda) - 1 < \epsilon$となる.ゆえに,
\[
 P_\epsilon(\phi_\lambda) \leq 2\epsilon p \left\|
 b \right\|_{L^{p+1}(\Omega)} \left\| \underline{u}_{\bar{\lambda}}
 \right\|_{L^{p+1}(\Omega)}^{p-1}
\]
が従う.\eqref{eq:supIlambdatphi1}より,
\[
 \sup_{t > 0} I_\lambda (t \phi_\lambda) \leq \frac{\left( 2p \left\|
 b \right\|_{L^{p+1}(\Omega)} \left\| \underline{u}_{\bar{\lambda}}
 \right\|_{L^{p+1}(\Omega)}^{p-1} \right)^{N/2} \epsilon}{K^{(N-2)/2}}
\]
と評価される.ゆえに,
\begin{equation}
 \epsilon < \frac{K^{(N-2)/2} S^{N/2}}{N
 \left( 2p \left\|
  b \right\|_{L^{p+1}(\Omega)} \left\| \underline{u}_{\bar{\lambda}}
 \right\|_{L^{p+1}(\Omega)}^{p-1} \right)^{N/2}
  \left\| b
  \right\|_{L^\infty(\Omega)} ^{(N-2)/2}}
 \label{eq:epsilon_gutaiteki}
\end{equation}
としておけば,\eqref{eq:exist_S}が成立する. \qedhere
\end{proof}

\begin{proof}[定理~\ref{thm:second_solution_nonex}.1]
 $\phi_\lambda > 0 ~\tin \Omega$であり,
 \eqref{eq:intbphi}の$K$は$K > 0$をみた
 すことは,命題~\ref{prop:exist}の証明中確かめた.加えて
 命題~\ref{prop:exist}の主張より,
 $v_0 = \phi_\lambda$とすると,命題~\ref{prop:second_1}の仮定をみたす.
 したがって,\ref{eq:prob_sec}は弱解$v$を持つ.
 これと補題~\ref{lem:rel_heart_spade}.2より,\ref{eq:prob_main}は
 minimal solution 以外の弱解を持つ. \qedhere
\end{proof}


% Local Variables:
% mode: yatex
% TeX-master: "main.tex"
% End: