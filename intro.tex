%#!platex main.tex
\section{概要}

$N$を$3$以上の自然数とする。$\Omega \subset \R^N$を有界領域とする。
$p = (N+2)/(N-2)$とする。$f \in H^{-1}(\Omega)$は、$f \geq 0$、
$f \not \equiv 0$をみたすとする。
$a, b \in L^\infty(\Omega)$とする。
$\kappa_1$を$-\Delta$の$\Omega$におけるディリクレ条件下での
第$1$固有値とする。$\kappa > - \kappa_1$があって、$a \geq \kappa$
となると仮定する。また、$b \geq 0$、$b \not \equiv 0$と仮定する。
$\lambda \geq 0$をパラメータとする。以下の方程式を考察する。
\begin{align}
 \left\{
 \begin{aligned}
  -\Delta u + a u &= b u^p + \lambda f  & &\text{in~} \Omega,  \\
  u &> 0 & &\text{in~} \Omega, \\
  u &= 0 & &\text{on~} \partial\Omega
 \end{aligned}
 \right. \tag*{$(\star)_\lambda$} \label{eq:prob_main}
\end{align}

\begin{thm} \label{thm:minimal_solution}
 \ref{eq:prob_main} には minimal solution が存在する。
\end{thm}

\begin{thm} \label{thm:extremal_solution}
 \ref{eq:prob_main} には extremal solution が存在する。
 また、\ref{eq:prob_main} の 
 extremal solution は、$\lambda = \bar{\lambda}$における
 \ref{eq:prob_main} の
 minimal solution に限る。
\end{thm}

\subsection{記号}

ルベーグ空間を$L^q(\Omega)$ ($1 \leq q \leq \infty$)と表記する。
ソボレフ空間$W^{1, 2}(\Omega)$を$H^1(\Omega)$と表記する。
トレースの意味で$u |_{\partial \Omega} = 0$が成立
する$u \in H^1(\Omega)$全体を$H_0^1(\Omega)$と表記する。
ヘルダー空間を$C^{k + \alpha}(\Omega)$ ($k \in \N$、$0 < \alpha < 1$)
と表記する。

ノルム空間$X$のノルムを$\dnorm_X$と表記する。
ノルム空間$X$の双対空間を$X^*$と表記する。
$H_0^1(\Omega)^*$を$H^{-1}(\Omega)$と表記する。
$w \in H_0^1(\Omega)$に対し、
\[
 \left\| w \right\|_\kappa = \left(\int_\Omega \left( \lvert Dw \rvert^2 +
 \kappa \lvert w \rvert^2 \right) dx\right)^{1/2}
\]
と定める。$\kappa > -\kappa_1$、$\Omega$が有界領域で
あることにより、ポアンカレの不等式から
$\dnorm_\kappa$は$\dnorm_{H_0^1(\Omega)}$
と同値なノルムであることが従う。また、
$w \in H_0^1(\Omega)$に対し、
\[
 \left\| w \right\| = \left(\int_\Omega \lvert Dw \rvert^2 dx\right)^{1/2}
\]
と定める。やはりポアンカレの不等式から
$\dnorm$は$\dnorm_{H_0^1(\Omega)}$
と同値なノルムであることが従う。