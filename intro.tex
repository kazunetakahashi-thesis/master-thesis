%#!platex main.tex
\section{概要}

$N$を$3$以上の自然数とする。$\Omega \subset \R^N$を有界領域とする。
$p = (N+2)/(N-2)$とする。$f \in H^{-1}(\Omega)$は、$f \geq 0$、
$f \not \equiv 0$をみたすとする。
$a, b \in L^\infty(\Omega)$とする。
$\kappa_1$を$-\Delta$の$\Omega$におけるディリクレ条件下での
第$1$固有値とする。$\kappa > - \kappa_1$が存在して、$a \geq \kappa$
となると仮定する。また、$b \geq 0$、$b \not \equiv 0$と仮定する。
$\lambda \geq 0$をパラメータとする。以下の方程式を考察する。
\begin{align}
 \left\{
 \begin{aligned}
  -\Delta u + a u &= b u^p + \lambda f  & &\text{in~} \Omega,  \\
  u &> 0 & &\text{in~} \Omega, \\
  u &= 0 & &\text{on~} \partial\Omega
 \end{aligned}
 \right. \tag*{$(\spadesuit)_\lambda$} \label{eq:prob_main}
\end{align}

\begin{thm} \label{thm:minimal_solution}
 \ref{eq:prob_main} には minimal solution が存在する。
\end{thm}

\begin{thm} \label{thm:extremal_solution}
 \ref{eq:prob_main} には extremal solution が存在する。
 とくに、$\lambda = \bar{\lambda}$における
 \ref{eq:prob_main} の
 minimal solution が存在する。
 また、$b > 0 ~\tin \Omega$ならば、\ref{eq:prob_main} の 
 extremal solution は、$\lambda = \bar{\lambda}$における
 \ref{eq:prob_main} の
 minimal solution に限る。
\end{thm}

\begin{thm} \label{thm:second_solution}
 $0 < \lambda < \bar{\lambda}$とする。$b$は$\Omega$上の
 ある点$p$で最大値$M_1 = \left\| b \right\|_{L^\infty}(\Omega) > 0$を
 達成するものと仮定する。$r_0 > 0$が存在し、
 $\{ \lvert x - p \rvert < 2r_0 \} \subset \Omega$、かつ、
 $\{ \lvert x - p \rvert < r_0 \}$上
 \begin{align*}
  b(x) &= M_1 - M_2 \lvert x-p \rvert^q,  \\
  a(x) &= m_1 + m_2 \lvert x-p \rvert^{q^\prime}
 \end{align*}
 であると仮定する。ここで$q, q^\prime > 0$、
 $M_2 > 0$、$m_1 > \kappa$、$m_2 \neq 0$は
 定数である。さらに、以下の(i) -- (iv)の
 いずれかの成立を仮定する。
 \begin{enumerate}[(i)]
  \item $m_1 < 0$、かつ、$N \geq 3$。
  \item $m_1 > 0$、かつ、$N = 3, 4, 5$。
  \item $m_1 = 0$、かつ、$m_2 < 0$、かつ、$N \geq 3$。
  \item $m_1 = 0$、かつ、$m_2 > 0$、かつ、$3 \leq N < 6 + 2q^\prime$。
 \end{enumerate}
 このとき、\ref{eq:prob_main}は、minimal solution
 $\underline{u}_\lambda$
 以外の弱解$\bar{u}_\lambda \in H_0^1(\Omega)$をもつ。
\end{thm}

\subsection{記号}

ルベーグ空間を$L^q(\Omega)$ ($1 \leq q \leq \infty$)と表記する。
ソボレフ空間$W^{1, 2}(\Omega)$を$H^1(\Omega)$と表記する。
トレースの意味で$u |_{\partial \Omega} = 0$が成立
する$u \in H^1(\Omega)$全体を$H_0^1(\Omega)$と表記する。
ヘルダー空間を$C^{k + \alpha}(\Omega)$ ($k \in \N$、$0 < \alpha < 1$)
と表記する。
コンパクト台を持つ$\Omega$上の$C^\infty$級関数全体を
$C^\infty_c (\Omega)$と表記する。

ノルム空間$X$のノルムを$\dnorm_X$と表記する。
ノルム空間$X$の双対空間を$X^*$と表記する。
$H_0^1(\Omega)^*$を$H^{-1}(\Omega)$と表記する。
$f \in H^{-1}$の$u \in H_0^1(\Omega)$への作用を$\langle f, u \rangle$
と表記する。
$H_0^1(\Omega)$上のノルム$\dnorm_{\kappa}$を、$w \in H_0^1(\Omega)$に対し、
\[
 \left\| w \right\|_\kappa = \left(\int_\Omega \left( \lvert Dw \rvert^2 +
 \kappa w ^2 \right) dx\right)^{1/2}
\]
と定める。$\kappa > -\kappa_1$、$\Omega$が有界領域で
あることにより、ポアンカレの不等式から
$\dnorm_\kappa$は$\dnorm_{H_0^1(\Omega)}$
と同値なノルムである。また、
$H_0^1(\Omega)$上のノルム$\dnorm$を、$w \in H_0^1(\Omega)$に対し、
\[
 \left\| w \right\| = \left(\int_\Omega \lvert Dw \rvert^2 dx\right)^{1/2}
\]
と定める。やはりポアンカレの不等式から
$\dnorm$は$\dnorm_{H_0^1(\Omega)}$
と同値なノルムであることがしたがう。