\section{概要}

$N$を$3$以上の自然数とする。$\Omega \subset \R^N$を有界領域とする。
$p = (N+2)/(N-2)$とする。$f \in H^{-1}(\Omega)$は、$f \geq 0$、
$f \not \equiv 0$をみたすとする。
$a, b \in L^\infty(\Omega)$とする。
$\kappa_1$を$-\Delta$の$\Omega$におけるディリクレ条件下での
第$1$固有値とする。$\kappa > - \kappa_1$があって、$a \geq \kappa$
となると仮定する。また、$b \geq 0$、$b \not \equiv 0$と仮定する。
$\lambda \geq 0$をパラメータとする。以下の方程式を考察する。
\begin{align}
 \left\{
 \begin{aligned}
  -\Delta u + a u &= b u^p + \lambda f  & &\text{in~} \Omega,  \\
  u &> 0 & &\text{in~} \Omega, \\
  u &= 0 & &\text{on~} \partial\Omega
 \end{aligned}
 \right. \tag*{$(\star)_\lambda$} \label{eq:prob_main}
\end{align}

$w \in H_0^1(\Omega)$に対し、
\[
 \left\| w \right\|_\kappa = \int_\Omega \left( \lvert Dw \rvert^2 +
 \kappa \lvert w \rvert^2 \right) dx
\]
と定める。$\kappa > -\kappa_1$、$\Omega$が有界領域で
あることにより、ポアンカレの不等式から
$\left\| \cdot \right\|_\kappa$は$\left\| \cdot
\right\|_{H_0^1(\Omega)}$
と同値なノルムであることが従う。

$w \in H_0^1(\Omega)$に対し、
\[
 \left\| w \right\| = \int_\Omega \lvert Dw \rvert^2 dx
\]
と定める。やはりポアンカレの不等式から
$\left\| \cdot \right\|$は$\left\| \cdot
\right\|_{H_0^1(\Omega)}$
と同値なノルムであることが従う。