%#!platex main.tex
\section{minimal solutionの存在と性質}

本節では、\ref{eq:prob_main}の解のうち、
minimal solution について取り扱う。
まずは minimal solution を定義する。

\begin{nota}
 $\lambda > 0$に対し、
 \[
 S_\lambda = \{ u \in H_0^1(\Omega) \mid u \text{は
 \ref{eq:prob_main} の弱解である}\}
 \]
 と定める。
\end{nota}
\begin{defn}
 $\underline{u}_\lambda \in S_\lambda$が{\bf minimal solution} で
 あるとは、任意の$u \in S_\lambda$に対し、
 $
  \underline{u}_\lambda \leq u ~\tin \Omega
 $
 が成立することをいう。
\end{defn}
\begin{nota}
 \ref{eq:prob_main}の minimal solution を $\underline{u}_\lambda$ と表記する。
\end{nota}

\subsection{$H_0^1(\Omega)$の原点付近における様子}

minimal solution を調べる第一歩として、
$\lambda > 0$が十分小さいときに、\ref{eq:prob_main}が
弱解を持つことを、陰関数定理を用いて示す。

\begin{lem} \label{lem:imp}
 \begin{enumerate}[1.] \sage
  \item $\lambda_0 > 0$と$H_0^1(\Omega)$の原点の近傍$U$
        が存在して、$0 < \lambda \leq \lambda_0$に対し、
        \ref{eq:prob_main}は$U$内の唯一の弱解
        $u_\lambda$をもつ。また、次が成立する。
        \[
        \left\| u_\lambda
        \right\|_{H^1_0(\Omega)} \to 0 \ \ (\lambda \searrow 0).
        \]
  \item さらに、$f \in C^\alpha(\bar{\Omega})$を仮定する。
        このとき、1.~の弱解$u_\lambda$は、$u_\lambda \in
        C^{2+\alpha}(\Omega)$を
        みたし、次が成立する。
        \[
        \left\| u_\lambda
        \right\|_{C^{2+\alpha}(\Omega)} \to 0 \ \ 
        (\lambda \searrow 0).
        \]
 \end{enumerate}
\end{lem}

\begin{proof}
 \begin{enumerate}[1.] \sage
  \item $\Phi \colon [0,\infty) \times H^1_0 (\Omega) \to H^{-1}(\Omega)$を
        \begin{equation}
         \Phi (\lambda, u) = -\Delta u + au - b (u_{+})^p - \lambda f
          \label{eq:def_of_Phi}
        \end{equation}
        とする。$\Phi$の$u$についてのフレッシェ微分は、
        $w \in H^1_0(\Omega)$に対し、
        \begin{equation}
         \Phi_u (\lambda, u) \colon w \mapsto -\Delta w + aw - b
          p(u_+)^{p-1} w.
          \label{eq:Phi_dr}
        \end{equation}
        と書かれる。
        特に、
        \[
         \Phi_u (0, 0) w = -\Delta w + aw.
        \]
        が成立する。
        $a > -\kappa_1$により、$\Phi_u(0,0) \colon
        H^1_0(\Omega) \to H^{-1} (\Omega)$は可逆である。ゆえに、
        陰関数定理より、$\lambda_0 > 0$と
        $H_0^1(\Omega)$の原点の近傍$U$が存在して、
        $0 < \lambda \leq \lambda_0$に対し、
        $\Phi(\lambda, u_\lambda) = 0$をみたす
        $u_\lambda \in U$が
        唯一つ存在し、次をみたす。
        \[\lim_{\lambda \searrow 0} \left\| u_\lambda
        \right\|_{H^1_0(\Omega)} 
        = 0. \]
        つまり、$u_\lambda$は、以下の方程式の弱解である。
        \begin{align}
         \left\{
         \begin{aligned}
          -\Delta u + a u &= b (u_+)^p + \lambda f  & &\tin \Omega,  \\
          u &= 0 & &\ton \partial\Omega.
         \end{aligned}
         \right. \label{eq:prob_lem_ifthm}
        \end{align}
        ここで$b (u_+)^p + \lambda f \geq 0$であり、$a > -\kappa_1$で
        あるから、強最大値原理により、$u_\lambda > 0 ~\tin
        \Omega$が成立する。
        よって、$u_\lambda$は\ref{eq:prob_main}の
        $U$における唯一の弱解である。
  \item $f \in C^\alpha(\bar{\Omega})$のとき、
        $\Phi \colon [0,\infty) \times C^{2+\alpha} (\bar{\Omega})
        \to C^\alpha(\bar{\Omega})$を、\eqref{eq:def_of_Phi}で定義する。
        以下、1.~の証明と同様にすると、
        $u_\lambda \in
        C^{2+\alpha}(\Omega)$と
        $\left\| u_\lambda
        \right\|_{C^{2+\alpha}(\Omega)} \to 0 \ \ (\lambda \searrow
        0)$
        が示される。\qedhere
 \end{enumerate}
\end{proof}

以下では基本的に、1.~の結果を使用し、弱解の枠組みで議論する。
2.~の結果は、\S~\ref{sec:sym}で使用する。

\subsection{優解との関係}

続いて、ある
$\lambda = \hat{\lambda}$で\ref{eq:prob_main}が
優解をもつときに、$0 < \lambda \leq \hat{\lambda}$で
minimal solution が存在することを示す。

\begin{lem} \label{lem:minimal_itt}
 $\hat{\lambda} > 0$とする。
 以下をみたす$\hat{u} \in H_0^1(\Omega)$が存在すると仮定する。
\begin{align}
 \left\{
 \begin{aligned}
  \Delta \hat{u} + a\hat{u} &\geq b \hat{u}^p + \hat{\lambda}f  &
  &\tin \Omega,  \\
  \hat{u} &> 0 & &\tin \Omega
 \end{aligned}
 \right. \label{eq:prob_lem_ifthm}
\end{align}
 このとき、$\lambda \in (0,\hat{\lambda} ]$
 に対し、\ref{eq:prob_main}の minimal solution $\underline{u}_\lambda$
 が存在する。
 また、$\underline{u}_\lambda < \hat{u} ~\tin \Omega$
 が成立する。
\end{lem}

\begin{proof}
 $H_0^1(\Omega)$の点列$\{ u_n \}_{n=0}^\infty$を、次の通りに
 帰納的に定める。$u_0 \equiv 0$とする。$u_n$が
 定まっているときに、線形方程式
  \begin{align}
   \left\{
   \begin{aligned}
    -\Delta u_{n+1} + a u_{n+1} 
    &= b u_n^p + \lambda f  & &\tin \Omega,  \\
    u_{n+1} &= 0 & &\ton \partial\Omega
   \end{aligned}
   \right. \label{eq:u_n+1}
  \end{align}
 の唯一の弱解を$u_{n+1} \in H_0^1(\Omega)$と定める。
 
 \eqref{eq:u_n+1}が唯一の弱解であることを確かめる。
 ソボレフ埋め込みにより、
 $
  u_n \in H_0^1(\Omega) \subset L^{p+1}(\Omega)
 $
 だから、
 $
  u_n^p \subset L^{(p+1)/p}(\Omega)
 = L^{2N/(N+2)}(\Omega) \subset H^{-1}(\Omega)
 $
 である。$b \in L^\infty(\Omega)$、$f \in H^{-1}(\Omega)$より、
 $b u_n^p + \lambda f \in H^{-1}(\Omega)$である。
 $a > -\kappa_1$と合わせて、\eqref{eq:u_n+1}には
 唯一の弱解が存在する。

 ここで、次の事実を、$n$についての数学的帰納法を用いて証明する。
 \begin{equation}
  0 = u_0 < u_1 < \dots < u_n < \hat{u} ~\tin \Omega. \label{eq:u_n_ind}
 \end{equation}

 $n = 0$のときは、$\hat{u} > 0 ~\tin \Omega$であることから、
 \eqref{eq:u_n_ind}が成立する。
 $n \in \N$とする。$n$における\eqref{eq:u_n_ind}の成立を仮定し、
 $n+1$における\eqref{eq:u_n_ind}の成立を示す。
 \begin{align*}
  -\Delta u_{n+1} + a u_{n+1} &= b u_n^p + \lambda f, \\
  -\Delta u_{n} + a u_{n} &= b u_{n-1}^p + \lambda f
 \end{align*}
 の両辺を引くと、次が成立する。
 \[
  -\Delta (u_{n+1} - u_n) + a(u_{n+1} - u_n) 
 = b(u_n^p - u_{n-1}^p).
 \]
 右辺は仮定により$0$以上である。
 ゆえに強最大値原理より、$u_{n+1} > u_n ~\tin\Omega$である。
 また、
 \begin{align*}
  -\Delta \hat{u} + a \hat{u} &> b \hat{u}^p + \lambda f, \\
  -\Delta u_{n+1} + a u_{n+1} &= b u_{n}^p + \lambda f
 \end{align*}
 の両辺を引いて同様にすると、$\hat{u} > u_{n+1} ~\tin\Omega$
 もしたがう。以上により、\eqref{eq:u_n_ind}は$n+1$でも正しい。
 数学的帰納法により、任意の$n \in \N$について
 \eqref{eq:u_n_ind}の成立が示された。

 続いて、$\{u_n\}$が$H_0^1(\Omega)$における
 有界列であることを示す。
 $u_{n+1}$は\eqref{eq:u_n+1}の弱解であるから、
 任意の$\psi \in H_0^1(\Omega)$
 に対し、次が成立する。
 \begin{equation}
  \int_\Omega (Du_{n+1} \cdot D\psi + a u_{n+1} \psi) dx 
   = \int_\Omega bu_n^p \psi dx + \lambda \int_\Omega f\psi dx
   \label{eq:u_n+1_weaksol}
 \end{equation}
 $\psi = u_{n+1}$とすると、次が成立する。
 \[
 \int_\Omega (\lvert Du_{n+1} \rvert^2 
 + a \lvert u_{n+1} \rvert^2) dx 
 = \int_\Omega bu_n^p u_{n+1} dx 
 + \lambda \int_\Omega f u_{n+1} dx.
 \]
 ここで、右辺は、次の通りに評価される。
 \begin{equation}
  (\text{右辺}) \leq \int_\Omega b\hat{u}^{p+1} dx + \lambda
   \int_\Omega f \hat{u} dx < \infty. \label{eq:u_n+1_right}
 \end{equation}
 ここで$\hat{u} \in H_0^1(\Omega) \subset L^{p+1}(\Omega)$に
 注意した。また左辺について、
 \begin{equation}
  (\text{左辺}) \geq \int_\Omega \left( \lvert Du_{n+1} \rvert^2 + \kappa
   \lvert u_{n+1} \rvert^2 \right) dx = \| u_{n+1} \|_{H_0^1(\Omega)}
  \label{eq:u_n+1_left}
 \end{equation}
 もわかる。$\dnorm_\kappa$は$\dnorm_{H_0^1(\Omega)}$
 と同値なノルムである。したがって、\eqref{eq:u_n+1_right}および
 \eqref{eq:u_n+1_left}より、
 $\{u_n\}$は$H_0^1(\Omega)$の有界列である。

 ゆえに、必要ならば部分列をとることにより、
 $u \in H_0^1(\Omega)$が存在して、$n \to \infty$とすると、
 以下が成立する。
 \begin{align}
  u_n \xrightharpoonup{ \mbox{ ~ } } u & \ \ \text{weakly~} \tin
  H_0^1(\Omega), \label{eq:minimal_u_n_weakly} \\
  u_n \xrightarrow{ \mbox{ ~ } } u & \ \ \tin L^q(\Omega) \notag \ \
   (q < p+1), \\
  u_n \xrightarrow{ \mbox{ ~ } } u & \ \ \ae \tin \Omega. 
    \label{eq:minimal_u_n_ae}
 \end{align}
 ここで$u$が\ref{eq:prob_main}の弱解であることを示す。
 \eqref{eq:minimal_u_n_weakly}により、次が成立する。
 \[
 \int_\Omega (Du_{n+1} \cdot D\psi + a u_{n+1} \psi) dx
 \xrightarrow{n \to \infty}
 \int_\Omega (Du \cdot D\psi + a u \psi) dx.
 \]
 また、$b \in L^\infty(\Omega)$、
 $\hat{u}, \psi \in H_0^1(\Omega) \subset L^{p+1}(\Omega)$より、
 \[
  \left\lvert bu_n \psi \right\rvert \leq b \hat{u}^p \lvert\psi\rvert \ \ \ae
 \tin \Omega
 \]
 の右辺は可積分である。\eqref{eq:minimal_u_n_ae}より、
 優収束定理から、次を得る。
 \[
 \int_\Omega bu_n^p \psi dx \xrightarrow{n \to \infty} 
 \int_\Omega bu^p \psi dx.
 \]
 したがって、\eqref{eq:u_n+1_weaksol}で$n \to \infty$とすると次を得る。
 \begin{equation}
  \int_\Omega (Du \cdot D\psi + a u \psi) dx 
   = \int_\Omega bu^p \psi dx + \lambda \int_\Omega f\psi dx.
   \label{eq:minimal_u_weaksol}
 \end{equation}
 $\psi \in H_0^1(\Omega)$は任意であるから、
 $u \in H_0^1(\Omega)$は\ref{eq:prob_main}の弱解である。
 
 最後に、$u$は
 \ref{eq:prob_main}のminimal solutionであることを示す。
 $\tilde{u} \in H_0^1(\Omega)$を\ref{eq:prob_main}の弱解とする。
 このとき、\eqref{eq:u_n_ind}と同様の議論により、
 $\tilde{u} > u_n ~\tin \Omega$が数学的帰納法で示される。
 $n \to \infty$として、$\tilde{u} \geq u ~\tin \Omega$となる。
 よって$u$は\ref{eq:prob_main}のminimal solutionである。\qedhere
\end{proof}

補題~\ref{lem:minimal_itt}から、次の事実がしたがう。

\begin{lem} \label{lem:minimal_va}
 \begin{enumerate}[1.] \sage
  \item $\lambda_0 > 0$が存在して、$S_{\lambda_0} \neq \emptyset$とする。
        このとき、$0 < \lambda < \lambda_0$に対し、
        $S_\lambda \neq \emptyset$となる。
  \item $\lambda > 0$とする。$S_\lambda \neq \emptyset$ならば、
        \ref{eq:prob_main}に
        は minimal solution $\underline{u}_\lambda
        \in S_\lambda$が存在する。
  \item $0 < \lambda_1 < \lambda_2$とする。$S_{\lambda_1} \neq
        \emptyset$、
        $S_{\lambda_2} \neq \emptyset$ならば、
        $\underline{u}_{\lambda_1} \in S_{\lambda_1}$
        $\underline{u}_{\lambda_2} \in S_{\lambda_2}$について、
        $\underline{u}_{\lambda_1} < \underline{u}_{\lambda_2} ~\tin
        \Omega$が成立する。
  \item 補題~\ref{lem:imp}における\ref{eq:prob_main}の弱解を
        $u_\lambda$とする。このとき、$u_\lambda =
        \underline{u}_\lambda$である。
 \end{enumerate}
\end{lem}

\begin{proof}
 \begin{enumerate}[1.] \sage
  \item $u_{\lambda_0} \in S_{\lambda_0}$とする。$\hat{u} =
        u_{\lambda_0}$とし
        補題~\ref{lem:minimal_itt}を適用すると結論が得られる。
  \item $u_{\lambda} \in S_{\lambda}$とする。$\hat{u} =
        u_{\lambda}$として
        補題~\ref{lem:minimal_itt}を適用すると、
        \ref{eq:prob_main}の
        minimal solution $\underline{u}_\lambda$が得られる。
  \item $\hat{u} = \underline{u}_{\lambda_2}$として、
        補題~\ref{lem:minimal_itt}
        \eqref{eq:u_n_ind}を適用すると、
        $\underline{u}_{\lambda_1} \leq
        \underline{u}_{\lambda_2} ~\tin \Omega$が得られる。
        \begin{align*}
         -\Delta \underline{u}_{\lambda_1} + a
         \underline{u}_{\lambda_1} 
         &= b \underline{u}_{\lambda_1}^p + \lambda_1 f, \\
         -\Delta \underline{u}_{\lambda_2} + a
         \underline{u}_{\lambda_2} 
         &= b \underline{u}_{\lambda_2}^p + \lambda_2 f
        \end{align*}
        の両辺を引くと、次が成立する。
        \[
         -\Delta (\underline{u}_{\lambda_2} - \underline{u}_{\lambda_1}) + a
         (\underline{u}_{\lambda_2} - \underline{u}_{\lambda_1} )
         = b (\underline{u}_{\lambda_2}^p -
        \underline{u}_{\lambda_2}) + (\lambda_2 - \lambda_1) f.
        \]
        右辺が$0$以上であること、および、
        $a > -\kappa_1$により、強最大値原理を用いると、
        $\underline{u}_{\lambda_1} <
        \underline{u}_{\lambda_2} ~\tin \Omega$がしたがう。
  \item $u_\lambda \in S_\lambda$より、
        $S_\lambda \neq \emptyset$である。
        したがって、2.~より、\ref{eq:prob_main}は minimal solution
        $\underline{u}_\lambda$をもつ。よって、
        \eqref{eq:minimal_u_weaksol}
        で$u = \psi = \underline{u}_\lambda$とおくと、
        以下が得られる。
        \begin{equation}
         \int_\Omega \left( \lvert D\underline{u}_\lambda \rvert^2 + a
                      \lvert \underline{u}_\lambda \rvert^2 \right) dx 
          = \int_\Omega b\underline{u}_\lambda^p dx 
          + \lambda \int_\Omega f \underline{u}_\lambda dx.
          \label{eq:minimal_inp_same_weak}
        \end{equation}
        ここで、
        minimal solution の
        $H_0^1(\Omega)$ノルムが、$\lambda \searrow
        0$のとき、$0$に収束することを示す。
        \begin{align*}
         (\text{\eqref{eq:minimal_inp_same_weak}の左辺}) \geq 
         \int_\Omega \left( \lvert
         D\underline{u}_\lambda
         \rvert^2 +
         \kappa \lvert \underline{u}_\lambda \rvert^2 \right) dx 
         \geq C
         \left\| \underline{u}_\lambda \right\|_{H_0^1(\Omega)}^2.
        \end{align*}
        中辺は$\| \underline{u}_\lambda \|_\kappa^2$であり、
        $\dnorm_{\kappa}$は
        $\dnorm_{H_0^1(\Omega)}$と同値であるから、
        $C > 0$は$\dnorm_{H_0^1(\Omega)}$の
        中身によらない定数であることに
        注意されたい。
        また、
        $\underline{u}_\lambda \leq u_\lambda ~\tin \Omega$より、
        次がしたがう。
        \begin{align*}
         (\text{\eqref{eq:minimal_inp_same_weak}の右辺})
         &\leq \int_\Omega b \underline{u}_\lambda^{p+1}
         dx + \lambda \int_\Omega f \underline{u}_\lambda dx \\
         &\leq \left\| b \right\|_{L^\infty(\Omega)} \left\|
         u_\lambda \right\|_{L^{p+1}(\Omega)}^{p+1} +
         \lambda \left\| f \right\|_{H^{-1}(\Omega)} \left\|
         u_\lambda \right\|_{L^2(\Omega)} \\
         &\leq C^\prime \left\|
         u_\lambda \right\|_{H_0^1(\Omega)}^{p+1} +
         C^{\prime\prime} \left\| u_\lambda
         \right\|_{H_0^1(\Omega)}
        \end{align*}
        ここで、$C^\prime, C^{\prime\prime} > 0$は、
        $\dnorm_{H_0^1(\Omega)}$の中身によらない定数である。
        以上より、以下が成立する。
        \[
         C
        \left\| u \right\|_{H_0^1(\Omega)} \leq 
        C^\prime \left\|
        u_\lambda \right\|_{H_0^1(\Omega)}^{p+1} +
        C^{\prime\prime}
        \left\| u_\lambda \right\|_{H_0^1(\Omega)}.
        \]
        補題~\ref{lem:imp}より、$\lambda \searrow 0$のとき、
        $\left\| u_\lambda \right\|_{H_0^1(\Omega)} \searrow 0$
        が成立する。
        ゆえに、$\left\| \underline{u}_\lambda
        \right\|_{H_0^1(\Omega)}
        \searrow 0$となる。
        再び補題~\ref{lem:imp}によると、
        $\lambda > 0$が十分小さいとき、
        $u_\lambda$は\ref{eq:prob_main}
        の唯一の弱解であった。したがってこのことは
        $u_\lambda = \underline{u}_\lambda$を示している。 \qedhere
 \end{enumerate}
\end{proof}

\subsection{解が存在する$\lambda$の有界性}

補題~\ref{lem:imp}により、$\lambda > 0$が存在して、
\ref{eq:prob_main}の解が存在する。
補題~\ref{lem:minimal_va}により、
\ref{eq:prob_main}の解が存在する$\lambda$が見つかれば、
それより小さい$\lambda$については、\ref{eq:prob_main}の解が存在する。
そこで、\ref{eq:prob_main}の解が存在する$\lambda$が
どこまで大きくなるのかを調べる。そのために次の記号を置く。

\begin{nota} \label{nota:ext}
 $\bar{\lambda} = \sup \{ \lambda \geq 0 \mid S_\lambda \neq \emptyset
 \}$と定める。
\end{nota}

ここから先は、$\bar{\lambda} < \infty$を示すことを目標に議論を進める。
その準備として、$\lambda > 0$によらない$H_0^1(\Omega)$の元$g_0$を用意
する。

\begin{nota} \label{nota:g_0}
 $g_0 \in H_0^1(\Omega)$を
 \begin{align}
  \left\{
  \begin{aligned}
   -\Delta g_0 + a g_0 
    &= f  & &\tin \Omega,  \\
   g_0 &= 0 & &\ton \partial\Omega
  \end{aligned}
  \right. \label{eq:g_0}
 \end{align}
 の唯一の弱解と定める。 
\end{nota}
 
$g_0$について、次の補題を示す。

\begin{lem} \label{lem:g_0}
 固有値問題
 \[
  -\Delta \phi + a \phi = \mu b (g_0)^{p-1}\phi ~\tin \Omega, \ \
 \phi \in H_0^1(\Omega)
 \]
 の第$1$固有値を$\mu_1$とする。このとき、$\mu_1 > 0$である。
 また、$\mu_1$に付随する固有関数$\phi_1$のうち、
 $\phi_1 > 0 ~\tin \Omega$を
 みたすものがある。
\end{lem}

\begin{proof}
 $\mu_1$はレーリッヒ商により、
 \begin{equation}
  \mu_1 = \inf_{\psi \in H_0^1(\Omega), \psi \not\equiv 0}
   \frac{\displaystyle \int_\Omega 
   \left( \lvert D \psi \rvert^2 + a
    \lvert \psi \rvert^2
   \right) dx}{\displaystyle \int_\Omega b (g_0)^{p-1} \psi^2 dx}
   \label{eq:g_0_lin_mu_1}
 \end{equation}
 と特徴付けられる。また、\eqref{eq:g_0_lin_mu_1}の右辺の下限を
 達成する関数$\phi \in H_0^1(\Omega)$があるとすれば、$\phi$が$\mu_1$に
 付随する固有関数である。

 \eqref{eq:g_0_lin_mu_1}より、
 以下が成立する$H_0^1(\Omega)$の点列$\{ \psi_n \}$が存在する。
 \begin{align}
  \int_\Omega b(g_0)^{p-1} \psi_n^2 dx &= 1, \label{eq:g_0_seq_psi_1} \\
  \int_\Omega \left( \lvert D\psi_n \rvert^2 
  + a \lvert \psi_n \rvert^2 \right) dx
  & \searrow \mu_1. \label{eq:g_0_seq_psi_2}
 \end{align}
 $a > \kappa$であるから、\eqref{eq:g_0_seq_psi_2}の左辺は
 $\left\| \psi_n \right\|_\kappa^2$以下である。
 $\dnorm_\kappa$は
 $\dnorm_{H_0^1(\Omega)}$と同値なノルムであるから、
 $\{ \psi_n \}$は$H_0^1(\Omega)$の有界列である。
 
 ゆえに、必要ならば部分列をとることにより、
 $\phi_1 \in H_0^1(\Omega)$が存在して、$n \to \infty$とすると、
 以下が成立する。
 \begin{align}
  \psi_n \xrightharpoonup{ \mbox{ ~ } } \phi_1 & \ \ \text{weakly~} \tin
  H_0^1(\Omega), \label{eq:psi_n_weakly} \\
  \psi_n \xrightarrow{ \mbox{ ~ } } \phi_1 & \ \ \tin L^q(\Omega) \ \
   (q < p+1), \label{eq:psi_n_L^q} \\
  \psi_n \xrightarrow{ \mbox{ ~ } } \phi_1 & \ \ \ae \tin \Omega. 
    \label{eq:psi_n_ae} 
 \end{align}
 \eqref{eq:psi_n_weakly}より、$H_0^1(\Omega)$ノルムの
 弱下半連続性から、次が成立する。
 \[
  \liminf_{n \to \infty} \left\| \psi_n \right\|_{H_0^1(\Omega)}
 \geq \left\| \phi_1 \right\|_{H_0^1(\Omega)}.
 \]
 ゆえに、\eqref{eq:psi_n_L^q}と合わせて、以下が成立する。
 \begin{equation}
  \mu_1 \geq \int_\Omega \left( \lvert D\phi_1 \rvert^2 + a \lvert
                          \phi_1 \rvert^2
                         \right) dx. \label{eq:psi_infty_1}
 \end{equation}
 また、ソボレフ埋め込み$H_0^1(\Omega) \subset L^{p+1}(\Omega)$より、
 $H_0^1(\Omega)$の有界列$\{ \psi_n \}$は$L^{p+1}(\Omega)$の
 有界列である。したがって、$\{ \psi_n^2 \}$は$L^{N/(N-2)}(\Omega)$の
 有界列である。よって、必要なら部分列をとると、
 $\{\psi_n^2 \}$は$L^{N/(N-2)}(\Omega)$の弱収束列となる。
 一方\eqref{eq:psi_n_ae}から、$\{ \psi_n^2 \}$は$\phi_1^2$に
 $\Omega$上ほとんどいたるところ収束する。したがって、次が成立する。
 \[
 \psi_n^2 \xrightharpoonup{ \mbox{ ~ } } \phi_1^2 \ \ \text{weakly~} \tin
 L^{N/(N-2)}(\Omega).
 \]
 $g_0 \in L^{p+1}(\Omega)$より、$b (g_0)^{p-1} \in
 L^{N/2} (\Omega)$である。$\left(L^{N/(N-2)}(\Omega)\right)^*
 \cong L^{N/2}(\Omega)$
 より、次が成立する。
 \begin{equation}
  \int_\Omega b(g_0)^{p-1} \psi_n^2 dx \xrightarrow{n \to \infty}
   \int_{\Omega} b(g_0)^{p-1} \phi_1^2 dx. \label{eq:psi_infty_2}
 \end{equation}
 \eqref{eq:psi_infty_2}の証明は、\cite{MR1400007}~の Lemma~2.13 によった。
 \eqref{eq:psi_infty_1}と\eqref{eq:psi_infty_2}により、
 次がしたがう。
 \begin{equation}
  \mu_1 \geq \frac{\displaystyle \int_\Omega 
   \left( \lvert D\phi_1 \rvert^2 + a \lvert \phi_1 \rvert^2
   \right) dx}{\displaystyle \int_{\Omega} b(g_0)^{p-1} 
   \phi_1^2 dx}. \label{eq:g_0_ineq}
 \end{equation}
 \eqref{eq:g_0_lin_mu_1}により、
 \eqref{eq:g_0_ineq}の不等号は
 実際には等号が成立する。すなわち、
 \eqref{eq:g_0_lin_mu_1}の右辺の下限は
 $\phi_1 \in H_0^1(\Omega)$により達成される。
 よって$\mu_1 > 0$である。
 
 \eqref{eq:g_0_lin_mu_1}の右辺の形から、
 $\phi_1$が\eqref{eq:g_0_lin_mu_1}の右辺の下限を達成するならば、
 $\left| \phi_1 \right|$も下限を達成する。
 すなわち、$\phi_1 \geq 0 ~\tin \Omega$となる第$1$固有関数がある。
 この$\phi_1$について、次が成立する。
 \[
  - \Delta \phi_1 + a \phi_1 = \mu_1 b (g_0)^{p-1} 
 \phi_1 \geq 0 ~\tin \Omega.
 \]
 ゆえに、強最大値原理により、$\phi_1 > 0 ~\tin \Omega$となる。
 \qedhere
\end{proof}

$g_0$を用いて、次の命題を証明する。

\begin{prop} \label{prop:bar_lambda}
 $\bar{\lambda}$を記号~\ref{nota:ext}のものとする。
 $0 < \bar{\lambda} < \infty$である。
\end{prop}

\begin{proof}
補題~\ref{lem:imp}により、$\lambda_0 > 0$が存在し、$0 < \lambda <
 \lambda_0$に対して、\ref{eq:prob_main}の解が存在する。
ゆえに$\bar{\lambda} > 0$である。
そこで、$\bar{\lambda} < \infty$を示せば証明が完了する。

$\lambda > 0$は、$S_\lambda \neq \emptyset$をみたすものとする。
$u \in S_\lambda$とし、$v = u - \lambda g_0$とする。
このとき、次が成立する。
\[
 -\Delta v + av = bu^p \geq 0
\]
したがって、強最大値原理より、$v > 0 ~\tin \Omega$である。
つまり、$u > \lambda g_0 ~\tin \Omega$がしたがう。
よって、以下が成立する。
\begin{equation}
 - \Delta u + au \geq bu^p \geq b \lambda^{p-1} (g_0)^{p-1} u ~\tin
  \Omega. \label{eq:lambda_infty_g_0} 
\end{equation}
一方、補題~\ref{lem:g_0}により、以下が成立する
$\mu_1 > 0$、
$\phi_1 \in H_0^1(\Omega)$、$\phi_1 > 0 ~\tin \Omega$
が存在する。
\begin{equation}
 -\Delta \phi_1 + a \phi_1 = \mu_1 b (g_0)^{p-1} \phi_1 ~\tin \Omega. 
  \label{eq:lambda_infty_phi_1} 
\end{equation}
 そこで、
 $\text{\eqref{eq:lambda_infty_g_0}} \times \phi_1 - 
 \text{\eqref{eq:lambda_infty_phi_1}} \times u $を$\Omega$
 上積分すると、次を得る。
 \[
  0 \geq (\lambda^{p-1} - \mu_1) \int_\Omega b(g_0)^{p-1} u \phi_1 dx.
 \]
 ここで、$b \geq 0 ~\tin \Omega$、
 $b \not \equiv 0$、$g_0, u, \phi_1 > 0 ~\tin \Omega$であるから、
 右辺の積分は正である。ゆえに、$\lambda^{p-1} - \mu_1 \leq 0$である。
 つまり、$\lambda \leq \mu_1^{1/(p-1)}$となる。
 $\lambda > 0$は$S_\lambda \neq \emptyset$をみたす任意の正の数
 であるから、$\bar{\lambda} \leq \mu_1 ^{1/(p-1)} < \infty$がしたがう。
 \qedhere
\end{proof}

\begin{proof}[定理~\ref{thm:minimal_solution}]
 命題~\ref{prop:bar_lambda}により、
\end{proof}

\subsection{minimal solutionに関する線形化固有値問題}

\ref{eq:prob_main}の minimal solution についての
線形化固有値問題
\begin{equation}
 -\Delta \phi + a \phi = \mu p b (\underline{u}_\lambda)^{p-1} \phi
  ~\tin \Omega, \ \ \phi \in H_0^1(\Omega) \label{eq:lin_prob_min_sol}
\end{equation}
を考察する。特に第$1$固有値、第$1$固有関数について論ずる。

\begin{nota}
 \ref{eq:prob_main}の
 minimal solution $\underline{u}_\lambda \in S_\lambda$ に関する
 線形化固有値問題\eqref{eq:lin_prob_min_sol}の第$1$固有値を
 $\mu_1(\lambda)$とかく。
\end{nota}

\begin{lem} \label{lem:lin_p}
 $0 < \lambda < \bar{\lambda}$とする。
 このとき、以下が成立する。
 \begin{enumerate}[1.] \sage
  \item $\mu_1(\lambda) > 0$である。
        また、$\mu_1(\lambda)$に付随する固有関数$\phi_1$のうち、
        $\phi_1 > 0 ~\tin \Omega$を
        みたすものが存在する。
  \item 任意の$\psi \in H_0^1(\Omega)$に対し、次が成立する。
        \begin{equation}
         \int_\Omega \left( \lvert D\psi \rvert^2 + a \lvert \psi
                      \rvert^2 \right)
          dx \geq \mu_1(\lambda) \int_\Omega pb(\underline{u}_\lambda)^{p-1}
          \psi^2 dx. \label{eq:lin_prob_min_sol_mu_1}
        \end{equation}
 \end{enumerate}
\end{lem}

\begin{proof}
 \begin{enumerate}[1.] \sage
  \item 補題~\ref{lem:g_0}と同様である。
  \item $\mu_1(\lambda)$のレーリッヒ商による特徴付け
        \begin{equation}
         \mu_1(\lambda) = \inf_{\psi \in H_0^1(\Omega), \psi \not \equiv 0} 
        \frac{\displaystyle \int_\Omega 
        \left( \left\lvert D\psi \right\rvert^2 + a\lvert\psi\rvert^2 \right)
          dx }{\displaystyle \int_\Omega pb(\underline{u}_\lambda)^{p-1}
          \psi^2 dx } \label{eq:mu1_quotient}        
        \end{equation}
        から\eqref{eq:lin_prob_min_sol_mu_1}が成立する。
 \end{enumerate}
\end{proof}

補題~\ref{lem:lin_p}から即座に、
$0 < \lambda < \bar{\lambda}$ならば
$\mu_1(\lambda) > 0$であることがわかる。
次の補題では、方程式\ref{eq:prob_main}に着目し、
$\mu_1(\lambda)$についてより多くの情報を引き出す。

\begin{lem} \label{lem:lin_1}
 $0 < \lambda < \bar{\lambda}$とする。このとき、$\mu_1(\lambda) > 1$で
 ある。
\end{lem}

\begin{proof}
 $\hat{\lambda}$を$0 < \lambda < \hat{\lambda} < \bar{\lambda}$を
 みたすものとする。$z = \underline{u}_{\hat{\lambda}} -
 \underline{u}_\lambda$とおく。補題~\ref{lem:minimal_va}.3 より、
 $z > 0 ~\tin \Omega$である。
 \begin{align*}
  -\Delta \underline{u}_{\hat{\lambda}} + a \underline{u}_{\hat{\lambda}} &= b
  \underline{u}_{\hat{\lambda}}^p + \hat{\lambda} f, \\   
  -\Delta \underline{u}_{\lambda} + a \underline{u}_\lambda &= b
  \underline{u}_\lambda^p + \lambda f
 \end{align*}
 の両辺を引いて、次を得る。
 \[
  -\Delta z + az = b (\underline{u}_{\hat{\lambda}}^p -
 \underline{u}_\lambda^p)
 + (\hat{\lambda} - \lambda) f.
 \]
 $x \geq 0$に対し、$x \mapsto x^p$は下に凸であるから、次がしたがう。
 \[
  \underline{u}_{\hat{\lambda}}^p - \underline{u}_\lambda^p > 
 p \underline{u}_\lambda^{p-1} (\underline{u}_{\hat{\lambda}} -
 \underline{u}_\lambda) = p \underline{u}_\lambda^{p-1} z.
 \]
 $(\hat{\lambda} - \lambda) f \geq 0$と合わせて、次を得る。
 \begin{equation}
  -\Delta z + az > bp \underline{u}_\lambda^{p-1} z  \ \ \tin \Omega.
   \label{eq:mu_1_z_1}
 \end{equation}

 $\mu_1 = \mu_1(\lambda)$とする。
 補題~\ref{lem:lin_p}より、$\phi_1 > 0 ~\tin \Omega$があって、
 \begin{equation}
  -\Delta \phi_1 + a \phi_1 =
   \mu p b \underline{u}_\lambda^{p-1} \phi_1  \ \ \tin \Omega
   \label{eq:mu_1_z_2}
 \end{equation}
 $ \text{\eqref{eq:mu_1_z_1}} \times \phi_1 - 
 \text{\eqref{eq:mu_1_z_2}} \times z$を
 $\Omega$上積分すると、
 \[
  0 > (1 - \mu_1) p \int_\Omega b \underline{u}_\lambda^{p-1} \phi_1 z dx
 \]
 となる。
 ここで、$b \geq 0 ~\tin \Omega$、
 $b \not \equiv 0$、
 $\underline{u}_\lambda, z, \phi_1 > 0 ~\tin \Omega$であるから、
 右辺の積分は正である。ゆえに、$1 - \mu_1 < 0$である。
 つまり$\mu_1 > 1$である。 \qedhere
\end{proof}

% Local Variables:
% mode: yatex
% TeX-master: "main.tex"
% End: