\section{minimal solutionの存在と性質}

本節では、\ref{eq:prob_main}の解のうち、
minimal solution について取り扱う。
まずは minimal solution を定義する。

\begin{defn}
 $\lambda > 0$に対し、
 \[
   S_\lambda = \{ u \in H_0^1(\Omega) \mid u \text{は
 \ref{eq:prob_main} の弱解である}\}
 \]
 と定める。$\underline{u}_\lambda \in S_\lambda$が{\bf minimal solution} で
 あるとは、任意の$u \in S_\lambda$に対し、
 \[
  \underline{u}_\lambda \leq u ~\tin \Omega
 \]
 が成立することをいう。
\end{defn}

以下では、\ref{eq:prob_main}の minimal solution を
$\underline{u}_\lambda$
と表記する。

minimal solution を調べる第一歩として、
$\lambda > 0$が十分小さいときに、\ref{eq:prob_main}が
弱解を持つことを、陰関数定理より示す。

\begin{lem} \label{lem:imp}
 \begin{enumerate}[1.]
  \item $H_0^1(\Omega)$の原点の近傍$U$があって、
        $\lambda_0 > 0$があって、$0 < \lambda \leq \lambda_0$に対し、
        \ref{eq:prob_main}は$U$における唯一の弱解
        $u_\lambda$をもつ。また、
        \[
        \left\| u_\lambda
        \right\|_{H^1_0(\Omega)} \to 0 \ \ (\lambda \searrow 0)
        \]
        となる。
  \item さらに、$f \in C^\alpha(\bar{\Omega})$を仮定する。
        このとき、1.~の$u_\lambda$は、$u_\lambda \in
        C^{2+\alpha}(\Omega)$を
        みたし、
        \[
        \left\| u_\lambda
        \right\|_{C^{2+\alpha}(\Omega)} \to 0 \ \ (\lambda \searrow 0)
        \]
        となる。
 \end{enumerate}
\end{lem}

\begin{proof}
 \begin{enumerate}[1.]
  \item $\Phi \colon [0,\infty) \times H^1_0 (\Omega) \to H^{-1}(\Omega)$を
        \begin{equation}
         \Phi (\lambda, u) = -\Delta u + au - b (u_{+})^p - \lambda f
          \label{eq:def_of_Phi}
        \end{equation}
        とする。$\Phi$の$u$についてのフレッシェ微分は、
        $w \in H^1_0(\Omega)$とすると、
        \[
         \Phi_u (\lambda, u) \colon w \mapsto -\Delta w + aw - b p(u_+)^{p-1} w
        \]
        となる。とくに、
        \[
         \Phi_u (0, 0) w = -\Delta w + aw 
        \]
        である。$a > -\kappa_1$により、$\Phi_u(0,0) \colon
        H^1_0(\Omega) \to H^{-1} (\Omega)$は可逆である。ゆえに、
        陰関数定理より、$\lambda_0 > 0$があって、
        $H_0^1(\Omega)$の原点の近傍$U$があって、
        $0 < \lambda \leq \lambda_0$に対し、$u_\lambda \in U$が
        ただ$1$つあって、$\Phi(\lambda, u_\lambda) = 0$、かつ、
        $\lim_{\lambda \searrow 0} \left\| u_\lambda \right\|_{H^1_0(\Omega)} =
        0$となる。つまり、$u_\lambda$は、
        \begin{align}
         \left\{
         \begin{aligned}
           -\Delta u + a u &= b (u_+)^p + \lambda f  & &\tin \Omega,  \\
           u &= 0 & &\ton \partial\Omega
         \end{aligned}
         \right. \label{eq:prob_lem_ifthm}
        \end{align}
        の弱解である。
        ここで$b (u_+)^p + \lambda f \geq 0$であり、$a > -\kappa_1$で
        あるから、強最大値原理により、$u_\lambda > 0 ~\tin
        \Omega$となる。よって、$u_\lambda$は\ref{eq:prob_main}の
        $U$における唯一の弱解である。
  \item $f \in C^\alpha(\bar{\Omega})$のとき、
        $\Phi \colon [0,\infty) \times C^{2+\alpha} (\bar{\Omega})
        \to C^\alpha(\bar{\Omega})$を、\eqref{eq:def_of_Phi}で定義する。
        以下、1.~の証明と同様にすると、
        $u_\lambda \in
        C^{2+\alpha}(\Omega)$と
        $\left\| u_\lambda
        \right\|_{C^{2+\alpha}(\Omega)} \to 0 \ \ (\lambda \searrow
        0)$
        が示される。\qedhere
 \end{enumerate}
\end{proof}

以下では基本的に、1.~の結果を使用し、弱解の枠組みで進行していく。
2.~の結果は、\S~\ref{sec:sym}で使用する。

続いては、ある
$\lambda = \hat{\lambda}$で\ref{eq:prob_main}が
優解をもつときに、$0 < \lambda \leq \hat{\lambda}$で
minimal solution の存在が従うことを示す。

\begin{lem} \label{lem:minimal_itt}
 $\hat{\lambda} > 0$とする。$\hat{u} \in H_0^1(\Omega)$があって、
 $\hat{u} > 0 ~\tin \Omega$かつ
 \[
  \Delta \hat{u} + a\hat{u} \geq b \hat{u}^p + \hat{\lambda}f
 \tin \Omega
 \]
 をみたすとする。このとき、$\lambda \in (0,\hat{\lambda} ]$
 に対し、\ref{eq:prob_main}の minimal solution $\underline{u}_\lambda$
 が存在する。
 また、$\underline{u}_\lambda < \hat{u} \tin \Omega$となる。
\end{lem}

\begin{proof}
 $H_0^1(\Omega)$の点列$\{ \underline{u}_n \}$を、次の通りに
 帰納的に定める。$u_0 \equiv 0$とする。$u_n$が
 定まっているときに、線形方程式
  \begin{align}
   \left\{
   \begin{aligned}
    -\Delta u_{n+1} + a u_{n+1} 
    &= b u_n^p + \lambda f  & &\tin \Omega,  \\
    u_{n+1} &= 0 & &\ton \partial\Omega
   \end{aligned}
   \right. \label{eq:u_n+1}
  \end{align}
 の唯一の弱解を$u_{n+1} \in H_0^1(\Omega)$と定める。
 
 \eqref{eq:u_n+1}が唯一の弱解であることを確かめる。
 ソボレフ埋め込みにより、
 $
  u_n \in H_0^1(\Omega) \subset L^{p+1}(\Omega)
 $
 だから、
 $
  u_n^p \subset L^{(p+1)/p}(\Omega)
 = L^{2N/(N+2)}(\Omega) \subset H^{-1}(\Omega)
 $
 である。$b \in L^\infty(\Omega)$、$f \in H^{-1}(\Omega)$より、
 $b u_n^p + \lambda f \in H^{-1}(\Omega)$である。
 $a > -\kappa_1$と合わせて、\eqref{eq:u_n+1}には確かに
 唯一の弱解が存在する。

 ここで、次の事実を、$n$についての数学的帰納法で証明する。
 \begin{equation}
  0 = u_0 < u_1 < \dots < u_n < \hat{u} ~\tin \Omega \label{eq:u_n_ind}
 \end{equation}

 $n = 0$のときは、$\hat{u} > 0 ~\tin \Omega$により成立する。
 $n \in \N$とする。$n$での\eqref{eq:u_n_ind}の成立を仮定し、
 $n+1$での成立を示す。
 \begin{align*}
  -\Delta u_{n+1} + a u_{n+1} &= b u_n^p + \lambda f, \\
  -\Delta u_{n} + a u_{n} &= b u_{n-1}^p + \lambda f
 \end{align*}
 の両辺を引くと、
 \[
  -\Delta (u_{n+1} - u_n) + a(u_{n+1} - u_n) = b(u_n^p - u_{n-1}^p)
 \]
 となる。右辺は仮定により$0$以上である。
 ゆえに強最大値原理より、$u_{n+1} > u_n ~\tin\Omega$である。
 また、
 \begin{align*}
  -\Delta \hat{u} + a \hat{u} &= b \hat{u}^p + \lambda f, \\
  -\Delta u_{n+1} + a u_{n+1} &= b u_{n}^p + \lambda f
 \end{align*}
 の両辺を引いて同様にすると、$\hat{u} > u_{n+1} ~\tin\Omega$
 も従う。以上により、\eqref{eq:u_n_ind}は$n+1$でも正しい。
 数学的帰納法により、$n \in \N$での\eqref{eq:u_n_ind}の成立が示された。

 続いて、$\{u_n\}$が$H_0^1(\Omega)$での有界列であることを示す。
 $u_{n+1}$は\eqref{eq:u_n+1}の弱解であるから、$\psi \in H_0^1(\Omega)$
 に対し、
 \begin{equation}
  \int_\Omega (Du_{n+1} \cdot D\psi + a u_{n+1} \psi) dx 
   = \int_\Omega bu_n^p \psi dx + \lambda \int_\Omega f\psi dx
   \label{eq:u_n+1_weaksol}
 \end{equation}
 となる。$\psi = u_{n+1}$とすると、
 \[
 \int_\Omega (\lvert Du_{n+1} \rvert^2  + a \lvert u_{n+1} \rvert^2) dx 
 = \int_\Omega bu_n^p u_{n+1} dx + \lambda \int_\Omega f u_{n+1} dx
 \]
 となる。ここで、右辺は、
 \begin{equation}
  (\text{右辺}) \leq \int_\Omega b\hat{u}^{p+1} dx + \lambda
   \int_\Omega f \hat{u} dx < \infty \label{eq:u_n+1_right}
 \end{equation}
 と評価される。
 ここで$\hat{u} \in H_0^1(\Omega) \subset L^{p+1}(\Omega)$に
 注意した。また左辺について、
 \begin{equation}
  (\text{左辺}) \geq \int_\Omega \left( \lvert Du_{n+1} \rvert^2 + \kappa
   \lvert u_{n+1} \rvert^2 \right) dx \label{eq:u_n+1_left}
 \end{equation}
 もわかる。ここで
 $w \in H_0^1(\Omega)$に対し、
 \[
  \left\| w \right\|_\kappa = \int_\Omega \left( \lvert Dw \rvert^2 +
 \kappa \lvert w \rvert^2 \right) dx
 \]
 と定めると、$\kappa > -\kappa_1$およびポアンカレの不等式により、
 $\left\| \cdot \right\|_\kappa$は$\left\| \cdot
 \right\|_{H_0^1(\Omega)}$
 と同値なノルムである。従って、\eqref{eq:u_n+1_right}および
 \eqref{eq:u_n+1_left}より、$\{u_n\}$は$H_0^1(\Omega)$の有界列である。

 ゆえに、必要ならば部分列をとることにより、
 $u \in H_0^1(\Omega)$があって、$n \to \infty$とすると、
 \begin{align}
  u_n \xrightharpoonup{ \mbox{ ~ } } u & \ \ \text{weakly~} \tin
  H_0^1(\Omega), \label{eq:minimal_u_n_weakly} \\
  u_n \xrightarrow{ \mbox{ ~ } } u & \ \ \tin L^q(\Omega) \notag \ \
   (q < p+1), \\
  u_n \xrightarrow{ \mbox{ ~ } } u & \ \ \ae \tin \Omega 
    \label{eq:minimal_u_n_ae}
 \end{align}
 となる。ここで$u$が\ref{eq:prob_main}の弱解であることを示す。
 \eqref{eq:u_n+1_weaksol}で$n \to \infty$とすることを試みる。
 \eqref{eq:minimal_u_n_weakly}により、
 \[
 \int_\Omega (Du_{n+1} \cdot D\psi + a u_{n+1} \psi) dx
 \xrightarrow{n \to \infty}
 \int_\Omega (Du \cdot D\psi + a u \psi) dx
 \]
 となる。また、\eqref{eq:minimal_u_n_ae}と
 \[
  \left\lvert bu_n \psi \right\rvert \leq b \hat{u}^p \lvert\psi\rvert \ \ \ae
 \tin \Omega
 \]
 の右辺は可積分であること
 ($\hat{u}, \psi \in H_0^1(\Omega) \subset L^{p+1}(\Omega)$からわかる)より、
 優収束定理から、
 \[
 \int_\Omega bu_n^p \psi dx \xrightarrow{n \to \infty} 
 \int_\Omega bu^p \psi dx
 \]
 となる。したがって、\eqref{eq:u_n+1_weaksol}で$n \to \infty$とすると、
 \begin{equation}
  \int_\Omega (Du \cdot D\psi + a u \psi) dx 
   = \int_\Omega bu^p \psi dx + \lambda \int_\Omega f\psi dx
   \label{eq:minimal_u_weaksol}
 \end{equation}
 を得る。$\psi \in H_0^1(\Omega)$は任意であるから、
 $u \in H_0^1(\Omega)$は\ref{eq:prob_main}の弱解である。
 
 最後に、$u$は\ref{eq:prob_main}のminimal solutionであることを示す。
 $\tilde{u} \in H_0^1(\Omega)$を\ref{eq:prob_main}の弱解とする。
 このとき、\eqref{eq:u_n_ind}と同様の議論により、
 $\tilde{u} > u_n \tin \Omega$が数学的帰納法で示される。
 $n \to \infty$として、$\tilde{u} \geq u \tin \Omega$となる。
 よって$u$は\ref{eq:prob_main}のminimal solutionである。\qedhere
\end{proof}

補題~\ref{lem:minimal_itt}から、次の事実が従う。

\begin{lem}
 \begin{enumerate}[1.]
  \item $\lambda_0 > 0$があって、$S_{\lambda_0} \neq \emptyset$とする。
        このとき、$0 < \lambda < \lambda_0$に対し、
        $S_\lambda \neq \emptyset$となる。
  \item $\lambda > 0$とする。$S_\lambda \neq \emptyset$ならば、
        \ref{eq:prob_main}には minimal solution $\underline{u}_\lambda
        \in S_\lambda$がある。
  \item $0 < \lambda_1 < \lambda_2$とする。$S_{\lambda_1} \neq
        \emptyset$、
        $S_{\lambda_2} \neq \emptyset$ならば、
        $\underline{u}_{\lambda_1} \in S_{\lambda_1}$
        $\underline{u}_{\lambda_2} \in S_{\lambda_2}$について、
        $\underline{u}_{\lambda_1} < \underline{u}_{\lambda_2} ~\tin
        \Omega$が成立する。
  \item 補題~\ref{lem:imp}における\ref{eq:prob_main}の弱解を
        $u_\lambda$yとする。このとき、$u_\lambda =
        \underline{u}_\lambda$である。
 \end{enumerate}
\end{lem}

\begin{proof}
 \begin{enumerate}
  \item $u_{\lambda_0} \in S_{\lambda_0}$とする。$\hat{u} =
        u_{\lambda_0}$として
        補題~\ref{lem:minimal_itt}を適用すると結論を得る。
  \item $u_{\lambda} \in S_{\lambda}$とする。$\hat{u} =
        u_{\lambda}$として
        補題~\ref{lem:minimal_itt}を適用すると、\ref{eq:prob_main}の
        minimal solution $\underline{u}_\lambda$を得る。
  \item $\hat{u} = \underline{u}_{\lambda_2}$として、
        補題~\ref{lem:minimal_itt}
        \eqref{eq:u_n_ind}を適用すると、$\underline{u}_{\lambda_1} \leq
        \underline{u}_{\lambda_2} \tin \Omega$を得る。
        \begin{align*}
         -\Delta \underline{u}_{\lambda_1} + a
         \underline{u}_{\lambda_1} 
         &= b \underline{u}_{\lambda_1}^p + \lambda_1 f, \\
         -\Delta \underline{u}_{\lambda_2} + a
         \underline{u}_{\lambda_2} 
         &= b \underline{u}_{\lambda_2}^p + \lambda_2 f
        \end{align*}
        の両辺を引くと、
        \[
         -\Delta (\underline{u}_{\lambda_2} - \underline{u}_{\lambda_1}) + a
         (\underline{u}_{\lambda_2} - \underline{u}_{\lambda_1} )
         = b (\underline{u}_{\lambda_2}^p -
        \underline{u}_{\lambda_2}) + (\lambda_2 - \lambda_1) f
        \]
        を得る。右辺が$0$以上であること、および、
        $a > -\kappa_1$により、強最大値原理を用いると、
        $\underline{u}_{\lambda_1} <
        \underline{u}_{\lambda_2} ~\tin \Omega$が従う。
  \item $u_\lambda \in S_\lambda$より、$S_\lambda \neq \empty$である。
        したがって、2.~より、\ref{eq:prob_main}は minimal solution
        $\underline{u}_\lambda$をもつ。よって、
        \eqref{eq:minimal_u_weaksol}
        で$u = \psi = \underline{u}_\lambda$とすると、
        \begin{equation}
         \int_\Omega \left( \lvert D\underline{u}_\lambda \rvert^2 + a
                      \lvert \underline{u}_\lambda \rvert^2 \right) dx 
          = \int_\Omega b\underline{u}_\lambda^p dx 
          + \lambda \int_\Omega f \underline{u}_\lambda dx
          \label{eq:minimal_inp_same_weak}
        \end{equation}
        を得る。
        
        ここで、
        minimal solution の$H_0^1(\Omega)$ノルムが、$\lambda \searrow
        0$のとき、$0$に収束することを示す。
        \begin{align*}
         (\text{\eqref{eq:minimal_inp_same_weak}の左辺}) \geq 
         \int_\Omega \left( \lvert
         D\underline{u}_\lambda
         \rvert^2 +
         \kappa \lvert \underline{u}_\lambda \rvert^2 \right) dx = C
         \left\| u \right\|_{H_0^1(\Omega)}
        \end{align*}
        である。中辺は$\| \underline{u}_\lambda \|_\kappa$であり、
        $\left\| \cdot
        \right\|_{\kappa}$は
        $\left\| \cdot
        \right\|_{H_0^1(\Omega)}$と同値であるから、
        $C > 0$は$\left\| \cdot
        \right\|_{H_0^1(\Omega)}$の中身によらない定数であることに
        注意されたい。
        また、$\underline{u}_\lambda \leq u_\lambda ~\tin \Omega$より、
        \begin{align*}
         (\text{\eqref{eq:minimal_inp_same_weak}の右辺})
         &\leq \int_\Omega b \underline{u}_\lambda^{p+1}
         dx + \lambda \int_\Omega f \underline{u}_\lambda dx \\
         &\leq \left\| b \right\|_{L^\infty(\Omega)} \left\|
         u_\lambda \right\|_{L^{p+1}(\Omega)}^{p+1} +
         \lambda \left\| f \right\|_{H^{-1}(\Omega)} \left\|
         u_\lambda \right\|_{L^2(\Omega)} \\
         & C^\prime \left\|
         u_\lambda \right\|_{H_0^1(\Omega)}^{p+1} +
         C^{\prime\prime} \left\| u_\lambda \right\|_{H_0^1(\Omega)}
        \end{align*}
        である。ここで、$C^\prime, C^{\prime\prime} > 0$は、$\left\| \cdot
        \right\|_{H_0^1(\Omega)}$の中身によらない定数である。以上より、
        \[
         C
        \left\| u \right\|_{H_0^1(\Omega)} \leq 
        C^\prime \left\|
        u_\lambda \right\|_{H_0^1(\Omega)}^{p+1} +
        C^{\prime\prime} \left\| u_\lambda \right\|_{H_0^1(\Omega)}
        \]
        となる。
        補題~\ref{lem:imp}より、$\lambda \searrow 0$のとき、
        $\left\| u_\lambda \right\|_{H_0^1(\Omega)} \searrow 0$となる。
        ゆえに、$\left\| \underline{u}_\lambda
        \right\|_{H_0^1(\Omega)}
        \searrow 0$となる。

        再び補題~\ref{lem:imp}によると、
        $\lambda > 0$が十分小さいとき、$u_\lambda$は\ref{eq:prob_main}
        の唯一の弱解であった。したがってこのことは
        $u_\lambda = \underline{u}_\lambda$を示している。 \qedhere
 \end{enumerate}
\end{proof}