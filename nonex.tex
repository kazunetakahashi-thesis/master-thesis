%#!platex main.tex
\section{$\lambda$が小さい場合の second solution の非存在}
\label{sec:sym}

本節では、定理~\ref{thm:second_solution_nonex}.2を証明する。
本節を通して、定理~\ref{thm:second_solution_nonex}.2の仮定をおく。
以下では、$r$を引数とする関数$f$の導関数を、
$f^\prime$とも、$f_r$とも表す。

まずは、定理~\ref{thm:second_solution_nonex}.2の仮定の下での
解の正則性を検討し、以降球対称解のみ考慮すればよいことを証明する。

\begin{lem}
 定理~\ref{thm:second_solution_nonex}.2の仮定の下で、
 $\lambda_0 > 0$が存在し、$0 < \lambda \leq \lambda_*$
 に対し、以下が成立する。
 \begin{enumerate}[1.] \sage
  \item \ref{eq:prob_main}の minimal solution $\underline{u}_\lambda$
        は球対称解であり、
        $u = u(\lvert x \rvert)
        \in C^{2+\alpha}([0, R])$
        をみたす。
  \item \ref{eq:prob_sec}の弱解$v$は
        球対称解に限り、
        $v = v(\lvert x \rvert)
        \in C^{2+\alpha}([0, R])$
        をみたす。
  \item $v = v(\lvert x \rvert) \in C^{2+\alpha}([0, R])$が
        \ref{eq:prob_sec}の解であることは、
        $v = v(r)$が
        次の常備分方程式の解であることと同値である。
        \begin{align}
         \left\{
          \begin{aligned}
           -v_{rr} - \frac{N-1}{r}v_r  &= g(v, \underline{u}_\lambda)
           & &\tin (0, R),  \\
           v &> 0 & &\tin (0, R), \\
           v_r(0) = v(R) &= 0.  & &
          \end{aligned}
         \right. \label{eq:prob_rad}
        \end{align}
 \end{enumerate}
\end{lem}

\begin{proof}
 \begin{enumerate}[1.] \sage
  \item $\lambda_0 > 0$を
        補題~\ref{lem:imp}における$\lambda_0$とする。
        このとき、補題~\ref{lem:imp}.2により、
        $u \in C^{2+\alpha}(\bar{\Omega})$である。
        $s \geq 0$に対し、$r \in [0, R]$の関数
        $-a(r) s + b(r) s^p + \lambda f(r)$は
        単調減少である。したがって、\cite{MR544879}~より、
        \ref{eq:prob_main}の解
        $u = u(\lvert x \rvert) \in C^{2+\alpha}(\bar{\Omega})$
        は球対称解に限り、
        $0 \leq r \leq R$に対し、$u_r (r) < 0$である。
        とくに、
        \begin{align}
         \underline{u}_\lambda(r) & \in C^{2+ \alpha}([0, R]), 
          \label{eq:under_u_r} \\
         \underline{u}_\lambda^\prime(r) &< 0 \ \ (0 \leq r \leq R)
          \label{eq:under_u_r_prime}
        \end{align}
        が成立する。
  \item $v \in C^{2+\alpha}(\bar{\Omega})$である。
        $a, b$は球対称であるから、$g = g(t, s, \lvert x \rvert)$とみな
        せる。$t \geq 0$、$r \in [0, R]$に対し、
        \begin{align}
         \pdif{}{r} g(t, \underline{u}_\lambda(r), r)
         &= \pdif{}{r} \left( b(r) \left( (t + u_\lambda(r))^p -
         \underline{u}_\lambda(r)^p \right) a(r) t \right) \notag \\
         &= b(r) p \left( (t+ \underline{u}_\lambda(r))^{p-1} -
         \underline{u}_\lambda(r)^{p-1} \right)
         \underline{u}_\lambda^\prime(r) + b^\prime(r) \left( (t +
         \underline{u}_\lambda(r))^p  - \underline{u}_\lambda(r)^p
         \right) - a^\prime(r) t \label{eq:pdifgtur}
        \end{align}
        である。\eqref{eq:under_u_r_prime}、$b^\prime(r) \leq 0$、
        $a^\prime(r) \geq 0$より、\eqref{eq:pdifgtur}の最右辺は
        $0$以下である。再び\cite{MR544879}~より、
        \ref{eq:prob_sec}の解
        $v = v(\lvert x \rvert) \in C^{2+\alpha}(\bar{\Omega})$は
        球対称解に限る。
  \item \ref{eq:prob_sec}を極座標形式に書き換える。2.~より、
        \ref{eq:prob_sec}の解は球対称解しかないことから、
        動径$r$以外の微分演算子の寄与はない。$v$は$0 \in \Omega$
        において微分可能であることとディリクレ境界条件も考慮し、
        \eqref{eq:prob_rad}が得られる。\qedhere
 \end{enumerate}
\end{proof}

ポホザエフ\cite{MR0192184}~式の議論から、
\eqref{eq:prob_rad}の解がみたすべき等式を導出する。

\begin{lem}
 \eqref{eq:prob_rad}の解$v = v(r) \in C^2([0, R])$は、以下の
 等式をみたす。
 \begin{multline}
  \int_0^R r^{N-1} \left( \frac{2N}{N-2} G(v, \underline{u}_\lambda,
  r) - g(v, \underline{u}_\lambda, r) v \right) dr
  + \frac{2}{N-2} \int_0^R r^N \left( G_s(v, \underline{u}_\lambda, r)
  (\underline{u}_\lambda)_r + G_r(v, \underline{u}_\lambda, r)
  \right) dr \\ = \frac{1}{N-2} R^N v_r(R)^2. \label{eq:poh_eq}
 \end{multline}
 ここで$G_s, G_r$は、以下の偏導関数である。
 \begin{align*}
  G_s(t, s, r) &= \pdif{}{s} G(t, s, r), \\
  G_r(t, s, r) &= \pdif{}{r} G(t, s, r).
 \end{align*}
\end{lem}

\begin{proof}
 \eqref{eq:prob_rad}の方程式に$r^{N-1}$をかけて変形すると、
 次式が得られる。
 \begin{equation}
  (r^{N-1} v_r)_r + r^{N-1} g(v, \underline{u}_\lambda, r) = 0
   \label{eq:rad_basic} 
 \end{equation}
 \eqref{eq:rad_basic}に$v$をかけて$[0, R]$上積分すると、
 部分積分により次式が得られる。
 \begin{equation}
  - \int_0^R r^{N-1} (v_r)^2 dr + \int_0^R r^{N-1} g(v,
   \underline{u}_\lambda, r) v dr = 0.  \label{eq:poh_times_v}
 \end{equation}
 \eqref{eq:rad_basic}に$rv_r$をかけて$[0, R]$上積分する。
 まず\eqref{eq:rad_basic}の第$1$項を計算する。
 部分積分により、順次以下の計算が進む。
 \begin{align*}
  \int_0^R r v_r (r^{N-1} v_r)_r dr &= R^N v_r(R)^2 - \int_0^R
  (rv_r)_r r^{N-1} v_r dr, \\
  \int_0^R (rv_r)_r r^{N-1} v_r dr &= \int_0^R r^{N-1} (v_r)^2 dr +
  \int_0^R r^N v_r v_{rr} dr, \\
  \int_0^R r^N v_r v_{rr} dr &= \int_0^R r^N \left( \frac{1}{2}
  (v_r)^2 \right)^\prime dr = \frac{1}{2} R^N v_r(R)^2 - \frac{N}{2}
  \int_0^R r^{N-1} (v_r)^2 dr.
 \end{align*}
 以上より、
 \begin{align}
  \int_0^R r v_r (r^{N-1} v_r)_r dr &= R^N v_r(R)^2 - 
  \int_0^R r^{N-1} (v_r)^2 dr - \frac{1}{2} R^N v_r(R)^2 + \frac{N}{2}
  \int_0^R r^{N-1} (v_r)^2 dr \notag \\ 
  &= \frac{1}{2} R^N v_r(R)^2 + \frac{N-2}{2}\int_0^R r^{N-1} (v_r)^2
  dr \label{eq:poh_times_rvr_1}
 \end{align}
 と変形される。次に、第$2$項の計算をする。以下の式に注意する。
 \begin{align*}
  \frac{d}{dr} \left( r^N G(v(r), \underline{u}_\lambda(r), r )
  \right) &= N r^{N-1} G( v(r), \underline{u}_\lambda(r), r) + g(
  v(r), \underline{u}_\lambda(r), r) + G_s(v(r),
  \underline{u}_\lambda(r), r) + G_r(v(r), \underline{u}_\lambda(r),
  r), \\
  R^N G(v(R), \underline{u}_\lambda(R), R) &= R^N G(0, 0, R) = 0. 
 \end{align*}
 やはり部分積分により、以下の通り計算がなされる。
 \begin{equation}
  \int_0^R r^N g(v, \underline{u}_\lambda, r) v_r dr
   = -N \int_0^R r^{N-1} G(v, \underline{u}_\lambda, r)
   (\underline{u}_\lambda)_r dr - \int_0^R r^N G_r(v,
   \underline{u}_\lambda, r) dr. \label{eq:poh_times_rvr_2}
 \end{equation}
 \eqref{eq:poh_times_rvr_1}、\eqref{eq:poh_times_rvr_2}より、
 次式が得られる。
 \begin{equation}
  \frac{1}{2} R^N v_r(R)^2 + \frac{N-2}{2}\int_0^R r^{N-1} (v_r)^2
   dr -N \int_0^R r^{N-1} G(v, \underline{u}_\lambda, r)
   (\underline{u}_\lambda)_r dr - \int_0^R r^N G_r(v,
   \underline{u}_\lambda, r) dr = 0. \label{eq:poh_times_rvr}
 \end{equation}
 $\text{\eqref{eq:poh_times_v}} + \text{\eqref{eq:poh_times_rvr}} \times
 2/(N-2)$より、\eqref{eq:poh_eq}が証明される。\qedhere
\end{proof}

\eqref{eq:poh_eq}から、不等式を導出する。

\begin{lem}
 \eqref{eq:prob_rad}の解$v = v(r) \in C^2([0, R])$は、以下の
 不等式をみたす。
 \begin{equation}
  \int_0^R r^{N-1} \left( p(2b(r) + b^\prime(r) r)
                    \underline{u}_\lambda^{p-1} 
                    - (2a(r) + a^\prime(r) r) \right) v^2 dr \geq 0.
  \label{eq:poh_ineq}
 \end{equation}
\end{lem}

\begin{proof}
 まず、\eqref{eq:poh_eq}の右辺について
 \begin{equation}
  \frac{1}{N-2} R^N v_r(R)^2 \geq 0 \label{eq:poh_ineq_0}
 \end{equation}
 である。
 次に、$G_s$を調べる。$t, s \geq 0$、$0 \leq r \leq R$に対し、
 \[
  G_s(t, s, r) = \pdif{}{s} \left( b(r) \left( \frac{1}{p+1}
 (t+s)^{p+1} - \frac{1}{p+1}s^{p+1} - s^pt \right)
 - \frac{1}{2}a(r) t^2 \right) = b(r) \left( (t+s)^p - s^p -p s^{p-1}
 t \right)
 \]
 である。テイラーの定理より、$0 < \theta < 1$を用いて、
 \[
  (t+s)^p - s^p - ps^{p-1} t = \frac{p(p-1)}{2} (s + \theta t)^{p-2} t^2
 \]
 と書ける。これは$0$以上である。
 よって、\eqref{eq:under_u_r_prime}も考慮すると、次式が得られる。
 \begin{equation}
  \int_0^R r^N G_s(v, \underline{u}_\lambda, r)
   (\underline{u}_\lambda)_r dr \leq 0. \label{eq:poh_ineq_1}
 \end{equation}
 続けて、$G_r$を調べる。$t, s \geq 0$、$0 \leq r \leq R$に対し、
 \[
  G_r(t, s, r) = b^\prime(r) \left( \frac{1}{p+1} (t+s)^{p+1} -
 \frac{1}{p+1} s^{p+1} - s^{p} t \right) - \frac{1}{2} s^\prime(r) t^2
 \]
 である。テイラーの定理より、
 \[
  \frac{1}{p+1} (t+s)^{p+1} - \frac{1}{p+1} s^{p+1} -s^p t -
 \frac{p}{2} s^{p-1} t^2 = \frac{p(p-1)}{6} (s + \theta t)^{p-2} t^3
 \]
 をみたす$0 < \theta < 1$が存在する。右辺は$0$以上であるから、
 $b^\prime(r) \leq 0$も合わせて、
 \begin{equation}
  \int_0^R r^N G_r(v, \underline{u}_\lambda, r) dr \leq \int_0^R
   \left( b^\prime \frac{p}{2} \underline{u}_\lambda^{p-1} v^2 -
    \frac{1}{2} s^\prime v^2 \right)dr = 
   \left( b^\prime \frac{p}{2} \underline{u}_\lambda^{p-1} -
    \frac{1}{2} s^\prime \right) v^2 dr \label{eq:poh_ineq_2} 
 \end{equation}
 と評価できる。最後に、
 \begin{align*}
  \frac{2N}{N-2}G(t, s, r) - g(t, s, r)t 
  &= \left( b \left( (t+s)^{p+1} - s^{p+1} - (p+1)s^p t - \right)
  \frac{p+1}{2} a(r) t^2 \right) - \left( b(r) \left( (t+s)^p - s^p
  \right) - a(r) t \right) t \\
  &= b(r) \left( (t+s)^{p+1} - (t+s)^p t - s^{p+1} - (p+1)s^p t + s^p
  t \right) -\frac{p-1}{2} a(r) t^2
 \end{align*}
 を考察する。$\alpha(t) = (t+s)^{p+1} - (t+s)^p t = (t+s)^p s$とおく。
 $1, 2, 3$階導関数を計算すると、それぞれ以下の通りである。
 \begin{align*}
  \alpha^\prime(t) &= p(t+s)^{p-1}s, \\
  \alpha^{\prime\prime}(t) &= p(p-1)(t+s)^{p-2}s, \\
  \alpha^{\prime\prime\prime}(t) &= p(p-1)(p-2)(t+s)^{p-3}s.
 \end{align*}
 $\alpha$の$t = 0$のまわりのテイラー多項式を考える。
 $3$次の剰余項を$R_3$とすると、
 \[
  R_3 = (t+s)^{p+1} - (t+s)^p t -s^{p+1} -(p+1)s^p t - s^p t -
 \frac{1}{2}p(p-1) s^{p-1}t^2
 \]
 である。テイラーの定理より、$R_3$は$0 < \theta < 1$を用いて
 \[
  R_3 = \frac{p(p-1)(p-2)}{6}(s + \theta t)^{p-3} st^3
 \]
 と表される。$N \geq 6$であるから、$1 < p \leq 2$である。
 ゆえに、$R_3 \leq 0$とわかる。以上より、
 \begin{equation}
  \int_0^R r^{N-1} \left( \frac{2N}{N-2} G(v, \underline{u}_\lambda,
           r) - g(v, \underline{u}_\lambda, r) v \right) dr \leq \int_0^R
  r^{N-1} \left( b \frac{1}{2} p(p-1) \underline{u}_\lambda^{p-1} -
 \frac{p-1}{2} a  \right) v^2 dr \label{eq:poh_ineq_3}
 \end{equation}
 と評価される。\eqref{eq:poh_ineq_0}、
 \eqref{eq:poh_ineq_1}、\eqref{eq:poh_ineq_2}、
 \eqref{eq:poh_ineq_3}より、\eqref{eq:poh_eq}から
 \[
  \int_0^R r^{N-1} \left( b \frac{1}{2} p(p-1)
 \underline{u}_\lambda^{p-1} - \frac{p-1}{2} a \right) v^2 dr 
 + \frac{p-1}{4} \int_0^R r^N \left( b^\prime p
 \underline{u}_\lambda^{p-1} - a^\prime \right) v^2 dr \geq 0
 \]
 が得られる。変形すると、
 \[
  \int_0^R \frac{p-1}{4}  r^{N-1} \left( p(2b + b^\prime r)
 \underline{u}_\lambda^{p-1} 
 - (2a + a^\prime r) \right) v^2 dr \geq 0.
 \]
 がわかる。$(p-1)/4 > 0$より、\eqref{eq:poh_ineq}が得られる。\qedhere
\end{proof}

\begin{proof}[定理~\ref{thm:second_solution_nonex}.2]
 
\end{proof}

% Local Variables:
% mode: yatex
% TeX-master: "main.tex"
% End: