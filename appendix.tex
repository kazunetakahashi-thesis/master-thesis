%#!platex main.tex

\section{命題~\ref{prop:exist}で使用される不等式の証明}

本付録では、命題~\ref{prop:exist}で使用される不等式の証明を与える。

\begin{lem} \label{lem:ineq_naito_sato}
 任意の$0 < \epsilon < 1$、$s, t \geq 0$に対し、
 次式が成立する。
 \begin{equation}
  \frac{p}{2}s^{p-1}t^2 - 
   \left( \frac{1}{p+1}(t+s)^{p+1} -
    \frac{1}{p+1}s^{p+1} - s^pt - \frac{1}{p+1}t^{p+1}
   \right) \leq \frac{\epsilon p}{2} s^{p-1}
   t^2 + \frac{1-\epsilon}{p+1} t^{p+1}. \label{eq:ineq_naito_sato}
 \end{equation}
\end{lem}
 
\begin{proof}
 \eqref{eq:ineq_naito_sato}の左辺を$Y(t, s)$とおく。
 $Y$の$t$についての$2$階偏導関数は、
 \begin{align*}
  Y_{tt}(t, s) &= p s^{p-1} - p(t+s)^{p-1} + pt^{p-1} \\
  &= p \left( (1- \epsilon) t^{p-1} + \epsilon t^{p-1} + s^{p-1} -
  (t+s)^{p-1} \right)
 \end{align*}
 である。$p - 1 > 0$より、以下の計算が進む。
 \begin{align*}
  \epsilon t^{p-1} + s^{p-1} - (s+t)^{p-1} &= 
  \epsilon t^{p-1} + s^{p-1} - (1-\epsilon)(t+s)^{p-1} - \epsilon
  (t+s)^{p-1} \\
  & \leq \epsilon t^{p-1} + s^{p-1} - (1-\epsilon) s^{p-1} - \epsilon
  t^{p-1} = \epsilon s^{p-1}.
 \end{align*}
 ゆえに、$Y_tt$は、
 \[
  Y_{tt}(t, s) \leq p \left( (1-\epsilon) t^{p-1} + \epsilon s^{p-1} \right)
 \]
 と評価される。$Y_t(0, s) = Y(0, s) = 0$であるから、以下が成り立つ。
 \begin{align*}
  Y_t(t, s) &\leq \int_0^t  p \left( (1-\epsilon) t^{p-1} + \epsilon
  s^{p-1} \right) dt = (1 - \epsilon) t^p + \epsilon p s^{p-1} t, \\
  Y(t, s) &\leq \int_0^t \left( (1 - \epsilon) t^p + \epsilon p
  s^{p-1} t \right)dt = \frac{1 -\epsilon}{p+1} t^{p+1} +
  \frac{\epsilon p}{2}s^{p-1}t^2.
 \end{align*}
 以上より、\eqref{eq:ineq_naito_sato}が成立する。\qedhere
\end{proof}

補題~\ref{lem:ineq_naito_sato}は、\cite{MR2886160}~のLemma~B.2 (iii)
とほぼ同一である。またその証明は、以上の証明と同一である。
それにもかかわらず、本付録で証明を改めて書いたのは、
\cite{MR2886160}~のLemma~B.2 (iii)では$1 < p \leq 2$という仮定が
ついているからである。この仮定がなくとも
補題~\ref{lem:ineq_naito_sato}は成立する。

% Local Variables:
% mode: yatex
% TeX-master: "main.tex"
% End: