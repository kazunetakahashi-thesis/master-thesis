%master-thesis
%#BIBTEX pbibtex main
\documentclass{jsarticle}

\title{Semilinear elliptic equations with a critical Sobolev exponent and a non-homogeneous term}
\author{Kazune Takahashi}
\date{24 January 2015}

\usepackage{geometry}
\geometry{a4paper,left=15mm,right=15mm,top=20mm,bottom=15mm}
\addtolength{\textheight}{-10mm}

\usepackage{relsize}
\usepackage{url}
\usepackage{enumerate}
\addtolength{\labelsep}{0mm}
\addtolength{\listparindent}{0mm}

\usepackage{mathtools}
\usepackage{cases}

\usepackage{fancyhdr}
\pagestyle{fancy}
\lhead{}
\chead{}
\rhead{\rightmark}
\lfoot{}
\cfoot{{\textit{\thepage}}}
\rfoot{}
\fancypagestyle{plainhead}{
\lhead{\leftmark}
\chead{}
\rhead{}
}
\fancypagestyle{plainfoot}{
\lfoot{}
\cfoot{{\textit{\thepage}}}
\rfoot{}
}
\renewcommand{\headrulewidth}{0pt}
\renewcommand{\footrulewidth}{0pt}
\renewcommand{\sectionmark}[1]{\markright{\S\thesection~~#1}{}}

\usepackage{amsmath,amssymb}
\usepackage[amsmath,framed,thmmarks]{ntheorem}
\allowdisplaybreaks[1]

\theoremstyle{plain}
\theoremseparator{.}
\theorembodyfont{\upshape}
\theoremprework{}
\theorempostwork{}
\newtheorem{thm}{定理}[section]

\theoremstyle{plain}
\theoremseparator{.}
\theorembodyfont{\upshape}
\theoremprework{}
\theorempostwork{}
\newtheorem{prop}[thm]{命題}

\theoremstyle{plain}
\theoremseparator{.}
\theorembodyfont{\upshape}
\theoremprework{}
\theorempostwork{}
\newtheorem{lem}[thm]{補題}

\theoremstyle{plain}
\theoremseparator{.}
\theorembodyfont{\upshape}
\theoremprework{}
\theorempostwork{}
\newtheorem{defn}[thm]{定義}

\theoremstyle{plain}
\theoremseparator{.}
\theorembodyfont{\upshape}
\theoremprework{}
\theorempostwork{}
\newtheorem{rem}[thm]{注意}

\newcommand{\qedhere}{\qedsymbol{\rule{1ex}{1ex}} \qed}

\theoremstyle{nonumberplain}
\theorembodyfont{\upshape}
\theoremsymbol{\rule{1ex}{1ex}}
\newtheorem{proof}{証明}

\usepackage{comment}

\newcommand{\pdif}[2]{\frac{\partial #1}{\partial #2}}

\renewcommand{\hat}[1]{\widehat{#1}}
\renewcommand{\tilde}[1]{\widetilde{#1}}
\renewcommand{\bar}[1]{\overline{#1}}
\renewcommand{\vec}[1]{\overrightarrow{#1}}

\newcommand{\N}{\mathbb{N}}
\newcommand{\R}{\mathbb{R}}
\newcommand{\Z}{\mathbb{Z}}
\newcommand{\C}{\mathbb{C}}
\newcommand{\Q}{\mathbb{Q}}

\renewcommand{\ae}{\text{a.\,e.}~}
\newcommand{\tin}{\text{in}~}
\newcommand{\ton}{\text{on}~}
\newcommand{\supp}{\operatorname{supp}} 

\renewcommand{\bibname}{参考文献}

\setcounter{tocdepth}{4}
\begin{document}

\maketitle

%#!platex main.tex
\section{概要}

$N$を$3$以上の自然数とする。$\Omega \subset \R^N$を有界領域とする。
$p = (N+2)/(N-2)$とする。$f \in H^{-1}(\Omega)$は、$f \geq 0$、
$f \not \equiv 0$をみたすとする。
$a, b \in L^\infty(\Omega)$とする。
$\kappa_1$を$-\Delta$の$\Omega$におけるディリクレ条件下での
第$1$固有値とする。$\kappa > - \kappa_1$が存在して、$a \geq \kappa$
となると仮定する。また、$b \geq 0$、$b \not \equiv 0$と仮定する。
$\lambda \geq 0$をパラメータとする。以下の方程式を考察する。
\begin{align}
 \left\{
 \begin{aligned}
  -\Delta u + a u &= b u^p + \lambda f  & &\text{in~} \Omega,  \\
  u &> 0 & &\text{in~} \Omega, \\
  u &= 0 & &\text{on~} \partial\Omega
 \end{aligned}
 \right. \tag*{$(\spadesuit)_\lambda$} \label{eq:prob_main}
\end{align}

\begin{thm} \label{thm:minimal_solution}
 \ref{eq:prob_main} には minimal solution が存在する。
\end{thm}

\begin{thm} \label{thm:extremal_solution}
 \ref{eq:prob_main} には extremal solution が存在する。
 とくに、$\lambda = \bar{\lambda}$における
 \ref{eq:prob_main} の
 minimal solution が存在する。
 また、$b > 0 ~\tin \Omega$ならば、\ref{eq:prob_main} の 
 extremal solution は、$\lambda = \bar{\lambda}$における
 \ref{eq:prob_main} の
 minimal solution に限る。
\end{thm}

\begin{thm} \label{thm:second_solution}
 $0 < \lambda < \bar{\lambda}$とする。$b$は$\Omega$上の
 ある点$p$で最大値$M_1 = \left\| b \right\|_{L^\infty}(\Omega) > 0$を
 達成するものと仮定する。$r_0 > 0$が存在し、
 $\{ \lvert x - p \rvert < 2r_0 \} \subset \Omega$、かつ、
 $\{ \lvert x - p \rvert < r_0 \}$上
 \begin{align*}
  b(x) &= M_1 - M_2 \lvert x-p \rvert^q,  \\
  a(x) &= m_1 + m_2 \lvert x-p \rvert^{q^\prime}
 \end{align*}
 であると仮定する。ここで$q, q^\prime > 0$、
 $M_2 > 0$、$m_1 > \kappa$、$m_2 \neq 0$は
 定数である。さらに、以下の(i) -- (iv)の
 いずれかの成立を仮定する。
 \begin{enumerate}[(i)]
  \item $m_1 < 0$、かつ、$N \geq 3$。
  \item $m_1 > 0$、かつ、$N = 3, 4, 5$。
  \item $m_1 = 0$、かつ、$m_2 < 0$、かつ、$N \geq 3$。
  \item $m_1 = 0$、かつ、$m_2 > 0$、かつ、$3 \leq N < 6 + 2q^\prime$。
 \end{enumerate}
 このとき、\ref{eq:prob_main}は、minimal solution
 $\underline{u}_\lambda$
 以外の弱解$\bar{u}_\lambda \in H_0^1(\Omega)$をもつ。
\end{thm}

\ref{eq:prob_main}の minimal solution 以外の解
$\bar{u}_\lambda$を見出すために、
以下の方程式\ref{eq:prob_sec}を考察する。
\begin{align}
 \left\{
 \begin{aligned}
   -\Delta v + a v &= b \left( (v + \underline{u}_\lambda)^p -
  (\underline{u}_\lambda)^p \right) 
  & &\text{in~} \Omega, \\
  v &> 0 & &\text{in~} \Omega, \\
  v &= 0 & &\text{on~} \partial\Omega
 \end{aligned}
 \right. \tag*{$(\heartsuit)_\lambda$} \label{eq:prob_sec}
\end{align}

\subsection{記号}

ルベーグ空間を$L^q(\Omega)$ ($1 \leq q \leq \infty$)と表記する。
ソボレフ空間$W^{1, 2}(\Omega)$を$H^1(\Omega)$と表記する。
トレースの意味で$u |_{\partial \Omega} = 0$が成立
する$u \in H^1(\Omega)$全体を$H_0^1(\Omega)$と表記する。
ヘルダー空間を$C^{k + \alpha}(\Omega)$ ($k \in \N$、$0 < \alpha < 1$)
と表記する。
コンパクト台を持つ$\Omega$上の$C^\infty$級関数全体を
$C^\infty_c (\Omega)$と表記する。

ノルム空間$X$のノルムを$\dnorm_X$と表記する。
ノルム空間$X$の双対空間を$X^*$と表記する。
$H_0^1(\Omega)^*$を$H^{-1}(\Omega)$と表記する。
$f \in H^{-1}$の$u \in H_0^1(\Omega)$への作用を$\langle f, u \rangle$
と表記する。
$H_0^1(\Omega)$上のノルム$\dnorm_{\kappa}$を、$w \in H_0^1(\Omega)$に対し、
\[
 \left\| w \right\|_\kappa = \left(\int_\Omega \left( \lvert Dw \rvert^2 +
 \kappa w ^2 \right) dx\right)^{1/2}
\]
と定める。$\kappa > -\kappa_1$、$\Omega$が有界領域で
あることにより、ポアンカレの不等式から
$\dnorm_\kappa$は$\dnorm_{H_0^1(\Omega)}$
と同値なノルムである。また、
$H_0^1(\Omega)$上のノルム$\dnorm$を、$w \in H_0^1(\Omega)$に対し、
\[
 \left\| w \right\| = \left(\int_\Omega \lvert Dw \rvert^2 dx\right)^{1/2}
\]
と定める。やはりポアンカレの不等式から
$\dnorm$は$\dnorm_{H_0^1(\Omega)}$
と同値なノルムであることがしたがう。

% Local Variables:
% mode: yatex
% TeX-master: "main.tex"
% End:

%#!platex main.tex
\section{minimal solutionの存在と性質} \label{sec:minimal_sol}

本節では、\ref{eq:prob_main}の解のうち、
minimal solution について取り扱う。
まずは minimal solution を定義する。

\begin{nota}
 $\lambda > 0$に対し、
 \[
 S_\lambda = \{ u \in H_0^1(\Omega) \mid u \text{は
 \ref{eq:prob_main} の弱解である}\}
 \]
 と定める。
\end{nota}
\begin{defn}
 $\underline{u}_\lambda \in S_\lambda$が{\bf minimal solution} で
 あるとは、任意の$u \in S_\lambda$に対し、
 $
  \underline{u}_\lambda \leq u ~\tin \Omega
 $
 が成立することをいう。
\end{defn}
\begin{nota}
 \ref{eq:prob_main}の minimal solution を $\underline{u}_\lambda$ と表記する。
\end{nota}

\subsection{$H_0^1(\Omega)$の原点付近における様子}

minimal solution を調べる第一歩として、
$\lambda > 0$が十分小さいときに、\ref{eq:prob_main}が
弱解を持つことを、陰関数定理を用いて示す。

\begin{lem} \label{lem:imp}
 \begin{enumerate}[1.] \sage
  \item $\lambda_0 > 0$と$H_0^1(\Omega)$の原点の近傍$U$
        が存在して、$0 < \lambda \leq \lambda_0$に対し、
        \ref{eq:prob_main}は$U$内の唯一の弱解
        $u_\lambda$をもつ。また、次が成立する。
        \[
        \left\| u_\lambda
        \right\|_{H^1_0(\Omega)} \to 0 \ \ (\lambda \searrow 0).
        \]
  \item さらに、$f \in C^\alpha(\bar{\Omega})$を仮定する。
        このとき、1.~の弱解$u_\lambda$は、$u_\lambda \in
        C^{2+\alpha}(\Omega)$を
        みたし、次が成立する。
        \[
        \left\| u_\lambda
        \right\|_{C^{2+\alpha}(\Omega)} \to 0 \ \ 
        (\lambda \searrow 0).
        \]
 \end{enumerate}
\end{lem}

\begin{proof}
 \begin{enumerate}[1.] \sage
  \item $\Phi \colon [0,\infty) \times H^1_0 (\Omega) \to H^{-1}(\Omega)$を
        \begin{equation}
         \Phi (\lambda, u) = -\Delta u + au - b (u_{+})^p - \lambda f
          \label{eq:def_of_Phi}
        \end{equation}
        とする。$\Phi$の$u$についてのフレッシェ微分は、
        $w \in H^1_0(\Omega)$に対し、
        \begin{equation}
         \Phi_u (\lambda, u) \colon w \mapsto -\Delta w + aw - b
          p(u_+)^{p-1} w.
          \label{eq:Phi_dr}
        \end{equation}
        と書かれる。
        特に、
        \[
         \Phi_u (0, 0) w = -\Delta w + aw.
        \]
        が成立する。
        $a > -\kappa_1$により、$\Phi_u(0,0) \colon
        H^1_0(\Omega) \to H^{-1} (\Omega)$は可逆である。ゆえに、
        陰関数定理より、$\lambda_0 > 0$と
        $H_0^1(\Omega)$の原点の近傍$U$が存在して、
        $0 < \lambda \leq \lambda_0$に対し、
        $\Phi(\lambda, u_\lambda) = 0$をみたす
        $u_\lambda \in U$が
        唯一つ存在し、次をみたす。
        \[\lim_{\lambda \searrow 0} \left\| u_\lambda
        \right\|_{H^1_0(\Omega)} 
        = 0. \]
        つまり、$u_\lambda$は、以下の方程式の弱解である。
        \begin{align}
         \left\{
         \begin{aligned}
          -\Delta u + a u &= b (u_+)^p + \lambda f  & &\tin \Omega,  \\
          u &= 0 & &\ton \partial\Omega.
         \end{aligned}
         \right. \label{eq:prob_lem_ifthm}
        \end{align}
        ここで$b (u_+)^p + \lambda f \geq 0$であり、$a > -\kappa_1$で
        あるから、強最大値原理により、$u_\lambda > 0 ~\tin
        \Omega$が成立する。
        よって、$u_\lambda$は\ref{eq:prob_main}の
        $U$における唯一の弱解である。
  \item $f \in C^\alpha(\bar{\Omega})$のとき、
        $\Phi \colon [0,\infty) \times C^{2+\alpha} (\bar{\Omega})
        \to C^\alpha(\bar{\Omega})$を、\eqref{eq:def_of_Phi}で定義する。
        以下、1.~の証明と同様にすると、
        $u_\lambda \in
        C^{2+\alpha}(\Omega)$と
        $\left\| u_\lambda
        \right\|_{C^{2+\alpha}(\Omega)} \to 0 \ \ (\lambda \searrow
        0)$
        が示される。\qedhere
 \end{enumerate}
\end{proof}

以下では基本的に、1.~の結果を使用し、弱解の枠組みで議論する。
2.~の結果は、\S~\ref{sec:sym}で使用する。

\subsection{優解との関係}

続いて、ある
$\lambda = \hat{\lambda}$で\ref{eq:prob_main}が
優解をもつときに、$0 < \lambda \leq \hat{\lambda}$で
minimal solution が存在することを示す。

\begin{lem} \label{lem:minimal_itt}
 $\hat{\lambda} > 0$とする。
 以下をみたす$\hat{u} \in H_0^1(\Omega)$が存在すると仮定する。
\begin{align}
 \left\{
 \begin{aligned}
  \Delta \hat{u} + a\hat{u} &\geq b \hat{u}^p + \hat{\lambda}f  &
  &\tin \Omega,  \\
  \hat{u} &> 0 & &\tin \Omega
 \end{aligned}
 \right. \label{eq:prob_lem_ifthm}
\end{align}
 このとき、$\lambda \in (0,\hat{\lambda} ]$
 に対し、\ref{eq:prob_main}の minimal solution $\underline{u}_\lambda$
 が存在する。
 また、$\underline{u}_\lambda < \hat{u} ~\tin \Omega$
 が成立する。
\end{lem}

\begin{proof}
 $H_0^1(\Omega)$の点列$\{ u_n \}_{n=0}^\infty$を、次の通りに
 帰納的に定める。$u_0 \equiv 0$とする。$u_n$が
 定まっているときに、線形方程式
  \begin{align}
   \left\{
   \begin{aligned}
    -\Delta u_{n+1} + a u_{n+1} 
    &= b u_n^p + \lambda f  & &\tin \Omega,  \\
    u_{n+1} &= 0 & &\ton \partial\Omega
   \end{aligned}
   \right. \label{eq:u_n+1}
  \end{align}
 の唯一の弱解を$u_{n+1} \in H_0^1(\Omega)$と定める。
 
 \eqref{eq:u_n+1}が唯一の弱解であることを確かめる。
 ソボレフ埋め込みにより、
 $
  u_n \in H_0^1(\Omega) \subset L^{p+1}(\Omega)
 $
 だから、
 $
  u_n^p \subset L^{(p+1)/p}(\Omega)
 = L^{2N/(N+2)}(\Omega) \subset H^{-1}(\Omega)
 $
 である。$b \in L^\infty(\Omega)$、$f \in H^{-1}(\Omega)$より、
 $b u_n^p + \lambda f \in H^{-1}(\Omega)$である。
 $a > -\kappa_1$と合わせて、\eqref{eq:u_n+1}には
 唯一の弱解が存在する。

 ここで、次の事実を、$n$についての数学的帰納法を用いて証明する。
 \begin{equation}
  0 = u_0 < u_1 < \dots < u_n < \hat{u} ~\tin \Omega. \label{eq:u_n_ind}
 \end{equation}

 $n = 0$のときは、$\hat{u} > 0 ~\tin \Omega$であることから、
 \eqref{eq:u_n_ind}が成立する。
 $n \in \N$とする。$n$における\eqref{eq:u_n_ind}の成立を仮定し、
 $n+1$における\eqref{eq:u_n_ind}の成立を示す。
 \begin{align*}
  -\Delta u_{n+1} + a u_{n+1} &= b u_n^p + \lambda f, \\
  -\Delta u_{n} + a u_{n} &= b u_{n-1}^p + \lambda f
 \end{align*}
 の両辺を引くと、次が成立する。
 \[
  -\Delta (u_{n+1} - u_n) + a(u_{n+1} - u_n) 
 = b(u_n^p - u_{n-1}^p).
 \]
 右辺は仮定により$0$以上である。
 ゆえに強最大値原理より、$u_{n+1} > u_n ~\tin\Omega$である。
 また、
 \begin{align*}
  -\Delta \hat{u} + a \hat{u} &> b \hat{u}^p + \lambda f, \\
  -\Delta u_{n+1} + a u_{n+1} &= b u_{n}^p + \lambda f
 \end{align*}
 の両辺を引いて同様にすると、$\hat{u} > u_{n+1} ~\tin\Omega$
 もしたがう。以上により、\eqref{eq:u_n_ind}は$n+1$でも正しい。
 数学的帰納法により、任意の$n \in \N$について
 \eqref{eq:u_n_ind}の成立が示された。

 続いて、$\{u_n\}$が$H_0^1(\Omega)$における
 有界列であることを示す。
 $u_{n+1}$は\eqref{eq:u_n+1}の弱解であるから、
 任意の$\psi \in H_0^1(\Omega)$
 に対し、次が成立する。
 \begin{equation}
  \int_\Omega (Du_{n+1} \cdot D\psi + a u_{n+1} \psi) dx 
   = \int_\Omega bu_n^p \psi dx + \lambda \int_\Omega f\psi dx
   \label{eq:u_n+1_weaksol}
 \end{equation}
 $\psi = u_{n+1}$とすると、次が成立する。
 \[
 \int_\Omega (\lvert Du_{n+1} \rvert^2 
 + a \lvert u_{n+1} \rvert^2) dx 
 = \int_\Omega bu_n^p u_{n+1} dx 
 + \lambda \int_\Omega f u_{n+1} dx.
 \]
 ここで、右辺は、次の通りに評価される。
 \begin{equation}
  (\text{右辺}) \leq \int_\Omega b\hat{u}^{p+1} dx + \lambda
   \int_\Omega f \hat{u} dx < \infty. \label{eq:u_n+1_right}
 \end{equation}
 ここで$\hat{u} \in H_0^1(\Omega) \subset L^{p+1}(\Omega)$に
 注意した。また左辺について、
 \begin{equation}
  (\text{左辺}) \geq \int_\Omega \left( \lvert Du_{n+1} \rvert^2 + \kappa
   \lvert u_{n+1} \rvert^2 \right) dx = \| u_{n+1} \|_{H_0^1(\Omega)}
  \label{eq:u_n+1_left}
 \end{equation}
 もわかる。$\dnorm_\kappa$は$\dnorm_{H_0^1(\Omega)}$
 と同値なノルムである。したがって、\eqref{eq:u_n+1_right}および
 \eqref{eq:u_n+1_left}より、
 $\{u_n\}$は$H_0^1(\Omega)$の有界列である。

 ゆえに、必要ならば部分列をとることにより、
 $u \in H_0^1(\Omega)$が存在して、$n \to \infty$とすると、
 以下が成立する。
 \begin{align}
  u_n \xrightharpoonup{ \mbox{ ~ } } u & \ \ \text{weakly~} \tin
  H_0^1(\Omega), \label{eq:minimal_u_n_weakly} \\
  u_n \xrightarrow{ \mbox{ ~ } } u & \ \ \tin L^q(\Omega) \notag \ \
   (q < p+1), \\
  u_n \xrightarrow{ \mbox{ ~ } } u & \ \ \ae \tin \Omega. 
    \label{eq:minimal_u_n_ae}
 \end{align}
 ここで$u$が\ref{eq:prob_main}の弱解であることを示す。
 \eqref{eq:minimal_u_n_weakly}により、次が成立する。
 \[
 \int_\Omega (Du_{n+1} \cdot D\psi + a u_{n+1} \psi) dx
 \xrightarrow{n \to \infty}
 \int_\Omega (Du \cdot D\psi + a u \psi) dx.
 \]
 また、$b \in L^\infty(\Omega)$、
 $\hat{u}, \psi \in H_0^1(\Omega) \subset L^{p+1}(\Omega)$より、
 \[
  \left\lvert bu_n \psi \right\rvert \leq b \hat{u}^p \lvert\psi\rvert \ \ \ae
 \tin \Omega
 \]
 の右辺は可積分である。\eqref{eq:minimal_u_n_ae}より、
 優収束定理から、次を得る。
 \[
 \int_\Omega bu_n^p \psi dx \xrightarrow{n \to \infty} 
 \int_\Omega bu^p \psi dx.
 \]
 したがって、\eqref{eq:u_n+1_weaksol}で$n \to \infty$とすると次を得る。
 \begin{equation}
  \int_\Omega (Du \cdot D\psi + a u \psi) dx 
   = \int_\Omega bu^p \psi dx + \lambda \int_\Omega f\psi dx.
   \label{eq:minimal_u_weaksol}
 \end{equation}
 $\psi \in H_0^1(\Omega)$は任意であるから、
 $u \in H_0^1(\Omega)$は\ref{eq:prob_main}の弱解である。
 
 最後に、$u$は
 \ref{eq:prob_main}のminimal solutionであることを示す。
 $\tilde{u} \in H_0^1(\Omega)$を\ref{eq:prob_main}の弱解とする。
 このとき、\eqref{eq:u_n_ind}と同様の議論により、
 $\tilde{u} > u_n ~\tin \Omega$が数学的帰納法で示される。
 $n \to \infty$として、$\tilde{u} \geq u ~\tin \Omega$となる。
 よって$u$は\ref{eq:prob_main}のminimal solutionである。\qedhere
\end{proof}

補題~\ref{lem:minimal_itt}から、次の事実がしたがう。

\begin{lem} \label{lem:minimal_va}
 \begin{enumerate}[1.] \sage
  \item $\lambda_0 > 0$が存在して、$S_{\lambda_0} \neq \emptyset$とする。
        このとき、$0 < \lambda < \lambda_0$に対し、
        $S_\lambda \neq \emptyset$となる。
  \item $\lambda > 0$とする。$S_\lambda \neq \emptyset$ならば、
        \ref{eq:prob_main}に
        は minimal solution $\underline{u}_\lambda
        \in S_\lambda$が存在する。
  \item $0 < \lambda_1 < \lambda_2$とする。$S_{\lambda_1} \neq
        \emptyset$、
        $S_{\lambda_2} \neq \emptyset$ならば、
        $\underline{u}_{\lambda_1} \in S_{\lambda_1}$
        $\underline{u}_{\lambda_2} \in S_{\lambda_2}$について、
        $\underline{u}_{\lambda_1} < \underline{u}_{\lambda_2} ~\tin
        \Omega$が成立する。
  \item 補題~\ref{lem:imp}における\ref{eq:prob_main}の弱解を
        $u_\lambda$とする。このとき、$u_\lambda =
        \underline{u}_\lambda$である。
 \end{enumerate}
\end{lem}

\begin{proof}
 \begin{enumerate}[1.] \sage
  \item $u_{\lambda_0} \in S_{\lambda_0}$とする。$\hat{u} =
        u_{\lambda_0}$とし
        補題~\ref{lem:minimal_itt}を適用すると結論が得られる。
  \item $u_{\lambda} \in S_{\lambda}$とする。$\hat{u} =
        u_{\lambda}$として
        補題~\ref{lem:minimal_itt}を適用すると、
        \ref{eq:prob_main}の
        minimal solution $\underline{u}_\lambda$が得られる。
  \item $\hat{u} = \underline{u}_{\lambda_2}$として、
        補題~\ref{lem:minimal_itt}
        \eqref{eq:u_n_ind}を適用すると、
        $\underline{u}_{\lambda_1} \leq
        \underline{u}_{\lambda_2} ~\tin \Omega$が得られる。
        \begin{align*}
         -\Delta \underline{u}_{\lambda_1} + a
         \underline{u}_{\lambda_1} 
         &= b \underline{u}_{\lambda_1}^p + \lambda_1 f, \\
         -\Delta \underline{u}_{\lambda_2} + a
         \underline{u}_{\lambda_2} 
         &= b \underline{u}_{\lambda_2}^p + \lambda_2 f
        \end{align*}
        の両辺を引くと、次が成立する。
        \[
         -\Delta (\underline{u}_{\lambda_2} - \underline{u}_{\lambda_1}) + a
         (\underline{u}_{\lambda_2} - \underline{u}_{\lambda_1} )
         = b (\underline{u}_{\lambda_2}^p -
        \underline{u}_{\lambda_2}) + (\lambda_2 - \lambda_1) f.
        \]
        右辺が$0$以上であること、および、
        $a > -\kappa_1$により、強最大値原理を用いると、
        $\underline{u}_{\lambda_1} <
        \underline{u}_{\lambda_2} ~\tin \Omega$がしたがう。
  \item $u_\lambda \in S_\lambda$より、
        $S_\lambda \neq \emptyset$である。
        したがって、2.~より、\ref{eq:prob_main}は minimal solution
        $\underline{u}_\lambda$をもつ。よって、
        \eqref{eq:minimal_u_weaksol}
        で$u = \psi = \underline{u}_\lambda$とおくと、
        以下が得られる。
        \begin{equation}
         \int_\Omega \left( \lvert D\underline{u}_\lambda \rvert^2 + a
                      \lvert \underline{u}_\lambda \rvert^2 \right) dx 
          = \int_\Omega b\underline{u}_\lambda^p dx 
          + \lambda \int_\Omega f \underline{u}_\lambda dx.
          \label{eq:minimal_inp_same_weak}
        \end{equation}
        ここで、
        minimal solution の
        $H_0^1(\Omega)$ノルムが、$\lambda \searrow
        0$のとき、$0$に収束することを示す。
        \begin{align*}
         (\text{\eqref{eq:minimal_inp_same_weak}の左辺}) \geq 
         \int_\Omega \left( \lvert
         D\underline{u}_\lambda
         \rvert^2 +
         \kappa \lvert \underline{u}_\lambda \rvert^2 \right) dx 
         \geq C
         \left\| \underline{u}_\lambda \right\|_{H_0^1(\Omega)}^2.
        \end{align*}
        中辺は$\| \underline{u}_\lambda \|_\kappa^2$であり、
        $\dnorm_{\kappa}$は
        $\dnorm_{H_0^1(\Omega)}$と同値であるから、
        $C > 0$は$\dnorm_{H_0^1(\Omega)}$の
        中身によらない定数であることに
        注意されたい。
        また、
        $\underline{u}_\lambda \leq u_\lambda ~\tin \Omega$より、
        次がしたがう。
        \begin{align*}
         (\text{\eqref{eq:minimal_inp_same_weak}の右辺})
         &\leq \int_\Omega b \underline{u}_\lambda^{p+1}
         dx + \lambda \int_\Omega f \underline{u}_\lambda dx \\
         &\leq \left\| b \right\|_{L^\infty(\Omega)} \left\|
         u_\lambda \right\|_{L^{p+1}(\Omega)}^{p+1} +
         \lambda \left\| f \right\|_{H^{-1}(\Omega)} \left\|
         u_\lambda \right\|_{L^2(\Omega)} \\
         &\leq C^\prime \left\|
         u_\lambda \right\|_{H_0^1(\Omega)}^{p+1} +
         C^{\prime\prime} \left\| u_\lambda
         \right\|_{H_0^1(\Omega)}
        \end{align*}
        ここで、$C^\prime, C^{\prime\prime} > 0$は、
        $\dnorm_{H_0^1(\Omega)}$の中身によらない定数である。
        以上より、以下が成立する。
        \[
         C
        \left\| u \right\|_{H_0^1(\Omega)} \leq 
        C^\prime \left\|
        u_\lambda \right\|_{H_0^1(\Omega)}^{p+1} +
        C^{\prime\prime}
        \left\| u_\lambda \right\|_{H_0^1(\Omega)}.
        \]
        補題~\ref{lem:imp}より、$\lambda \searrow 0$のとき、
        $\left\| u_\lambda \right\|_{H_0^1(\Omega)} \searrow 0$
        が成立する。
        ゆえに、$\left\| \underline{u}_\lambda
        \right\|_{H_0^1(\Omega)}
        \searrow 0$となる。
        再び補題~\ref{lem:imp}によると、
        $\lambda > 0$が十分小さいとき、
        $u_\lambda$は\ref{eq:prob_main}
        の唯一の弱解であった。したがってこのことは
        $u_\lambda = \underline{u}_\lambda$を示している。 \qedhere
 \end{enumerate}
\end{proof}

\subsection{解が存在する$\lambda$の有界性}

補題~\ref{lem:imp}により、$\lambda > 0$が存在して、
\ref{eq:prob_main}の解が存在する。
補題~\ref{lem:minimal_va}により、
\ref{eq:prob_main}の解が存在する$\lambda$が見つかれば、
それより小さい$\lambda$については、\ref{eq:prob_main}の解が存在する。
そこで、\ref{eq:prob_main}の解が存在する$\lambda$が
どこまで大きくなるのかを調べる。そのために次の記号を置く。

\begin{nota} \label{nota:ext}
 $\bar{\lambda} = \sup \{ \lambda \geq 0 \mid S_\lambda \neq \emptyset
 \}$と定める。
\end{nota}

ここから先は、$\bar{\lambda} < \infty$を示すことを目標に議論を進める。
その準備として、$\lambda > 0$によらない$H_0^1(\Omega)$の元$g_0$を用意
する。

\begin{nota} \label{nota:g_0}
 $g_0 \in H_0^1(\Omega)$を
 \begin{align}
  \left\{
  \begin{aligned}
   -\Delta g_0 + a g_0 
    &= f  & &\tin \Omega,  \\
   g_0 &= 0 & &\ton \partial\Omega
  \end{aligned}
  \right. \label{eq:g_0}
 \end{align}
 の唯一の弱解と定める。 
\end{nota}
 
$g_0$について、次の補題を示す。

\begin{lem} \label{lem:g_0}
 固有値問題
 \[
  -\Delta \phi + a \phi = \mu b (g_0)^{p-1}\phi ~\tin \Omega, \ \
 \phi \in H_0^1(\Omega)
 \]
 の第$1$固有値を$\mu_1$とする。このとき、$\mu_1 > 0$である。
 また、$\mu_1$に付随する固有関数$\phi_1$のうち、
 $\phi_1 > 0 ~\tin \Omega$を
 みたすものがある。
\end{lem}

\begin{proof}
 $\mu_1$はレーリッヒ商により、
 \begin{equation}
  \mu_1 = \inf_{\psi \in H_0^1(\Omega), \psi \not\equiv 0}
   \frac{\displaystyle \int_\Omega 
   \left( \lvert D \psi \rvert^2 + a
    \lvert \psi \rvert^2
   \right) dx}{\displaystyle \int_\Omega b (g_0)^{p-1} \psi^2 dx}
   \label{eq:g_0_lin_mu_1}
 \end{equation}
 と特徴付けられる。また、\eqref{eq:g_0_lin_mu_1}の右辺の下限を
 達成する関数$\phi \in H_0^1(\Omega)$があるとすれば、$\phi$が$\mu_1$に
 付随する固有関数である。

 \eqref{eq:g_0_lin_mu_1}より、
 以下が成立する$H_0^1(\Omega)$の点列$\{ \psi_n \}$が存在する。
 \begin{align}
  \int_\Omega b(g_0)^{p-1} \psi_n^2 dx &= 1, \label{eq:g_0_seq_psi_1} \\
  \int_\Omega \left( \lvert D\psi_n \rvert^2 
  + a \lvert \psi_n \rvert^2 \right) dx
  & \searrow \mu_1. \label{eq:g_0_seq_psi_2}
 \end{align}
 $a > \kappa$であるから、\eqref{eq:g_0_seq_psi_2}の左辺は
 $\left\| \psi_n \right\|_\kappa^2$以下である。
 $\dnorm_\kappa$は
 $\dnorm_{H_0^1(\Omega)}$と同値なノルムであるから、
 $\{ \psi_n \}$は$H_0^1(\Omega)$の有界列である。
 
 ゆえに、必要ならば部分列をとることにより、
 $\phi_1 \in H_0^1(\Omega)$が存在して、$n \to \infty$とすると、
 以下が成立する。
 \begin{align}
  \psi_n \xrightharpoonup{ \mbox{ ~ } } \phi_1 & \ \ \text{weakly~} \tin
  H_0^1(\Omega), \label{eq:psi_n_weakly} \\
  \psi_n \xrightarrow{ \mbox{ ~ } } \phi_1 & \ \ \tin L^q(\Omega) \ \
   (q < p+1), \label{eq:psi_n_L^q} \\
  \psi_n \xrightarrow{ \mbox{ ~ } } \phi_1 & \ \ \ae \tin \Omega. 
    \label{eq:psi_n_ae} 
 \end{align}
 \eqref{eq:psi_n_weakly}より、$H_0^1(\Omega)$ノルムの
 弱下半連続性から、次が成立する。
 \[
  \liminf_{n \to \infty} \left\| \psi_n \right\|_{H_0^1(\Omega)}
 \geq \left\| \phi_1 \right\|_{H_0^1(\Omega)}.
 \]
 ゆえに、\eqref{eq:psi_n_L^q}と合わせて、以下が成立する。
 \begin{equation}
  \mu_1 \geq \int_\Omega \left( \lvert D\phi_1 \rvert^2 + a \lvert
                          \phi_1 \rvert^2
                         \right) dx. \label{eq:psi_infty_1}
 \end{equation}
 また、ソボレフ埋め込み$H_0^1(\Omega) \subset L^{p+1}(\Omega)$より、
 $H_0^1(\Omega)$の有界列$\{ \psi_n \}$は$L^{p+1}(\Omega)$の
 有界列である。したがって、$\{ \psi_n^2 \}$は$L^{N/(N-2)}(\Omega)$の
 有界列である。よって、必要なら部分列をとると、
 $\{\psi_n^2 \}$は$L^{N/(N-2)}(\Omega)$の弱収束列となる。
 一方\eqref{eq:psi_n_ae}から、$\{ \psi_n^2 \}$は$\phi_1^2$に
 $\Omega$上ほとんどいたるところ収束する。したがって、次が成立する。
 \[
 \psi_n^2 \xrightharpoonup{ \mbox{ ~ } } \phi_1^2 \ \ \text{weakly~} \tin
 L^{N/(N-2)}(\Omega).
 \]
 $g_0 \in L^{p+1}(\Omega)$より、$b (g_0)^{p-1} \in
 L^{N/2} (\Omega)$である。$\left(L^{N/(N-2)}(\Omega)\right)^*
 \cong L^{N/2}(\Omega)$
 より、次が成立する。
 \begin{equation}
  \int_\Omega b(g_0)^{p-1} \psi_n^2 dx \xrightarrow{n \to \infty}
   \int_{\Omega} b(g_0)^{p-1} \phi_1^2 dx. \label{eq:psi_infty_2}
 \end{equation}
 \eqref{eq:psi_infty_2}の証明は、\cite{MR1400007}~の Lemma~2.13 によった。
 \eqref{eq:psi_infty_1}と\eqref{eq:psi_infty_2}により、
 次がしたがう。
 \begin{equation}
  \mu_1 \geq \frac{\displaystyle \int_\Omega 
   \left( \lvert D\phi_1 \rvert^2 + a \lvert \phi_1 \rvert^2
   \right) dx}{\displaystyle \int_{\Omega} b(g_0)^{p-1} 
   \phi_1^2 dx}. \label{eq:g_0_ineq}
 \end{equation}
 \eqref{eq:g_0_lin_mu_1}により、
 \eqref{eq:g_0_ineq}の不等号は
 実際には等号が成立する。すなわち、
 \eqref{eq:g_0_lin_mu_1}の右辺の下限は
 $\phi_1 \in H_0^1(\Omega)$により達成される。
 よって$\mu_1 > 0$である。
 
 \eqref{eq:g_0_lin_mu_1}の右辺の形から、
 $\phi_1$が\eqref{eq:g_0_lin_mu_1}の右辺の下限を達成するならば、
 $\left| \phi_1 \right|$も下限を達成する。
 すなわち、$\phi_1 \geq 0 ~\tin \Omega$となる第$1$固有関数がある。
 この$\phi_1$について、次が成立する。
 \[
  - \Delta \phi_1 + a \phi_1 = \mu_1 b (g_0)^{p-1} 
 \phi_1 \geq 0 ~\tin \Omega.
 \]
 ゆえに、強最大値原理により、$\phi_1 > 0 ~\tin \Omega$となる。
 \qedhere
\end{proof}

$g_0$を用いて、次の命題を証明する。

\begin{prop} \label{prop:bar_lambda}
 $\bar{\lambda}$を記号~\ref{nota:ext}のものとする。
 $0 < \bar{\lambda} < \infty$である。
\end{prop}

\begin{proof}
補題~\ref{lem:imp}により、$\lambda_0 > 0$が存在し、$0 < \lambda <
 \lambda_0$に対して、\ref{eq:prob_main}の解が存在する。
ゆえに$\bar{\lambda} > 0$である。
そこで、$\bar{\lambda} < \infty$を示せば証明が完了する。

$\lambda > 0$は、$S_\lambda \neq \emptyset$をみたすものとする。
$u \in S_\lambda$とし、$v = u - \lambda g_0$とする。
このとき、次が成立する。
\[
 -\Delta v + av = bu^p \geq 0
\]
したがって、強最大値原理より、$v > 0 ~\tin \Omega$である。
つまり、$u > \lambda g_0 ~\tin \Omega$がしたがう。
よって、以下が成立する。
\begin{equation}
 - \Delta u + au \geq bu^p \geq b \lambda^{p-1} (g_0)^{p-1} u ~\tin
  \Omega. \label{eq:lambda_infty_g_0} 
\end{equation}
一方、補題~\ref{lem:g_0}により、以下が成立する
$\mu_1 > 0$、
$\phi_1 \in H_0^1(\Omega)$、$\phi_1 > 0 ~\tin \Omega$
が存在する。
\begin{equation}
 -\Delta \phi_1 + a \phi_1 = \mu_1 b (g_0)^{p-1} \phi_1 ~\tin \Omega. 
  \label{eq:lambda_infty_phi_1} 
\end{equation}
 そこで、
 $\text{\eqref{eq:lambda_infty_g_0}} \times \phi_1 - 
 \text{\eqref{eq:lambda_infty_phi_1}} \times u $を$\Omega$
 上積分すると、次を得る。
 \[
  0 \geq (\lambda^{p-1} - \mu_1) \int_\Omega b(g_0)^{p-1} u \phi_1 dx.
 \]
 ここで、$b \geq 0 ~\tin \Omega$、
 $b \not \equiv 0$、$g_0, u, \phi_1 > 0 ~\tin \Omega$であるから、
 右辺の積分は正である。ゆえに、$\lambda^{p-1} - \mu_1 \leq 0$である。
 つまり、$\lambda \leq \mu_1^{1/(p-1)}$となる。
 $\lambda > 0$は$S_\lambda \neq \emptyset$をみたす任意の正の数
 であるから、$\bar{\lambda} \leq \mu_1 ^{1/(p-1)} < \infty$がしたがう。
 \qedhere
\end{proof}

\begin{proof}[定理~\ref{thm:minimal_solution}]
 命題~\ref{prop:bar_lambda}により、
\end{proof}

\subsection{minimal solutionに関する線形化固有値問題}

\ref{eq:prob_main}の minimal solution についての
線形化固有値問題
\begin{equation}
 -\Delta \phi + a \phi = \mu p b (\underline{u}_\lambda)^{p-1} \phi
  ~\tin \Omega, \ \ \phi \in H_0^1(\Omega) \label{eq:lin_prob_min_sol}
\end{equation}
を考察する。特に第$1$固有値、第$1$固有関数について論ずる。

\begin{nota}
 \ref{eq:prob_main}の
 minimal solution $\underline{u}_\lambda \in S_\lambda$ に関する
 線形化固有値問題\eqref{eq:lin_prob_min_sol}の第$1$固有値を
 $\mu_1(\lambda)$とかく。
\end{nota}

\begin{lem} \label{lem:lin_p}
 $0 < \lambda < \bar{\lambda}$とする。
 このとき、以下が成立する。
 \begin{enumerate}[1.] \sage
  \item $\mu_1(\lambda) > 0$である。
        また、$\mu_1(\lambda)$に付随する
        \eqref{eq:lin_prob_min_sol}の固有関数$\phi_1$のうち、
        $\phi_1 > 0 ~\tin \Omega$を
        みたすものが存在する。
  \item 任意の$\psi \in H_0^1(\Omega)$に対し、次が成立する。
        \begin{equation}
         \int_\Omega \left( \lvert D\psi \rvert^2 + a \lvert \psi
                      \rvert^2 \right)
          dx \geq \mu_1(\lambda) \int_\Omega pb(\underline{u}_\lambda)^{p-1}
          \psi^2 dx. \label{eq:lin_prob_min_sol_mu_1}
        \end{equation}
        また、$\psi$が$\mu_1(\lambda)$に付随する固有関数ならば、
        \eqref{eq:lin_prob_min_sol_mu_1}の
        等号が成立する。
 \end{enumerate}
\end{lem}

\begin{proof}
 \begin{enumerate}[1.] \sage
  \item 補題~\ref{lem:g_0}と同様である。
  \item $\mu_1(\lambda)$のレーリッヒ商による特徴付け
        \begin{equation}
         \mu_1(\lambda) = \inf_{\psi \in H_0^1(\Omega), \psi \not \equiv 0} 
        \frac{\displaystyle \int_\Omega 
        \left( \left\lvert D\psi \right\rvert^2 + a\lvert\psi\rvert^2 \right)
          dx }{\displaystyle \int_\Omega pb(\underline{u}_\lambda)^{p-1}
          \psi^2 dx } \label{eq:mu1_quotient}        
        \end{equation}
        から\eqref{eq:lin_prob_min_sol_mu_1}が成立する。
 \end{enumerate}
\end{proof}

補題~\ref{lem:lin_p}から即座に、
$0 < \lambda < \bar{\lambda}$ならば
$\mu_1(\lambda) > 0$であることがわかる。
次の補題では、方程式\ref{eq:prob_main}に着目し、
$\mu_1(\lambda)$についてより多くの情報を引き出す。

\begin{lem} \label{lem:lin_1}
 $0 < \lambda < \bar{\lambda}$とする。このとき、$\mu_1(\lambda) > 1$で
 ある。
\end{lem}

\begin{proof}
 $\hat{\lambda}$を$0 < \lambda < \hat{\lambda} < \bar{\lambda}$を
 みたすものとする。$z = \underline{u}_{\hat{\lambda}} -
 \underline{u}_\lambda$とおく。補題~\ref{lem:minimal_va}.3 より、
 $z > 0 ~\tin \Omega$である。
 \begin{align*}
  -\Delta \underline{u}_{\hat{\lambda}} + a \underline{u}_{\hat{\lambda}} &= b
  \underline{u}_{\hat{\lambda}}^p + \hat{\lambda} f, \\   
  -\Delta \underline{u}_{\lambda} + a \underline{u}_\lambda &= b
  \underline{u}_\lambda^p + \lambda f
 \end{align*}
 の両辺を引いて、次を得る。
 \[
  -\Delta z + az = b (\underline{u}_{\hat{\lambda}}^p -
 \underline{u}_\lambda^p)
 + (\hat{\lambda} - \lambda) f.
 \]
 $x \geq 0$に対し、$x \mapsto x^p$は下に凸であるから、次がしたがう。
 \[
  \underline{u}_{\hat{\lambda}}^p - \underline{u}_\lambda^p > 
 p \underline{u}_\lambda^{p-1} (\underline{u}_{\hat{\lambda}} -
 \underline{u}_\lambda) = p \underline{u}_\lambda^{p-1} z.
 \]
 $(\hat{\lambda} - \lambda) f \geq 0$と合わせて、次を得る。
 \begin{equation}
  -\Delta z + az > bp \underline{u}_\lambda^{p-1} z  \ \ \tin \Omega.
   \label{eq:mu_1_z_1}
 \end{equation}

 $\mu_1 = \mu_1(\lambda)$とする。
 補題~\ref{lem:lin_p}より、$\phi_1 > 0 ~\tin \Omega$があって、
 \begin{equation}
  -\Delta \phi_1 + a \phi_1 =
   \mu p b \underline{u}_\lambda^{p-1} \phi_1  \ \ \tin \Omega
   \label{eq:mu_1_z_2}
 \end{equation}
 $ \text{\eqref{eq:mu_1_z_1}} \times \phi_1 - 
 \text{\eqref{eq:mu_1_z_2}} \times z$を
 $\Omega$上積分すると、
 \[
  0 > (1 - \mu_1) p \int_\Omega b \underline{u}_\lambda^{p-1} \phi_1 z dx
 \]
 となる。
 ここで、$b \geq 0 ~\tin \Omega$、
 $b \not \equiv 0$、
 $\underline{u}_\lambda, z, \phi_1 > 0 ~\tin \Omega$であるから、
 右辺の積分は正である。ゆえに、$1 - \mu_1 < 0$である。
 つまり$\mu_1 > 1$である。 \qedhere
\end{proof}

% Local Variables:
% mode: yatex
% TeX-master: "main.tex"
% End:
%#!platex main.tex
\section{second solutionの存在 1 --- 命題~\ref{prop:second_1}の証明} \label{sec:second_sol}

\subsection{second solution を求めるための方針}

本節と次節で、定理~\ref{thm:second_solution}を証明する。
本節と次節を通し、$0 < \lambda < \bar{\lambda}$とする。
方程式\ref{eq:prob_sec}を考察するために、以下の記号をおく。

\begin{nota}
 \begin{enumerate}[1.]
  \item $\R \times \R \times \Omega$を定義域とする実数値関数$g, G$を
        以下の通りに定める。
        \begin{align}
         g(t, s, x) &= b(x) \left( (t_+ + s)^p - s^p \right) - a t_+, 
         \label{eq:def_g} \\
         G(t, s, x) &= \int_0^{t_+} g(t, s, x) dt
         \notag \\
         &= b(x) \left( \frac{1}{p+1} (t_+ + s)^{p+1} - \frac{1}{p+1}
         s^{p+1} - s^p t_+ \right) - \frac{1}{2} a(x) t_+^2.
         \label{eq:def_G}
        \end{align}
        $g(v, \underline{u}_\lambda, x)$を$g(v, \underline{u}_\lambda
        )$と表記する。
        $G(v,\underline{u}_\lambda, x)$を$G(v, \underline{u}_\lambda
        )$と表記する。
  \item $I_{\lambda} \colon H_0^1(\Omega) \to \R$を以下の通りに定める。
        \begin{equation}
         I_\lambda (v) = \frac{1}{2} \int_\Omega \lvert Dv \rvert^2 dx
          - \int_\Omega G(v, \underline{u}_\lambda) dx. \label{eq:def_I}
        \end{equation}
        $I_\lambda$のフレッシェ微分を$I_\lambda^\prime$と表記する。
 \end{enumerate}
\end{nota}

\ref{eq:prob_sec}の考察を始める前に、
\ref{eq:prob_main}と\ref{eq:prob_sec}の関係、および、
\ref{eq:prob_sec}と$I_\lambda$の関係を明らかにする。

\begin{lem} \label{lem:rel_heart_spade}
 \begin{enumerate}[1.]
  \item 以下の(1), (2)は同値である。
        \begin{enumerate}[(1)]
         \item \ref{eq:prob_main}の minimal solution $\underline{u}_\lambda$
               以外の弱解$\bar{u}_\lambda \in H_0^1(\Omega)$が存在する。
         \item \ref{eq:prob_sec}の
               弱解$v \in H_0^1(\Omega)$が存在する。
        \end{enumerate}
  \item $v \in H_0^1(\Omega)$は
        \eqref{eq:def_I}で定まる$I_\lambda$の臨界点であると仮定する。
        このとき、$v$は\ref{eq:prob_sec}の弱解である。
 \end{enumerate}
\end{lem}

\begin{proof}
 \begin{enumerate}[1.] \sage
  \item \ulinej{(1){$\Rightarrow$}(2)}:
        $v = \bar{u}_\lambda - \underline{u}_\lambda$とする。
        $\underline{u}_\lambda$は\ref{eq:prob_main}のminimal solution
        であるから、$v \geq 0 ~\tin \Omega$である。
        そこで、
        \begin{align*}
         -\Delta \bar{u}_\lambda + a \bar{u}_\lambda &= b
         \bar{u}_\lambda^p + \lambda f, \\
         -\Delta \underline{u}_\lambda + a \underline{u}_\lambda &= b
         \underline{u}_\lambda^p + \lambda f
        \end{align*}
        の両辺を引くと、
        \[
        -\Delta v + a v = b \left( (\underline{u}_\lambda + v)^p -
        \underline{u}_\lambda^p \right)
        \]
        が得られる。この右辺は非負である。$a \geq \kappa > -\kappa_1$であ
        るから、強最大値原理より、$v > 0 ~\tin \Omega$である。
        以上より、$v \in H_0^1(\Omega)$は\ref{eq:prob_sec}の弱解である。

        \ulinej{(2){$\Rightarrow$}(1)}:$\bar{u}_\lambda = v +
        \underline{u}_\lambda$とすれば、$\bar{u}_\lambda$は
        \ref{eq:prob_main}の弱解である。
  \item $I_\lambda$は$C^1$級であり、そのフレッシェ微分は、
        $u \in H_0^1(\Omega)$、$\psi \in H_0^1(\Omega)$として、
        \[
         I^\prime_\lambda(u)\psi = \int_\Omega \left( Dv \cdot D\psi -
        g(v, \underline{u}_\lambda) \psi \right) dx.
        \]
        と表される。$v \in H_0^1(\Omega)$は$I_\lambda$の
        臨界点であるから、$I^\prime_\lambda(v) = 0$である。
        すなわち、
        \begin{equation}
         \int_\Omega \left( Dv \cdot D\psi - g(v,
                      \underline{u}_\lambda)\psi \right) dx = 0
         \label{eq:weak_sol_of_heart}        
        \end{equation}
        が成立する。この$\psi \in H_0^1(\Omega)$は任意であるから、
        $v \in H_0^1(\Omega)$は
        \begin{align}
         \left\{
          \begin{aligned}
           -\Delta v + a v &= b \left( (v_+ + \underline{u}_\lambda)^p -
           (\underline{u}_\lambda)^p \right) 
           & &\text{in~} \Omega, \\
           v &= 0 & &\text{on~} \partial\Omega
          \end{aligned}
         \right.
        \end{align}
        の弱解である。$(v_+ + \underline{u}_\lambda)^p -
           (\underline{u}_\lambda)^p \geq 0 ~\tin \Omega$、
        $a \geq \kappa > -\kappa_1$より、強最大値原理から、
        $v > 0 ~\tin \Omega$が従う。ゆえに$v \in H_0^1(\Omega)$
        は\ref{eq:prob_sec}の
        弱解である。 \qedhere
 \end{enumerate}
\end{proof}

ここで次の記号を置く。

\begin{defn} \label{defn:S_def}
 $V \subset \R^N$を領域とする。
 \begin{equation}
  S = \inf_{u \in H^1_0(V), u \not \equiv 0}
 \frac{\left\| Du \right\|_{L^2(V)}^2}{\left\| u
                                       \right\|_{L^{p+1}(V)}^2}  
 \label{eq:S_def}
 \end{equation}
 を{\bf ソボレフ最良定数}という。
\end{defn}

$S$は$V$には依存しないことが知られている。例えば~\cite{田中200808}の定
理~2.31~(i) を参照されたい。

次の2つの命題を証明することにより、
定理~\ref{thm:second_solution}を証明する。

\begin{prop} \label{prop:second_1}
 $0 < \lambda < \bar{\lambda}$とする。
 $v \geq 0 ~\tin \Omega$、かつ、
 \begin{equation}
  \int_\Omega b v^{p+1} dx > 0, \label{eq:int_omega_bvp+1}  
 \end{equation}
 かつ、
 \begin{equation}
  \sup_{t > 0} I_\lambda (tv_0) < 
   \frac{1}{N\left\| b
             \right\|_{L^\infty(\Omega)} ^{(N-2)/2}} S^{N/2} 
   \label{eq:ineq_S}
 \end{equation}
 をみたす$v_0 \in H_0^1(\Omega)$が存在することを仮定する。
 このとき、\ref{eq:prob_sec}の弱解$v \in H_0^1(\Omega)$が存在する。
\end{prop}

\begin{prop} \label{prop:second_2}
 定理~\ref{thm:second_solution}の仮定のもとで、$v_0 \geq 0 ~\tin
 \Omega$、\eqref{eq:int_omega_bvp+1}、
 および、\eqref{eq:ineq_S}をみたす
 $v_0 \in H_0^1(\Omega)$が存在する。
\end{prop}

命題~\ref{prop:second_1}の証明は本節、
命題~\ref{prop:second_2}の証明は次節でおこなう。

\subsection{命題~\ref{prop:second_1}の証明}

本小節では、命題~\ref{prop:second_1}の証明を与える。

まずは、以降の議論で使用する積分の極限について議論する。

\begin{nota}
 関数$H, h, H^\prime, h^\prime, G^\prime, g^\prime$を以下の通りに定める。
 \begin{align*}
  H(t, s, x) &= G(t, s, x) - \frac{1}{p+1}b(x) t_+ ^{p+1}, \\
  h(t, s, x) &= g(t, s, x) - b(x) t_+ ^{p}, \\
  H^\prime (t, s, x) &= H(t, s, x) + \frac{1}{2}a(x) t_+^2, \\
  h^\prime (t, s, x) &= h(t, s, x) + a(x) t_+, \\
  G^\prime (t, s, x) &= G(t, s, x) + \frac{1}{2}a(x) t_+^2, \\
  g^\prime (t, s, x) &= g(t, s, x) + a(x) t_+.
 \end{align*}
 $H(v, \underline{u}_\lambda, x)$を
 $H(v, \underline{u}_\lambda)$と表記する。
 $h(v, \underline{u}_\lambda)$、
 $H^\prime(v, \underline{u}_\lambda)$、
 $h^\prime(v, \underline{u}_\lambda)$、
 $G^\prime(v, \underline{u}_\lambda)$、
 $g^\prime(v, \underline{u}_\lambda)$も全て同様である。
\end{nota}

\begin{lem} \label{lem:conv}
 $v \in H_0^1(\Omega)$とし、$\{ v_k \}_{k = 0}^\infty$を
 $H_0^1(\Omega)$の有界列とする。$k \to \infty$のとき、
 $v_k \to v ~\ae ~\tin \Omega$と仮定する。このとき、$k \to \infty$
 とすると、以下が成立する。
 \begin{align}
  \int_\Omega H(v_k, \underline{u}_\lambda)dx &\xrightarrow{ \mbox{ ~
  } } 
  \int_\Omega H(v, \underline{u}_\lambda)dx, \label{eq:conv_H} \\
  \int_\Omega h(v_k, \underline{u}_\lambda)v_k dx &\xrightarrow{
  \mbox{ ~ } } 
  \int_\Omega h(v, \underline{u}_\lambda)v dx. \label{eq:conv_h} \\
  \intertext{また、任意の$\psi \in H_0^1(\Omega)$に対し、}
  \int_\Omega g(v_k, \underline{u}_\lambda)\psi dx &\xrightarrow{
  \mbox{ ~ } } 
  \int_\Omega g(v, \underline{u}_\lambda)\psi dx \label{eq:conv_g}
 \end{align}
 が成立する。
\end{lem}

\begin{proof}
 まず、\eqref{eq:conv_H}を証明する。
 $\{ v_k \}$は$H_0^1(\Omega)$の有界列で、$v$に$\Omega$上
 ほとんどいたるところ
 収束するから、
 必要ならば部分列をとることにより、$k \to \infty$とすると、
 以下が成立する。
 \begin{align}
  v_k \xrightharpoonup{ \mbox{ ~ } } v & \ \ \text{weakly~} \tin
  H_0^1(\Omega), \label{eq:minimal_vk_weakly} \\
  v_k \xrightarrow{ \mbox{ ~ } } v & \ \ \tin L^q(\Omega) \ \
   (q < p+1), \label{eq:minimal_vk_Lq} \\
  v_k \xrightarrow{ \mbox{ ~ } } v & \ \ \ae \tin \Omega. 
    \label{eq:minimal_vk_ae}
 \end{align}
 \eqref{eq:minimal_vk_Lq}より、
 \[
  \int_\Omega \frac{1}{2} a v_k^2 dx \xrightarrow{k \to \infty}
 \int_\Omega \frac{1}{2}av^2 dx
 \]
 がわかるので、\eqref{eq:conv_H}を示すためには、
 \begin{equation}
  \int_\Omega H^\prime(v_k, \underline{u}_\lambda)dx \xrightarrow{ k \to \infty } 
  \int_\Omega H^\prime(v, \underline{u}_\lambda)dx \label{eq:conv_Hprime} \\  
 \end{equation}
 を示せば十分である。以下\eqref{eq:conv_Hprime}を示す。
 $t, s \geq 0$のとき、次式が成立する。
 \begin{align}
  H^\prime(t, s, x) &= b(x) \left( \frac{1}{p+1}(t+s)^{p+1} -
  \frac{1}{p+1} s^{p+1} - s^p t - \frac{1}{p+1} t^{p+1} \right) \notag \\
  & \leq b(x) \left( \frac{1}{p+1}(t+s)^{p+1} - \frac{1}{p+1}s^{p+1} -
  \frac{1}{p+1} t^{p+1}\right) \notag \\
  & \leq b(x) \int_0^t \left( (\tau + s)^p - \tau^p \right)
  d\tau. \label{eq:dtau} 
 \end{align}
 ここで、$x \geq 0$に対し、$x \mapsto x^p$は下に凸であるから、
 $(\tau + s)^p - \tau^p \leq p(\tau + s)^{p-1} s$である。
 さらに、
 \begin{equation}
  (\tau + s)^{p-1} \leq (2 \max\{\tau , s\})^{p-1} = 2^{p-1} \max \{
   \tau^{p-1} + s^{p-1} \} \leq C (\tau^{p-1} + s^{p-1}) \label{eq:taus2p-1}
 \end{equation}
 であるから、次が得られる。
 \begin{equation}
  H^\prime(t, s, x) \leq C b \int_0^t (\tau^{p-1} + s^{p-1}) s d \tau
   \leq C b ( t^{p} s + s^{p} t). \label{eq:fromlemma}
 \end{equation}
 \eqref{eq:fromlemma}の証明は、\cite{MR2317491}~の Lemma~C.4を参考にし
 た。さらにヤングの不等式を適用すると、任意の$\epsilon > 0$に対し、
 $C > 0$が存在し、$s, t \geq 0$に対し、
 $H^\prime(t, s, x) \leq b( \epsilon t^{p+1} + C s^{p+1})$
 が成立する。ゆえに、次式が得られる。
 \begin{equation}
  \left\lvert H^\prime(v_k, \underline{u}_\lambda) - H^\prime(v,
   \underline{u}_\lambda ) \right\rvert \leq b \left( \epsilon
   (v_k)_+^{p+1}  + \epsilon v_+^{p+1} + C
   \underline{u}_\lambda^{p+1} \right). \label{eq:conv_H_abs}
 \end{equation}
 そこで、
 \begin{equation}
  W_{\epsilon, k} = \left( \left\lvert H^\prime(v_k,
                            \underline{u}_\lambda) - H^\prime(v,
                            \underline{u}_\lambda) \right\rvert
  -\epsilon b (v_k)_+^{p+1}
                    \right)_+ \label{eq:conv_H_W}
 \end{equation}
 とおくと、$k \to \infty$のとき、$W_{\epsilon, k} \to 0 ~\ae ~\tin
 \Omega$である。また、\eqref{eq:conv_H_abs}より、
 $\lvert W_{\epsilon, k} \rvert \leq b \left( \epsilon v_+^{p+1} + C
 \underline{u}_\lambda^{p+1} \right)$であり、この右辺は可積分である。
 したがって、優収束定理により、
 \[
  \lim_{k \to \infty} \int_\Omega  W_{\epsilon, k}(x) dx = 0
 \]
 である。さて、$\{v_k \}$は$H_0^1(\Omega)$の有界列であった。
 $H_0^1(\Omega) \subset L^{p+1}(\Omega)$のソボレフ不等式も
 考慮すると、
 \[
   \int_\Omega b ( v_k )_+^{p+1} dx \leq C
 \]
 をみたす$k$によらない$C > 0$が存在する。
 \eqref{eq:conv_H_W}より、
 \[
  \int_\Omega \left\lvert H(v_k, \underline{u}_\lambda) - H(v,
 \underline{u}_\lambda ) \right\rvert dx \leq \int_\Omega W_{\epsilon,
 k} (x)dx + \epsilon \int_\Omega b(v_k)_+^{p+1} dx
 \]
 であるから、$k \to \infty$の上極限をとると、次式が得られる。
 \[
 \limsup_{k \to \infty} \int_\Omega 
 \left\lvert H(v_k, \underline{u}_\lambda) - H(v,
 \underline{u}_\lambda ) \right\rvert dx \leq C \epsilon.
 \]
 $C > 0$は$k, \epsilon$によらず、$\epsilon > 0$は任意であるから、
 このことは
 \[
  \lim_{k \to \infty}
 \int_\Omega \left\lvert H(v_k, \underline{u}_\lambda) - H(v,
 \underline{u}_\lambda ) \right\rvert dx = 0
 \]
 と同値である。ゆえに\eqref{eq:conv_H}が成立する。
 以上の証明は、直接は~\cite{MR2886160}の Lemma~3.1 を参考にしているが、
 ~\cite{MR699419}のアイデアを参考にした。
 
 \eqref{eq:conv_h}も\eqref{eq:conv_H}と同様に証明される。
 \eqref{eq:conv_h}を示すためには、やはり
 \[
  \int_\Omega h^\prime(v_k, \underline{u}_\lambda) v_k dx
 \xrightarrow{k \to \infty} \int_\Omega h^\prime (v,
 \underline{u}_\lambda ) v dx 
 \]
 を示せば十分である。$t, s \geq 0$に対し、
 \[
  h(t, s, x) t \leq C b(x) (t^p s + s^p t)
 \]
 が従うため、\eqref{eq:conv_h}と同様に\eqref{eq:conv_H}も得られる。
 
 最後に\eqref{eq:conv_g}を証明する。\eqref{eq:minimal_vk_weakly}より、
 \[
  \int_\Omega a v_k \psi dx \xrightarrow{k \to \infty} \int_\Omega
 av\psi dx
 \]
 であるから、\eqref{eq:conv_g}を示すためには、
 \[
  \int_\Omega g^\prime(v_k, \underline{u}_\lambda)\psi dx \xrightarrow{
  k \to \infty } 
  \int_\Omega g^\prime(v, \underline{u}_\lambda)\psi dx
 \]
 を示せば十分である。
 \eqref{eq:dtau}、\eqref{eq:taus2p-1}と同様にすれば、
 $s, t, r \geq 0$に対し、次式が従う。
 \[
  g^\prime(t, s, x)r = b(x) \left( (t+s)^p - s^p \right) r \leq C b(x)
 \left( t^pr + s^p r \right).
 \]
 ヤングの不等式を$2$回使用すると、$\epsilon > 0$に対し、$C, C^\prime >
 0$が存在し、$s, t, r \geq 0$に対し、
 \[
  t^p r + s^p r \leq \epsilon t^{p+1} + C r^{p+1} + s^p r \leq
 \epsilon t^{p+1} + C^\prime (r^{p+1} + s^{p+1})
 \]
 が成立する。ゆえに、$g^\prime$は
 \[
  g^\prime(t, s, x)r \leq b \left( \epsilon t^{p+1} + C ( r^{p+1} +
 s^{p+1}) \right)
 \]
 と評価される。したがって、次式が成立する。
 \[  
  \left\lvert g^\prime(v_k, \underline{u}_\lambda)\psi - g^\prime(v,
   \underline{u}_\lambda)\psi \right\rvert \leq b \left( \epsilon
   (v_k)_+^{p+1}  + \epsilon v_+^{p+1} + C
   (\underline{u}_\lambda^{p+1} + \lvert \psi \rvert^{p+1})
                                 \right).
 \]
 そこで、$\tilde{W}_{\epsilon, k}$を
\[  \tilde{W}_{\epsilon, k} = 
   \left( \left\lvert g^\prime(v_k, \underline{u}_\lambda)\psi - g^\prime(v,
    \underline{u}_\lambda)\psi \right\rvert
  -\epsilon b (v_k)_+^{p+1}
          \right)_+ \]
 と定める。以降は、\eqref{eq:conv_H}の証明と同様に、\eqref{eq:conv_g}
 が示される。\qedhere
\end{proof}

命題~\ref{prop:second_1}を証明には、
$(\mathrm{PS})$条件を課さない峠の定理を使用する。
このため、$I_\lambda$の
$(\mathrm{PS})_c$条件が必要となる。
$I_\lambda$の
$(\mathrm{PS})_c$条件を調べる準備として、
$I_\lambda$についてのパレ・スメイル列が
$H_0^1(\Omega)$の有界列であることを証明する。

\begin{lem} \label{lem:PS_seq}
 $\{ v_k \}_{k=0}^\infty$は、$I_\lambda$についての
 パレ・スメイル列であると仮定する。すなわち、
 以下の(i)、(ii)を仮定する。
 \begin{enumerate}[(i)]
  \item $\{ I_\lambda(v_k) \}$は有界列。
  \item $\displaystyle I^\prime_{\lambda} (v_k) \xrightarrow{k \to
        \infty} 0 ~\tin H^{-1}(\Omega)$。
 \end{enumerate}
 このとき、$\{ v_k \}$は$H_0^1(\Omega)$の有界列である。
\end{lem}

\begin{proof}
 (i)より、
 \begin{equation}
  \frac{1}{2} \left\| v_k \right\|^2 - \int_\Omega G(v_k,
   \underline{u}_\lambda )dx \leq M \label{eq:ps_1}
 \end{equation}
 となる$k \in \N$によらない$M > 0$が存在する。
 $\epsilon > 0$とする。(ii)より、$K \in \N$が存在し、
 $k \geq K$、$\psi \in H_0^1(\Omega)$に対し、
 \[
  \int_\Omega (D v_k \cdot D \psi) dx - \int_\Omega g(v_k,
 \underline{u}_\lambda)\psi dx \leq \epsilon \left\| \psi
 \right\|_{H_0^1(\Omega) }
 \]
 が成立する。$\psi = v_k$とすると、次式が得られる。
 \begin{equation}
  \left\| v_k \right\|^2 \geq \int_\Omega g(v_k,
   \underline{u}_\lambda)v_k dx - \epsilon \left\| v_k
   \right\|_{H_0^1(\Omega)}. \label{eq:ps_2}
 \end{equation}
 $\alpha > 0$とする。
 \eqref{eq:ps_1}、\eqref{eq:ps_2}より、以下が従う。
 \begin{align}
  \alpha M &\geq  \frac{\alpha}{2} \left\| v_k \right\|^2 - \alpha
  \int_\Omega G(v_k, \underline{u}_\lambda) dx \notag \\
  & \geq \frac{\alpha-2}{2} \left\| v_k \right\|^2 + \int_\Omega 
  \left( g(v_k, \underline{u}_\lambda)v_k - \alpha G(v_k,
  \underline{u}_\lambda)  \right) dx - \epsilon \left\| v_k
  \right\|_{H^1_0(\Omega)}. \label{eq:alpha_M}
 \end{align}
 右辺の積分の中身を考察する。$t, s \geq 0$に対し、
 \[
   g(t, s)t - \alpha G(t, s)
  = b \left( (t+s)^p t - s^p t - \frac{\alpha}{p+1}(t+s)^{p+1} +
  \frac{\alpha}{p+1}s^{p+1} + \alpha s^p t \right) - a \left( t^2 -
  \frac{\alpha}{2} t^2 \right)
 \]
 である。ここで
 \[
  F(t) = (t+s)^p t - \frac{\alpha}{p+1}(t+s)^{p+1}
 \]
 の$t = 0$のまわりの$2$次のテイラー多項式は、
 \[
 s^pt + ps^{p-1} t^2 - \frac{\alpha}{p+1} s^{p+1} - \alpha s^p t -
 \frac{\alpha p}{2} s^{p-1} t^2 
 \]
 と計算される。$F$の$3$階の導関数は、
 \[
  F^{\prime\prime\prime}(t) = p(p-1)(p-2)(t+s)^{p-3} t + (3 - \alpha)
 p(p-1) (t+s)^{p-2}
 \]
 と計算される。テイラーの定理より、
 $3$次の剰余項$R_3$は、$0 < \theta < 1$を用いて、
 \[
  R_3 = \frac{F^{\prime\prime\prime}(\theta t)}{3!}t^3 = \frac{t^3}{6}
 \left( p(p-1)(p-2)(\theta t + s)^{p-3} \theta t + (3-\alpha) p(p-1)
 (\theta t + s)^{p-2} \right)
 \]
 とかける。以下では、$\alpha$を$p$に応じて定め、$R_3 \geq 0$となるように
 する。$p$の値に応じて場合分けをする。

 \ulinej{{$p \geq 2$}のとき}:$\alpha = 3$とすると、$R_3 \geq 0$が従う。

 \ulinej{{$1 < p \leq 2$}のとき}:$\alpha = p + 1$とすると、以下の通り
 $R_3 \geq 0$が従う。
 \begin{align*}
  R_3 &= \frac{t^3}{6} p(p-1)(\theta t + s)^{p-3} \left( (p-2)\theta t
  + (2-p) (\theta t + s)\right) \\
  &= \frac{t^3}{6}p(p-1)(2-p) s (\theta t + s)^{p-3} \geq 0.
 \end{align*}
 以上より、
 \begin{align*}
  g(t, s, x)t - \alpha G(t, s, x) &= b\left( ps^{p-1}t^2 -
  \frac{\alpha p}{2} s^{p-1} t^2 + R_3 \right) - a \left( t^2 -
  \frac{\alpha}{2} t^2 \right) \\
  & \geq b\left( ps^{p-1}t^2 -
  \frac{\alpha p}{2} s^{p-1} t^2 \right) - a \left( t^2 -
  \frac{\alpha}{2} t^2 \right) \\
  & = \left( \frac{\alpha}{2} - 1 \right) \left( at^2 - bps^{p-1} t^2 \right)
 \end{align*}
 と下から評価される。これを
 \eqref{eq:alpha_M}に適用すると、$a >
 \kappa$、及び、
 $\dnorm_\kappa$と$\dnorm_{H_0^1(\Omega)}$が
 同値であることから、以下の式変形が進む。
 \begin{align*}
  \alpha M &\geq \frac{\alpha-2}{2} \left( \left\| v_k \right\|^2 -
  \int_\Omega bp \underline{u}_\lambda^{p-1} v_k^2 dx + \int_\Omega
  av_k^2 dx \right) - \epsilon \left\| v_k \right\|_{H_0^1(\Omega)} \\
  & \geq \frac{\alpha -2}{2} \left( 1 - \frac{1}{\mu_1(\lambda)} \right)
  \left( \left\| v_k \right\|^2 + \int_\Omega av_k^2 dx \right)
  - \epsilon \left\|
  v_k \right\|_{H_0^1(\Omega)} \\
  & \geq \frac{\alpha -2}{2} \left( 1 - \frac{1}{\mu_1(\lambda)}
  \right) \left\| v_k \right\|_\kappa^2
  - \epsilon \left\|
  v_k \right\|_{H_0^1(\Omega)} \\
  & \geq  \frac{\alpha -2}{2} \left( 1 - \frac{1}{\mu_1(\lambda)}
  \right) C \left\| v_k \right\|_{H_0^1(\Omega)}^2 - \epsilon \left\|
  v_k \right\|_{H_0^1(\Omega)}.
 \end{align*}
 $C >0$はポアンカレの不等式から決まる$k$によらない定数である。
 $\alpha$の定め方から$\alpha > 2$、補題~\ref{lem:lin_1}から
 $1 - 1/\mu_1(\lambda) > 0$であるから、結局$k \geq K$に対し、
 \[
  \alpha M \geq C^\prime \left\| v_k \right\|_{H_0^1(\Omega)}^2 - \epsilon
 \left\| v_k \right\|_{H_0^1(\Omega)}
 \]
 が成立する。ゆえに、$\{ v_k \}$は$H_0^1(\Omega)$の有界列である。 \qedhere
\end{proof}

補題~\ref{lem:PS_seq}を用いて、
$I_\lambda$の$(\mathrm{PS})_c$条件を調べる。

\begin{lem} \label{lem:PS_c}
 \begin{enumerate}[1.] \sage
  \item  $0 < c < S^{N/2}/N\left\| b \right\|_{L^\infty(\Omega)}
         ^{(N-2)/2}$とする。このとき、
         $I_\lambda$は$(\mathrm{PS})_c$条件をみたす。すなわち、
         次の条件(i), (ii)をみたす$H_0^1(\Omega))$の点列
         $\{ v_k \}_{k = 0}^\infty$は、収束する部分列をもつ。
         \begin{enumerate}[(i)]
          \item $\displaystyle \lim_{k \to \infty} I_\lambda (v_k) = c$。
          \item $\displaystyle I^\prime_\lambda (v_k) \xrightarrow{k \to
                \infty} 0 ~\tin H^{-1}(\Omega)$。
         \end{enumerate}
  \item 1.~における$\{ v_k \}$の部分列の収束先を$v$とする。
        $v$は\ref{eq:prob_sec}の弱解である。
 \end{enumerate}
\end{lem}

\begin{proof}
 仮定(i), (ii)と補題~\ref{lem:PS_seq}より、$\{v_k \}$は
 $H_0^1(\Omega)$の有界列である。
 したがって、必要ならば部分列をとることにより、$v \in H_0^1(\Omega)$
 が存在し、$k \to \infty$とすると、
 以下が成立する。
 \begin{align}
  v_k \xrightharpoonup{ \mbox{ ~ } } v & \ \ \text{weakly~} \tin
  H_0^1(\Omega), \label{eq:vk_weakly} \\
  v_k \xrightarrow{ \mbox{ ~ } } v & \ \ \tin L^q(\Omega) \ \
   (q < p+1), \label{eq:vk_Lq} \\
  v_k \xrightarrow{ \mbox{ ~ } } v & \ \ \ae \tin \Omega. 
    \label{eq:vk_ae}
 \end{align}
 (ii)より、任意の$\psi \in H_0^1(\Omega)$に対し、
 \[
 \int_\Omega (Dv_k \cdot D\psi) dx - \int_\Omega g(v_k,
 \underline{u}_\lambda) \psi dx = o(1) \ \ (k \to \infty)
 \]
 が成り立つ。
 \eqref{eq:vk_weakly}と補題~\ref{lem:conv}より、
 $k \to \infty$とすると、次式が従う。
 \begin{equation}
  \int_\Omega (Dv \cdot D\psi) dx - \int_\Omega g(v,
   \underline{u}_\lambda) \psi dx = 0. \label{eq:intvpsi}
 \end{equation}
 つまり、$v \in H_0^1(\Omega)$は、$I_\lambda$の臨界点である。
 よって補題~\ref{lem:rel_heart_spade}.2により、$v$は
 \ref{eq:prob_sec}の弱解である。
 したがって、あとは、
 $k \to \infty$のとき、$v_k \to v ~\tin H_0^1(\Omega)$
 であることを示せば、補題の証明が完了する。

 ここで、
 \begin{equation}
  I_\lambda(v) \geq 0 \label{eq:Ilambdageq0}
 \end{equation}
 であることを示す。
 \eqref{eq:intvpsi}で$\psi = v$とすると、
 \begin{equation}
  \int_\Omega \lvert Dv \rvert^2 dx = \int_\Omega g(v,
   \underline{u}_\lambda)v dx \label{eq:v_kankei}
 \end{equation}
 という関係式が導かれる。ゆえに、$v$における$I_\lambda$の値は
 \begin{equation}
  I_\lambda (v) = \frac{1}{2} \int_\Omega g(v, \underline{u}_\lambda)
   v dx - \int_\Omega G(v, \underline{u}_\lambda) dx \label{eq:Ilambda_v}
 \end{equation}
 と書けることがわかる。
 そこで、$t, s \geq 0$、$x \in \Omega$に対し、
 \begin{align*}
  \frac{1}{2} g(t, s, x) t - G(t, s, x) &= \frac{1}{2} \left(b \left(
  (t+s)^p - s^p \right) at \right)t - \left( b \left(
  \frac{1}{p+1}(t+s)^{p+1} - \frac{1}{p+1}s^{p+1} - s^p t -
  \frac{1}{2}at^2  \right) \right) \\
  &= b \left( \frac{1}{2} \left( (t+s)^pt - s^pt \right) -\left(
  \frac{1}{p+1} (t+s)^{p+1} - \frac{1}{p+1} s^{p+1} - s^p t \right) \right)
 \end{align*}
 を考える。
 \[
  \alpha(t) = \frac{1}{2} (t+s)^p t - \frac{1}{p+1}(t+s)^{p+1}
 \]
 の$t = 0$のまわりの$1$次のテイラー多項式は、
 \[
  \frac{1}{2} s^p t - \frac{1}{p+1}s^{p+1} - s^p t
 \]
 である。$\alpha$の$2$階の導関数は、
 \[
  \alpha^{\prime\prime}(t) = \frac{p(p-1)}{2}(t+s)^{p-2}t
 \]
 と計算されるから、$2$次の剰余項は、$0 < \theta < 1$を用いて、
 \[
  \frac{\alpha^{\prime\prime}(\theta t)}{2} t^2 =
 \frac{p(p-1)}{2}(\theta t + s)^{p-2} \theta t^3
 \]
 と表すことができる。ゆえに、
 \[
  \frac{1}{2} g(t, s, x)t - G(t, s, x) = b(x) \frac{p(p-2)}{2} (\theta
 t + s)^{p-2} \theta t^3 \geq 0
 \]
 とわかる。\eqref{eq:Ilambda_v}と合わせ、
 \eqref{eq:Ilambdageq0}が得られる。
 \eqref{eq:Ilambdageq0}は、本証明の最後で重要な役割を担う。

 $k \to \infty$のとき
 $v_k \to v ~\tin H_0^1(\Omega)$であることを示すために、
 $w_k = v_k - v$とおく。$H_0^1(\Omega)$の点列
 $\{ w_k \}_{k=0}^\infty$について、以下が成立する。
 \begin{align}
  w_k \xrightharpoonup{ \mbox{ ~ } } 0 & \ \ \text{weakly~} \tin
  H_0^1(\Omega), \label{eq:wk_weakly} \\
  w_k \xrightarrow{ \mbox{ ~ } } 0 & \ \ \tin L^q(\Omega) \ \
  (q < p+1), \label{eq:wk_Lq} \\
  w_k \xrightarrow{ \mbox{ ~ } } 0 & \ \ \ae \tin \Omega. 
  \label{eq:wk_ae}
 \end{align}
 $I_\lambda(v)$と$I_\lambda(v_k)$の差を、$w_k$を用いて評価する。
 \eqref{eq:wk_weakly}より、以下が成立する。
 \begin{equation}
  \int_\Omega \lvert D v_k \rvert^2 dx = \int_\Omega \lvert Dw_k + Dv
   \rvert^2 dx = \int_\Omega \lvert Dv \rvert^2 dx + \int_\Omega \lvert
   Dw_k \rvert^2 dx + o(1). \label{eq:Dvk_Dwk}
 \end{equation}
 ここで、$\tilde{w}_k = (v_k)_+ - v$とおく。
 $v > 0 ~\tin \Omega$より、$\lvert \tilde{w}_k \rvert \leq \lvert w_k
 \rvert ~\tin \Omega$である。また、
 $k \to \infty$とすると、$\tilde{w}_k \to 0 ~\ae \tin \Omega$となる。
 ゆえに、ブレジス・リーブの補題~\cite{MR699419}より、次式が成立する。
 \begin{equation}
  \int_\Omega b(v_k)_{+}^{p+1} dx = \int_\Omega bv^{p+1} dx +
   \int_\Omega \lvert b\tilde{w}_k \rvert^{p+1} dx + o(1) \ \ (k \to
   \infty).
   \label{eq:bzlblem}
 \end{equation}
 \eqref{eq:bzlblem}と補題~\ref{lem:conv}、および、$k \to \infty$のとき
 $(v_k)_+ \to v ~\ae ~\tin \Omega$より、次式が成立する。
 \begin{align}
  \int_\Omega g(v_k, \underline{u}_\lambda) v_k dx &= \int_\Omega
  h(v_k, \underline{u}_\lambda) v_k dx + \int_\Omega b (v_k)_+^{p+1}
  dx \notag \\
  &= \int_\Omega
  h(v, \underline{u}_\lambda) v dx + \int_\Omega b v^{p+1}
  dx + \int_\Omega b \lvert \tilde{w}_k \lvert^{p+1} dx + o(1) \notag \\
  &= \int_\Omega g(v, \underline{u}_\lambda) v dx + 
  \int_\Omega b \lvert \tilde{w}_k \lvert^{p+1} dx + o(1). \label{eq:w_k_1}
 \end{align}
 同様にして、次式も得られる。
 \begin{align}
  \int_\Omega G(v_k, \underline{u}_\lambda) dx &= 
  \int_\Omega H(v_k, \underline{u}_\lambda) dx + \frac{1}{p+1}
  \int_\Omega b (v_k)_+^{p+1} dx \notag \\
  &= \int_\Omega G(v, \underline{u}_\lambda) dx + \frac{1}{p+1}
  \int_\Omega b \lvert \tilde{w}_k \rvert^{p+1} dx + o(1). \label{eq:w_k_2}
 \end{align}
 \eqref{eq:Dvk_Dwk}、\eqref{eq:w_k_2}より、
 $I_\lambda(v_k)$と$I_\lambda(v)$の差は以下の通りに書ける。
 \[
  I_\lambda(v_k) = I_\lambda(v) + \frac{1}{2} \int_\Omega \lvert Dw_k
 \rvert^2 dx - \frac{1}{p+1} \int_\Omega \lvert \tilde{w}_k
 \rvert^{p+1} dx + o(1).
 \]
 ここで(i)より、次式が得られる。
 \begin{equation}
  I_\lambda(v) + \frac{1}{2} \int_\Omega \lvert Dw_k
   \rvert^2 dx - \frac{1}{p+1} \int_\Omega \lvert \tilde{w}_k
   \rvert^{p+1} dx = c + o(1). \label{eq:w_k_c}
 \end{equation}
 (ii)より、任意の$\epsilon > 0$に対し、$K \in \N$が存在し、
 $k > K$に対し、
 \[
 \left\lvert \int_\Omega \lvert Dv_k \rvert^2 dx - \int_\Omega g(v_k,
 \underline{u}_\lambda) v_k dx \right\rvert \leq \epsilon \left\| v_k
 \right\|_{H_0^1(\Omega)} 
 \]
 が成立する。$\{ v_k \}$は$H_0^1(\Omega)$の有界列であるから、
 $\epsilon$、$k$によらない$C > 0$が存在し、$k > K$のとき、
 \[
 \left\lvert \int_\Omega \lvert Dv_k \rvert^2 dx - \int_\Omega g(v_k,
 \underline{u}_\lambda) v_k dx \right\rvert \leq \epsilon C
 \]
 が成り立つ。ゆえに、次式が従う。
 \[
 \lim_{k \to \infty} \left( \int_\Omega \lvert Dv_k \rvert^2 dx -
 \int_\Omega g(v_k, \underline{u}_\lambda) v_k dx \right) = 0
 \]
 \eqref{eq:v_kankei}、\eqref{eq:Dvk_Dwk}、および、\eqref{eq:w_k_1}より、
 \[
 \lim_{k \to \infty} \left( \int_\Omega \lvert Dw_k \rvert^2 dx -
 \int_\Omega b \lvert \tilde{w}_k \rvert^{p+1} dx \right) = 0 
 \]
 が成立する。この事実と、$\{ v_k \}$が$H_0^1(\Omega)$の有界列であるこ
 とから、実数列
 \begin{equation}
  \left\{ \int_\Omega \lvert Dw_k \rvert^2 dx \right\}_k , \ 
  \left\{ \int_\Omega b \lvert \tilde{w}_k \rvert^{p+1} dx \right\}_k
  \label{eq:real_seq}
 \end{equation}
 は両方共ある有界閉区間上の数列である。
 したがって、必要ならば$\{ w_k \}$の部分列を取ると、
 \eqref{eq:real_seq}は収束する。
 収束先を$l \geq 0$とすると、
 \[
  \lim_{k \to \infty} \int_\Omega \lvert Dw_k \rvert^2 dx = \lim_{k
 \to \infty} \int_\Omega b \lvert \tilde{w}_k \rvert^{p+1} dx = l
 \]
 をみたす。\eqref{eq:S_def}により、
 \[
  \int_\Omega \lvert Dw_k \rvert^2 dx \geq S \left( \int_\Omega \lvert
 \tilde{w}_k \rvert^{p+1} dx \right)^{2/(p+1)} \geq S \left(
 \frac{1}{\left\| b \right\|_{L^\infty(\Omega)}} \int_\Omega b \lvert
 \tilde{w}_k \rvert^{p+1} dx \right)^{2/(p+1)}
 \]
 が成立する。$k \to \infty$として、
 \begin{equation}
  S l^{2/(p+1)} \leq \left\| b \right\|_{L^\infty(\Omega)}^{2/(p+1)} l 
   \label{eq:PS_c_S}
 \end{equation}
 が得られる。ここで$l = 0$であることを背理法を用いて示す。
 $l > 0$と仮定する。\eqref{eq:PS_c_S}より、
 $l \geq S^{N/2}/ \left\| b \right\|_{L^\infty(\Omega)}^{(N-2)/2}$
 である。一方、\eqref{eq:w_k_c}で$k \to \infty$とすると、
 \[
  I_\lambda(v) = c - \frac{1}{N}l \leq c - \frac{S^{N/2}}{N \left\| b
 \right\|_{L^\infty(\Omega)}^{(N-2)/2}} < 0
 \]
 が得られる。これは\eqref{eq:Ilambdageq0}に反する。
 したがって、$l = 0$である。
 以上より、$k \to \infty$のとき、
 $w_k \to 0 ~\tin H_0^1(\Omega)$である。すなわち、
 $v_k \to v ~\tin H_0^1(\Omega)$である。これが示すべきことであった。\qedhere
\end{proof}

続いて、
$(\mathrm{PS})$条件を課さない峠の定理の仮定がみたされていることを
確認する。

\begin{lem} \label{lem:delta_rho}
 以下の条件をみたす$\delta > 0$、$\rho > 0$が存在する。
 \begin{equation}
  \text{$\left\|v \right\|_{H^1_0(\Omega)} = \delta$をみたす$v \in
   H_0^1(\Omega)$に対し、$I_\lambda(v) \geq \rho$が成立する。} 
   \label{eq:delta_rho}
 \end{equation}
\end{lem}

\begin{proof}
 $v \in H_0^1(\Omega)$は任意のものとする。
 $I_\lambda(v) = I_\lambda(v_+)$であるから、$v \geq 0 ~\tin \Omega$と
 仮定しても一般性を失わない。
 \begin{align*}
  I_\lambda(v) &= \frac{1}{2} \left\| v \right\|^2 - \int_\Omega G(v,
  \underline{u}_\lambda)dx \\
  &= \frac{1}{2} \left( \int_\Omega \left( \lvert Dv \rvert^2 + a v^2
  \right) dx -\int_\Omega pb\underline{u}_\lambda^{p-1} v^2 dx
  \right)
  - \int_\Omega b \left( \frac{1}{p+1}(v +
  \underline{u}_\lambda)^{p-1} - \frac{1}{p+1}
  \underline{u}_\lambda^{p+1} - \underline{u}_\lambda^p v -
  \frac{p}{2} \underline{u}_\lambda^{p-1} v^2 \right) dx.
 \end{align*}
 第$1$項を$J_1$とおき、第$2$項の積分を$J_2$とおく。
 補題~\ref{lem:lin_p}.2 より、
 \begin{equation}
  J_1 \geq \frac{1}{2} \left( 1 - \frac{1}{\mu_1(\lambda)} \right)
   \int_\Omega \left( \lvert Dv \rvert^2 + a v^2 \right) dx \label{eq:J1}
 \end{equation}
 と下から評価される。補題~\ref{lem:lin_1}より、この括弧の中は正である。
 次に、$t, s \geq 0$に対し、$\alpha(t) = (t+s)^{p+1}/(p+1)$と定める。
 ここで、$p$と$2$の大小で場合分けをして$J_2$を評価する。

 \underline{{$p > 2$}のとき}:
 $\alpha$の$t = 0$の周りの$2$次のテイラー多項式は、
 \[
 \frac{1}{p+1} (t+s)^{p+1} + s^p t + \frac{p}{2}s^{p-1}t^2
 \]
 である。$\alpha$の$3$階の導関数は
 $\alpha^{\prime\prime\prime}(t) = p(p-1)(t+s)^{p-2}$であるか
 ら、$3$次の剰余項
 \begin{equation}
  R_3 = \frac{1}{p+1}(t+s)^{p+1} - 
 \frac{1}{p+1} s^{p+1} - s^p t - \frac{p}{2}s^{p-1}t^2 \label{eq:R_3_1}
 \end{equation}
 は、テイラーの定理より、$0 < \theta < 1$を用いて、
 \[
  R_3 = \frac{\alpha^{\prime\prime\prime}(\theta t)}{3!} t^3 =
 \frac{p(p-1)}{6} (\theta t + s)^{p-2} t^3
 \]
 とかける。この$R_3$は、$p - 2 \geq 0$のとき、
 \begin{equation}  
  R_3  \leq C (2 \max \{ t, s \})^{p-2} t^3 
   = C 2^{p-2} ( \max \{ t^{p-2}, s^{p-2} \})t^3
   \leq  C ( t^{p-2} + s^{p-2} )t^3
   = C ( t^{p+1} + s^{p-2} t^3) \label{eq:R_3_2_pow}
 \end{equation}
 と評価される。
 さらに、ヤングの不等式より、任意の$\epsilon > 0$に対し、
 $C > 0$が存在し、$s, t \geq 0$に対し、
 $s^{p-2} t^3 \leq \epsilon s^{p-1} t^2 + C t^{p+1}$
 となる。ゆえに、$R_3$は
 \begin{equation}
  R_3 \leq \epsilon s^{p-1}t^2 + C t^{p+1} \label{eq:R_3_2}
 \end{equation}
 と評価される。\eqref{eq:R_3_1}と\eqref{eq:R_3_2}より、
 $J_2$の評価
 \begin{equation}
  J_2 \leq \epsilon \int_\Omega b \underline{u}_\lambda^{p-1} v^2 dx +
   C \int_\Omega b v^{p+1} dx \label{eq:J2}
 \end{equation}
 が得られる。

 \underline{{$p \leq 2$}のとき}:
 $\alpha$の$t = 0$の周りのテイラー多項式における$2$次の剰余項
 \[
  R_2 = \frac{1}{p+1} (t+s)^{p+1} - \frac{1}{p+1} s^{p+1} -s^p t
 \]
 はテイラーの定理より、$0 < \theta^\prime < 1$を用いて、
 \[
  R_2 = \frac{\alpha^{\prime\prime}(\theta t)}{2!} t^2 = \frac{p}{2}(s
 + \theta t)^{p-1} t^2
 \]
 と書ける。そこで、
 \[
  R_2 - \frac{p}{2}s^{p-1} t^2 = \frac{p}{2}(s
 + \theta t)^{p-1} t^2 - \frac{p}{2}s^{p-1} t^2
 \]
 を考えると、$p - 1 \leq 1$であるから、
 \[
  (s + \theta t)^{p-1} \leq (t+ s)^{p-1} \leq t^{p-1} + s^{p-1}
 \]
 である。ゆえに
 \[
  R_2 - \frac{p}{2}s^{p-1} t^2 \leq \frac{p}{2} t^{p+1}
 \]
 である。よって、\eqref{eq:J2}において$C = p/2$、
 $\epsilon = 0$としたものが
 成立する。$J_2$の評価式としては、$\epsilon > 0$の項を加えた
 \eqref{eq:J2}に合流する。

 \eqref{eq:J2}の$2$つの項は、それぞれ
 補題~\ref{lem:lin_p}.2、ソボレフ不等式より、
 \begin{align*}
  \int_\Omega b\underline{u}_\lambda^{p-1} v^2 dx &\leq
  \frac{1}{p\mu_1(\lambda)} \int_\Omega \left( \lvert Dv \rvert^2 +
  av^2 \right) dx, \\
  \int_\Omega bv^{p+1} dx &\leq \left\| b \right\|_{L^\infty(\Omega)}
  \left\| v \right\|_{L^{p+1}(\Omega)}^{p+1} \leq C \left\| v
  \right\|_{H_0^1(\Omega)}^{p+1}
 \end{align*}
 と更に評価が進む。これらと\eqref{eq:J1}より、$I_\lambda(v)$は
 \[
  I_\lambda(v) = J_1 - J_2 \geq C \int_\Omega \left( \lvert Dv
 \rvert^2 + a v^2 \right) dx - \epsilon C^\prime \int_\Omega \left(
 \lvert Dv \rvert^2 + av^2 \right) dx - C^{\prime\prime} \left\| v \right\|^{p+1}_{H_0^1(\Omega)}
 \]
 と下から評価される。必要ならば$\epsilon > 0$を小さくすれば、
 次式が得られる。
 \[
  I_\lambda (v) \geq C \int_\Omega \left( \lvert Dv \rvert^2 + av^2
 \right) dx - C^\prime \left\| v \right\|_{H_0^1(\Omega)}^{p+1} \leq C
 \left\| v \right\|_{\kappa}^p - C^\prime \left\| v
 \right\|^{p+1}_{H_0^1(\Omega)}.
 \]
 $\dnorm_\kappa$と$\dnorm_{H_0^1(\Omega)}$が同値なノルムであることを考
 慮すると、
 \[
  I_\lambda \geq C^{\prime\prime} \left\| v \right\|_{H_0^1(\Omega)}^2
 - C^{\prime} \left\| v \right\|_{H_0^1(\Omega)}^{p+1}
 \]
 と導かれる。$C^{\prime\prime}, C^\prime > 0$は$v$によらない。
 $2 < p+1$であるから、$\delta > 0$を十分小さくとれば、$\rho =
 C^{\prime\prime} \delta^2 - C^{\prime} \delta^{p+1} > 0$とできる。
 つまり、\eqref{eq:delta_rho}が成立する。 \qedhere
\end{proof}

命題~\ref{prop:second_1}を証明する最後の準備として、次の補題を証明する。

\begin{lem} \label{lem:mountain_dec}
 $v \geq 0 ~\tin \Omega$および
 \eqref{eq:int_omega_bvp+1}をみたす
 $v \in H_0^1(\Omega)$について、次式が成立する。
 \begin{equation}
  \lim_{t \to \infty} I_\lambda(tv) =  -\infty. \label{eq:mountain_dec}
 \end{equation}
\end{lem}

\begin{proof}
 $t, s \geq 0$、$x \in \Omega$とする。
 \[
  G^\prime(t, s, x) - \frac{b(x)}{p+1} t^{p+1} = b(x) \left(
 \frac{1}{p+1}(t+s)^{p+1} - \frac{1}{p+1} s^{p+1} - s^p t -
 \frac{1}{p+1} t^{p+1} \right)
 \]
 である。右辺の括弧の中を$\alpha(s)$とおく。$\alpha$の$1$階導関数は、
 \[
  \alpha^\prime(s) = (t+s)^p - s^p - p s^{p-1} t
 \]
 である。
 右辺を$t$の関数とみると、テイラーの定理より、
 \[
  \alpha^\prime(s) = \frac{p(p-1)}{2}(s+ \theta t)^{p-1} t^2
 \]
 をみたす$0 < \theta < 1$が存在する。したがって、$\alpha^\prime(s)
 \geq 0$である。すなわち、$\alpha$は$s$についての単調増加関数である。
 ゆえに、
 \[
 G^\prime(t, s, x) - \frac{b(x)}{p+1} t^{p+1} \geq 
 G^\prime(t, 0, x) - \frac{b(x)}{p+1} t^{p+1} = 0
 \]
 が成立する。したがって、
 \[
  \int_\Omega G(tv, \underline{u}_\lambda) dx \geq
 \frac{t^{p+1}}{p+1} \int_\Omega b v^{p+1} dx
 \]
 である。ゆえに、$I_\lambda(tv)$は、次の通りに上から評価される。
 \[
  I_\lambda(tv) \leq \frac{t^2}{2} \int_\Omega \left( \lvert Dv
 \rvert^2 + av^2 \right) dx - \frac{t^{p+1}}{p+1} \int_\Omega bv^{p+1} dx.
 \]
 $2 \leq p+1$、及び、
 \eqref{eq:int_omega_bvp+1}より、\eqref{eq:mountain_dec}が成立する。\qedhere
\end{proof}

$(\mathrm{PS})$条件を課さない峠の定理を用いて、命題
~\ref{prop:second_1}を証明する。

\begin{proof}[命題~\ref{prop:second_1}]
 補題~\ref{lem:delta_rho}の$\delta > 0$について、
 $e = Tv_0$が$\left\| e \right\|_{H_0^1(\Omega)} > \delta$をみたすよう、
 $T > 0$を定める。この$T > 0$は、補題~\ref{lem:mountain_dec}により存在
 する。
 \[
   c = \inf_{\gamma \in \Gamma} \max_{s \in [0, 1]} I_\lambda ( \gamma
 (s ))
 \]
 と定める。ここで$\Gamma$は、$\gamma(0) = 0$を始点、$\gamma(1) = e$を
 終点とする$H_0^1(\Omega)$上の連続な道$\gamma$全体である。
 補題~\ref{lem:delta_rho}、および、\eqref{eq:ineq_S}より、
 補題~\ref{lem:delta_rho}の$\rho > 0$について、
 \[
  0 < \rho \leq c <    \frac{1}{N\left\| b
             \right\|_{L^\infty(\Omega)} ^{(N-2)/2}} S^{N/2} 
 \]
 が成立する。補題~\ref{lem:PS_c}.1より、$(\mathrm{PS})_c$条件は
 成立する。ゆえに、
 $(\mathrm{PS})$条件を課さない峠の定理~\cite{MR0370183}が
 適用できる。すると、補題~\ref{lem:PS_c}.1の(i), (ii)を
 みたす$H_0^1(\Omega)$の点列$\{ v_k \}$が得られる。
 補題~\ref{lem:PS_c}.2より、\ref{eq:prob_sec}の弱解$v \in
 H_0^1(\Omega)$が得られる。\qedhere
\end{proof}

% Local Variables:
% mode: yatex
% TeX-master: "main.tex"
% End:

\appendix

%#!platex main.tex
\bibliographystyle{jalpha}
\bibliography{bunken}

\end{document}
